

%Congratulations! You did a great job on the project, the writing and the presentation.  It was fun to watch!
%You should be proud of your work as you move on to better things (and after you get papers sent out).
 
%As I read through your dissertation, I highlighted some of the typos that I noticed – but I’m sure not all of them.  I have attached the PDF with highlights to point them out – this might help you make those minor revisions more efficiently.
 
%Below, I have pasted some comments/thoughts that I noted while reading your dissertation and during your defense.  I am sending them along to give you some things to think about  - maybe they will be helpful when writing papers or in other next steps.  Use them as you see fit.  I do not expect you to address these in the revised version of your dissertation (except for sharpening up your abstract).
 
%All the best!
 
%Jimmy
%====
%Some concern that your ‘conclusions’ are better characterized as ‘observations’ (e.g. discussion about mean model).
% switch mean model idea to conclusion.

%%%
% re-discuss mean model as mean measurement as an OBSERVATION not a conclusion.
%%%

%%
% What are your strongest ‘conclusions’
% spike shape and spike amplitude
%%

%%
%How can this guide your plans for publication?
%%

%Features that you focused on:
%%
%How important are they?  You selected some that other models have implicitly (or explicitly) deemed irrelevant.
%%
%%
%Reduced models:
%%You focused on the speed advantage – but are there others? 
% Interpretability of networks.
% self constraining governing equations good
% simplicity no back propogating AP.

%Discussed elitist forces in GA:
%How much did you explore this?
%How might this have impacted your work? Could it explain some of your results or problems that you experienced?

%Weighting different aspects of your cost function(s) – you discussed how this gets handled and some implications
%There are interactions in how parameters affect features – can these lead to implicit ‘preferences’ in which features get fit better than others or can it cause changes in some parameters to get emphasized over others? (i.e. does it put an implicit weighting in the cost function?)

%%
% small rheobase errors
%%

 
% Hello Russell, 

%The Graduate College Format Team has reviewed your document and provided feedback. All feedback is based on compliance with the ASU Format Manual (https://graduate.asu.edu/format-manual). Please see the list below for needed revisions:

1. On title page please only have one line between title and by-line, and by-line and your name. See format manual page 5 for guidance. 

2. On title page please list both Sharon Crook and Rick Gerkin as Co-Chair, as this is how they appear in your iPOS. If this is incorrect, you must make all committee changes BEFORE your defense. 

3. Please reformat page numbers to comport with the format manual. E.g., ABSTRACT should begin with page i, not 2. Update table of contents after you revise all page numbers. See the format manual for instruction on how pages should be numbered. 

4. Please add ABSTRACT section header. 

5. Please remove colored boxes, e.g., red boxes in table of contents. 

6. In TABLE OF CONTENTS please use all caps for chapter titles. 

7. In TABLE OF CONTENTS please use title case for all chapter subheadings and sub-subheadings. For help with title case you can go to titlecase.com. 

8. In TABLE OF CONTENTS please list APPENDICES or delete item if there are no appendices. 

9. In LIST OF TABLES please ensure all captions/subheadings are in title case. 

10. In LIST OF FIGUES please move Figure/Page to above listed items. 

11. In LIST OF FIGURES please ensure all captions/subheadings are in title case. 

12. Chapter 1 should begin with page 1. Update table of contents after this revision. 

%If applicable, you may wait until after your defense has been held to resubmit. Please be mindful of Graduate College deadlines, available at graduate.asu.edu. 

%It is important that students do not assume the modifications we have listed are the only items that should be addressed. Please go through the entire document to ensure the items we have listed do not occur elsewhere in your paper. In addition, please also make sure any modifications made to the document do not shift other sections or any tables/images/figures that the document may have.

%Once you have made the above changes, as well as any modifications your committee may require, please resubmit your document in Microsoft Word or PDF form to your IPOS for further review. You will be notified via email when it has been completed. Please note that format review should not be your sole resource for revisions. It is the students responsibility to ensure that their document is in accordance with the requirements and guidelines set by Graduate College.



%Please note that we may not approve the document for ProQuest until Graduate College has received your final Pass/Fail form.

%If you have further questions regarding format, please refer to the ASU Graduate College Format Manual at https://graduate.asu.edu/current-students/completing-your-degree/formatting-your-thesis-or-dissertation/asu-graduate-college. 


If you have any additional questions that are not addressed in the format manual, please email gradformat@asu.edu.   

Regards, 

Format Review Team
Graduate College