
%Test combinations that worked and did not work.
%note move the majority to the appendix
%Moved to appendix, will move back specific results



\subsection{Section 3.11}
Tests that were not always amenable to optimization:
\begin{itemize}
\item ThresholdTest
\item SpikeHalfWidth
\item Spike Amplitude
\end{itemize}
as discussed previously this is because of a threshold measurement that differs between cells. This may be more of a problem in certain regions of model parameter space, but the problem was general, it occurred in multiple models.
%Aim 1A, write something about tests overall.
%Overall the some 
Tests of static electrical properties amenable to optimization:
\begin{itemize}
\item FISlopeTests
\item Rheobase
\item Capacitance
\item Input Resistance
\item Time Constant 
\end{itemize}
%, , , , test worked but was conflicted. The tests that did not work. This is somewhere else.

Tests that worked within optimization:
Via \emph{Elephant} toolchain: FITests, Rheobase, Capacitance, Input Resistance, Time Constant, Resting Membrane Potential.
Via. 

When optimizing in the supra threshold regime Druckmann used:
(1) spike rate; (2) an accommodation index; (3) latency to first spike;(4) average AP overshoot; (5)average depth of after hyperpolarization (AHP); 
(6) average AP width similar to Druckman, when optimizing in the supra threshold regime.
When optimizing with reduced models, I found that the those 6 measurements were not enough to tightly constrain a fit, and additional constraints were helpful. In this work a minimum of 12 constraints were typically used:
\emph{EFEL} tool chain:
\begin{itemize}
\item AHP_depth
\item all_ISI_values,
\item Spikecount (similar to rate)
\item adaptation_index
\item mean_AP_amplitude  
\item min_voltage_between_spikes
\item minimum_voltage
\item peak_voltage
\item spike_half_width
\item time_to_first_spike
\item time_to_last_spike
\item voltage_base
\end{itemize}
