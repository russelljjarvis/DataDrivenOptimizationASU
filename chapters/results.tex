\begin{comment}

Is Optimization possible?
       %1a. Construction of tests from diverse experimental sources (I wrote the neuroelectro api and its use in neuronunit, and wrote the original Allen one
       
       Allen API.
       
       %, but you have put in work to create runnable tests from these and other sources).  This is in a sense a method, but you can still report that these tests are runnable, even outside of optimization.
       1b. Simulated data tests of optimization.  What works?  What doesn’t?  Why (briefly, saving some for discussion)?  NeuroElectro vs Allen also belong here, and fits in with 1a.  You should talk about model means vs means of models (or whatever we are calling it) here, if you have the results for it or think you can in 3 weeks.  You can talk about rippled error surfaces — this is such a deep technical detail that I wouldn’t spend a lot of time on it.  In other words it may be important but it will be almost impossible to follow even if written well.
2. Results for several optimized models.
    2a. First just do basic ones (like Izhikevich) for a few cell types, then you can close with L5PC.
    2b. The app (which supports 2a).
3. Study of variance between models and data.  The optimized models part of this section is predicated on result 1b (so that optimization results can be believed).  You already have the poster for this.
\end{comment}

In order to optimize reduced models, it was first necessary to develop optimizer-friendly implementations of these models.  I developed four different optimizer friendly models: Izhikevich model (spanning seven regimes), Adaptive Exponential Integrate and Fire model, and General Leaky Integrate and Fire Model, and a conductance based model.

The range of tests available for optimization in NeuronUnit was also incomplete, especially as it focused on passive membrane properties or the shapes of single action potentials.  In order for model output to reflect experimental data, it was also necessary that the timing and patterns of action potentials be assessed and tested. 

Therefore, I developed the ability for EFEL features to be calculated on NeuronUnit models, as well as tests based on FISlope in Allen celltypes data, and neuroelectro data sources.

This required both the extraction of novel features [EFEL] and novel test types: [ISI, AHP\_depth, adaption ratio etc.] Ultimately, I still found that some tests were intractable, because they depended on measured features, relative to some other changeable measurement inside the cell usually $max(\frac{dV_{m}}{dt})/10.$

Multi-core and or multi-threaded optimization requires that information about model properties be shareable between various processor threads. However, some common data types contain information that is not shareable between CPUs, and they must be translated into a globally sharable framework. I created a class, that could translate and envelope only shareable data to solve this problem.

In order to both control optimization parameters and visualize optimization results, I also developed a web application, the application requires no programming skill to use, and it allows a user of the to select among multiple cell specific constraints, and multiple model types. Once the user has specified enough parameters to define an optization job, the job can launch, and then return interactive results to the user when the optimization job is complete typically in minutes.
%
%\begin{comment}

%\subsection{Neocortical Layer 4/5 Pyramidal Cell Test Suite}
\subsubsection{2a}

%Direct Quote: "widening of the spike shape, decrease of the firing rate and change in the interspike interval distribution". %All these single unit waveform shapes increased their width with temperature.\cite{goldin2017temperature}

It is possible that the majority neuroelectro recordings of L5PC, spike width were conducted under room temperature as opposed to body temperature, and the relative cooling may have contracted their spike width.
\cite{goldin2017temperature}

\subsubsection{Conflicts between experimental constraints}

Conflicts between Rheobase value, and other static electrical measurements were observed in NeuroElectro and Allen Cell types measurements. When optimizing against NeuroElectro averaged measurements, and Allen Cell types single cell observations, it was common to experience a conflicted ability for the cell to satisfy all constraints simultaneously. 
Models seemed to have particular difficulty in recapitulating an accurate fit for rheobase, while simultaneously satisfying the larger set of fitness criteria: time constant, input resistance, capacitance and resting membrane potential. There was better agreement between firing rate versus current (FISlope) and the remainder of the electrical observations.

%\subsection{Section 2.1}
%
Note this is a cell from the l5 somatosensory rat, hind leg region, so it is probably not comparable to NeuroElectro Data.
\subsubsection{Performance of Layer 5 Prefrontal cortex Pyramidal Neuron on NeuronUnit tests of model data agreement}
\cite{van2016bluepyopt}
A suite of neuronunit tests containing the tests: rheobase value, membrane voltage time constant ($tau_{m})$, input resistance was computed. This multi-compartment, conductance based model served as a useful benchmark, for us to evaluate the relative performance of reduced model fits.

Significant development work went into making the model eligible to take NeuronUnit tests, by way of creating a specially dedicated NeuronUnit backend, to run this complicated conductance based multi-compartment model originating from the blue brain model \cite{markram2015reconstruction}. The intention is that by making this model interoperable with NeuronUnit the model will be able to amenable to optimization.

This elaborate biophysical model includes the backpropogating dendritic action potential.
\url{}



\begin{figure}
\begin{center}


\centering
\begin{subfigure}{.2}
  \centering
    \includegraphics[scale=0.5]{figures/correct_active_l5pc.png}
    \caption{A current injection sufficient for causing a single spike is applied for a whole second from $100ms-1100ms$}
  \label{fig:sub1}
\end{subfigure}

\centering
\begin{subfigure}{.2}
  \centering
    \includegraphics[scale=0.5]{figures/spike_shape.png}
    \caption{The spike shape is very brief in duration, and so it is worth zooming in for a closer look}
  \label{fig:sub1}
\end{subfigure}


\begin{subfigure}{.2}
  \centering
    \includegraphics[scale=0.5]{figures/correct_passive_l5pc.png}
    \caption{A current injection value of -$10pA$ is applied to the cell for the duration of $200ms-700ms$}
  \label{fig:sub2}
\end{subfigure}
\label{fig:test}
\end{center}
\end{figure}


A test suite was constructed using NeuroElectro for the layer 4/5 Prefrontal Cortex pyramidal cell, and we were able to evaluate this layer 5 PC cells against the criteria of the neuroelectro test suite. 


\begin{table}[ht]
\centering
\resizebox{\textwidth}{!}{
\begin{tabular}{lllll}
\toprule
{} & observations &   predictions & Z-Scores & SEM \\
\midrule
RheobaseTest                   &    213.85 pA &      225.0 pA &  0.06542 & 128.868981 \\
InputResistanceTest            &  120.67 Mohm &  50.7 megaohm &  -0.9013 & 27.928023 \\
TimeConstantTest               &     15.73 ms &      16.76 ms &   0.1409 & 2.562106 \\
CapacitanceTest                &    150.58 pF &     330.66 pF &    1.289 & 1.488977 \\
RestingPotentialTest           &    -68.25 mV &     -78.04 mV &   -1.499 & 44.038610 \\
InjectedCurrentAPWidthTest     &      1.21 ms &       0.15 ms &   -1.979 & 0.174910\\
InjectedCurrentAPAmplitudeTest &     80.44 mV &      89.58 mV &   0.7174 & 1.488977\\
InjectedCurrentAPThresholdTest &    -42.74 mV &     -59.57 mV &   -2.094 & 1.922930\\
\bottomrule
\end{tabular}}
\end{table}
The corresponding statistics were
$(\chi^{2},p_{value})=(13.5609360364, 0.093951963105254)$

It is worth noting that the layer 5 neocortical pyramidal neuron was very slow to dispatch relative to the reduced models developed in this thesis work. Where as a typical reduced model described here evaluated in the order of $2.5 ms$, this model on average took $5.74$s, for a single run and $34.8$s to solve for the models Rheobase, current. To be fair, the model was run without activating NEURONs variable time step cvode. However, even with variable time step applied to the differential equation solver the magnitude of the disparity is still still several $seconds:$ several $ ms$.

Neuroelectro lumps together, prefrontal cortex, somatosensory cortex and V1 PC cells together.

This model was pre-optimized to fit to spike times and F/I mainly, and so it should not necessarily be expected to fit other electrical characteristics of the cell. Only the rheobase test, and the time constant test seemed to fall within the range of biological plausibility.
None the less, this model remains a useful benchmark for reduced neuronal models.