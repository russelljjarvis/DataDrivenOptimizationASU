\chapter{Results}
In the results section I hope to help the reader address doubts about aim of this work, it is a good goal to produce more convincing models, to this end one should ask is improving models via optimization even possible?

In section \textbf{1a}. I discuss the construction of tests from diverse experimental sources. In section \textbf{1b} I discuss how and why it I used simulated data to verify that the optimizer functions correctly. I also discuss optimization techniques that work, and techniques work less well, and the reasons behind model fitting success and failure. 

In section \textbf{1c}, I compare the different collections of data driven tests, and then I discuss, an experiment where I use models to ascertain if some of the assumptions underlying the NeuroElectro data driven tests. 

After setting up the context by describing background considerations I then discuss the results of the majority of model fits performed in this work, although an exhaustive account of the optimized models are available in the appendix, I detail just a few typical results that are representative of whole.

I then discuss which model lead to the best fits overall, as I have ranked the quality of fits between models as if the models are competing against each other, I also come back to the issue of the best model performance in the discussion. %Later in the discussion, 

%    2a. First just do basic ones (like Izhikevich) for a few cell types, then you can close with L5PC.
%    2b. The app (which supports 2a).
Next I describe the study of variance between models and data, where I use sparse PCA, to identify the major sources of disagreement between models and experiments. Locating specific sources of model and experimental disagreement. I explain how locating and labeling model/experiment conflicts creates an opportunity for optimization jobs that which can seek to close the gap.

% The optimized models part of this section is predicated on result 1b (so that optimization results can be believed).  You already have the poster for this. I think this captures most of your results in three themes.  Other results which are really methods, like parallel rheobase search, can stay in the methods, and you will get credit for them there.


% , although some pre-existing APIs already existed, I needed to write new ones.  This is in a sense a method, but you can still report that these tests are runnable, even outside of optimization.
 %Why (briefly, saving some for discussion)?  NeuroElectro vs Allen also belong here, and fits in with 1a.  You should talk about model means vs means of models (or whatever we are calling it) 
       
  %here, if you have the results for it or think you can in 3 weeks.  
       %You can talk about rippled error surfaces — this is such a deep technical detail that I wouldn’t spend a lot of time on it.  In other words it may be important but it will be almost impossible to follow even if written well.

\section{Verification of the Optimizer}
\label{sec:optimizer-verification}
The reduced models I used are known to be too simple to precisely match all electrophysiological features exhibit by all cell types; this can be an advantage, as this also means that they are too inflexible to fit the ``noise" in the data, whether it be random thermal fluctuations or systematic recording errors.
These latter may be quite common; sites like NeuroElectro control only the extracted data is faithful to that reported in the publication, not that it reflects what was seen in the actual experiments, or that those experiments were carefully performed.
Therefore, if the first step were applying the optimizer to fit models to real biological data, we would have no idea whether any optimization failures (poorly fitting models) reflected the limitations of the biological data, the limitations of the underlying model, or the limitations of the optimizer algorithms.
%And before attempting to identify the magnitude or cause of poor fits, at least four factors that shape optimizer validation and design must be controlled for.
%These are: the informativeness of measurement errors, the performance of the optimizer, the quality of the data, and the quality of the models.

It is therefore necessary to create data-source independent ``ground truths", by simulating data from the models themselves, and using the optimizer to see if those model parameters can be recovered from that simulated data.
Because genetic algorithms (such as the DEAP \citep{DEAP_JMLR2012} implementation used here) has notably good performance in handling complicated error surfaces (such as Rastrigrin's function), I expect that my optimization frame work, which derives from it, should also be able to handle potentially complicated features spaces such as one would expect from the output of a complex dynamical system like a neuron.

I wrote an algorithm that senses the edges of defined model parameter boundaries, and defines uniformly distributed random numbers within those boundaries. Alternatively I could have chosen to draw numbers from uniformly distributed PRNGs, but doing so would risk testing only typical optimization cases and excluding edge cases. In figure (\ref{fig:})
% The algorithm I wrote to choose random model parameters samples from uniform distributions across parameter ranges, and the case above, shows a model in a fringe case.


\subsection{Verification Endpoints}
My optimizer was capable of identifying model parameterizations that nearly perfectly matched the parameters of the ``ground truth" models that generated the constraining simulated data (Figure \ref{fig:optimizer-verification-radar}).
Simulated output of the two models matched closely as well (as expected from similar model parameters) (Figures \ref{fig:optimizer-verification-traces-1}, \ref{fig:adexp_model_rebound_spike}, \ref{fig:optimizer-verification-traces-3}).
NeuronUnit-based electrophysiological features extracted from the traces shown in those figures, and from other traces simulated in the course of optimization, also showed a near-perfect match (i.e. Z-score of 0, indicating perfect optimization) (Table \ref{table:optimizer-verification}).

\begin{table}[ht]
\centering
\resizebox{\textwidth}{!}{
\begin{tabular}{llll}
\toprule
{} &    observations &     predictions & Z-Scores \\
\midrule
RheobaseTest         &         1.62 pA &         1.62 pA &        0 \\
TimeConstantTest     &        13.18 ms &        13.18 ms &        0 \\
RestingPotentialTest &       -77.43 mV &       -77.43 mV &        0 \\
InputResistanceTest  &  270.84 megaohm &  270.84 megaohm &        0 \\
CapacitanceTest      &        48.65 pF &        48.65 pF &        0 \\
FITest               &    7.51 Hz/pA &    7.51 Hz/pA &        0 \\
\bottomrule
\end{tabular}}
\caption[Optimizer verification table]{``Observed" and ``Predicted" electrophysiological features match nearly perfectly.
Generally, ``observed" would refer to observations taken from biological data, but here it refers to observations from simulations of the ground truth model using various stimuli.
Each test extracts an electrophysiological features from some stimulus.
``Predicted refers to the predicted values of these features from the optimized model.
Z-Scores of 0, self-evident from the equality of the first two columns, indicate perfect optimization for this set of features.}
\label{table:optimizer-verification}
\end{table}

\begin{figure}
    \centering
    \includegraphics[scale=0.75]{figures/radar_coordinates.png}
    \caption[Polor Plot of Optimizer Verification against Ground Truth]{\textbf{Radar Plot of Optimizer Verification against Ground Truth.} This radar plot shows the values of each parameter from the AdEx model for a randomly-chosen target parameterization (red) and the optimizer-discovered optimum parameterization (gray). 
    The optimizer did not observe the parameter values in red directly, instead it used error guidance described above to ``recover" the parameters, as follows: Before optimization output features of the red model, i.e. it simulated membrane potential traces using the parameters shown in red and found that the parameters shown in gray produce traces that were similar on some features encoded in NeuronUnit tests.
    
    The result of this process is two almost identical sets of parameters.
    Out of the 11 parameters in this model, there is only a trivially small discrepancy in $C_{M}$, $v-reset$ and $v-spike$.
    }
    \label{fig:optimizer-verification-radar}
\end{figure}

\begin{figure}
    \centering
    \includegraphics[scale=0.85]{figures/simulated_data_supra_threshold.png}
    \caption[Optimizer Verification Example 1]{\textbf{Optimizer Verification Example 1.} The orange trace shows the model waveform recovered during the process of optimization. Blue trace (not visible due to occlusion by orange trace) depicts the source of the electrical measurements that were used to guide optimization.
    Figure shows agreement in simulated response between the ground truth model and optimized model from Figure \ref{fig:optimizer-verification-radar}. Both the ground truth model and the optimized model are depicted for rheobase current injection values.}
\label{fig:optimizer-verification-traces-1}
\end{figure}

\begin{figure}
    \centering
    \includegraphics[scale=0.85]{figures/simulated_data_sub_threshold.png}
    \caption[Optimizer verification example 2]{\textbf{Optimizer Verification Example 2.}
    In this plot the blue trace represents the ground truth model, and the green trace represents the best solution obtain via optimization.
    \ref{fig:optimizer-verification-traces-1}. Here I used a constant negative current injection value of $-10pA$, applied between $100ms-600ms$. Because model parameters are assigned to the ground truth model randomly, an unsually small value of 'C' in the AdEx model has lead to the occurrence of multiple rebound bursts, after the inhibitory current was applied.
    
    This demonstrates that the match between the ground truth model and the optimized model is not limited to a single stimulus condition or exclusively to only typical model parameters.}
    \label{fig:adexp_model_rebound_spike}
\end{figure}


\begin{figure}
    \centering
    \includegraphics[scale=0.85]{figures/passive_model_agreement}
    \caption[Optimizer verification example 3]{
    This plot follows the same color coding conventions to the plot above. The main purpose of this plot, is to show similar sorts of results but in a different model type. As stated in the plot above an unusual value of capacitance was randomly selected as a model parameter in the AdEx model. In the current plot a more conventional set of model parameters were randomly sampeled in the Izhikevich model. The addition of this extra plot is to better represent typical
    model behavior under $10pA$ current injection.   
        
    \ref{fig:adexp_model_rebound_spike} 
    
    The same $-10.pA$ current injection stimulus is applied, this time in the Izhikevich model. 
    The model with the blue trace does rebound visibly by an additional $\sim0.1 mV$, reflecting a limitation of the features used for optimization to distinguish changes that are a) so small and b) irrelevant to the features that most physiologists would care about.
    Obtaining a near-perfect match on a detail such as this would require introducing a new test that measures rebound depolarization after hyperpolarization.}
    \label{fig:optimizer-verification-traces-3}
\end{figure}

In these results, only a small number of features were used (Time constant, Capacitance, Rheobase, Resting Potential and Input Resistance, and FI; interestingly only two of these involved a measurement of action potentials.
Other NeuronUnit tests that measure and judge models according to action potential \emph{width, amplitude, or threshold} did not participate in guiding optimization, because in the model-testing frame-work that I inherited when I entered the ICON laboratory the tests: \emph{width, amplitude, or threshold} where all contingent on variable, per-model rheobase current injection values.
When spike shape measurements are contingent on variable rheobase values, small rheobase approximation errors get amplified confounding the final high resolution stage of genetic algorithm search. Ordinarily this would not preclude achieving reasonable optimisation results, but this context is different. When seeking to recapitulate a ground truth very high precision of results is paramount, therefore I eschewed several possible tests of spike shape that amplify rheobase approximation errors, because unless they were re-written to use fixed current values they could not give me the level of precision I needed.

%% No, this is not the reason.
% because many reduced models simply cannot produce realistic looking waveforms without imputation -- currently false, actually AP shape is fine in these versions of Izhikevich and Adex and they have been shown to achieve experimental agreement in those measurements.
% The reason is the corrogations/ripples I have been talking about for the last while.

The AdExp model, for example, is far from an arbitrary waveform generators.
Results for additional models and parameterizations are given in the Appendix.

Failure of optimization verification, when it occurs, could be indicative of insufficient constraints.
By analogy, when solving a system of linear equations, finding a unique solution requires that the number of constraining equations is greater than the number of free variables you are solving for.
Similarly, in a genetic optimization algorithm we solve for unknown variables using stochastic principles, but the number of variables (i.e. model parameters) we can identify is still limited by the number of independent measurements of model output that we use.
In other words, assuming that the number of objectives in the multiobjective optimization problem, $NOBJ$ is greater than $NDIM$, the number of model parameters, optimization should be achievable.
Even when $NOBJ<NDIM$, such as in the examples above, optimization can still work due to correlations or redundancies in the model that lead the actual manifold on which models live to be of lower dimension than $NDIM$ itself.

\subsection{Verification Efficiency}
My optimizer can recover ground truth models from simulated data in several cases.
Does it do so efficiently?
How long does this optimization take, and does it get stuck exploring irrelevant regions of parameter space?

In Figure \ref{fig:optimizer-evolution}, I show how the optimized model converges towards the ground truth model over time.
In can require up to $200$ generations of parameter set evolution for tight convergence to be realized.
This takes approximately $10-20$ minutes on a my personal laptop.

% I have not done this (since numba on Adex, and IZHI existed) so I don't know the times for HPC. can be accelerated $XXXX$-fold on HPC architecture. 

%\begin{comment}
%\begin{figure}
%    \centering
%  \includegraphics{figures/simulated_data_stats.png}
%    \caption{Optimizer evolution, green line tracks evolution of best fitness, blue line average fitness, %orange line is worst fitness. GA params, $NGEN=200$, $MU=50$,$cxp=0.3$,$mupb=0.2$ from this plot can see that %genes are storing and exploiting information, $cxp+mutpb=0.5$, so $50\%$ of genes are conserved between %generations }
%    \label{fig:my_label}
%\end{figure}
%\end{comment}

\begin{figure}
    \centering
    \includegraphics[scale=0.7]{figures/optimizer_internal_validation}
    \caption[Optimizer error over generations]{Evolution of optimized model quality over generations.
    The green line tracks the lowest error in each generation of candidate models, the blue line shows the average error, orange line the highest error.
    Optimization hyperparameters were: $Number of Generations=200$, $Population Size=50$, $Crossover Probability=0.3$, $Mutation Probability=0.2$.
    The periodic jumps in the orange error represent mutation or crossover events in which poor-performing models were generated.
    These were not typically selected into the subsequent generation.
    Because the crossover and mutation probabilities only sum to 0.5, specifically (0.3cxpb, and 0.2mutpb), the remaining half of all parameter sets in each generation either filtered out by selection or, are simply carried over to the subsequent generation.
    Extensive hyperparameter tuning showed that this level of chromosome conservation was a good balance between exploration and exploitation (not shown.}
    \label{fig:optimizer-evolution}
\end{figure}

\subsection{Alternatives for Verification}
The ground truth is known independently of the optimizer, so one can test alternative strategies to see if there is a solution that represent a better match to ground truth than the one obtained through optimization.
For example, one can exhaustively search the solution space to look for the best model.
An exhaustive search might be a reasonable (but lazy) approach when there are only a small number of free parameters, however using a sampling grid of 25 distinct parameter values for each of N parameters mean that $25^{N}$ total models must be examined.
Even for a simple model ($N=5$) this represents nearly 10,000,000 model evaluations, and of course the problem gets much worse with each additional parameter.
25 values for each parameter may also be insufficient resolution when the parameter regime exhibiting the desired behavior is narrow.

\subsection{Implications of a verified optimizer}
Success here suggests that for reduced models, there is unlikely to be much degeneracy in model parameters, with distinct combinations of parameters producing identical simulated responses across a range of stimuli.
% Actually I did find some degenerecy, but I didn't find a way to systematically quantify it, because I spent too much time confounded by the ripple in surface problems.
This might also be expected from the motivation of reduced models, which is to identify and consolidate redundant or unimportant biophysical equations/parameters into a handful of key reduced model equations/parameters.
Importantly, I showed that the electrophysiological features used here, which correspond to measurements that can be made in real neurons, were sufficient for this task.

Confident that the optimizer can identify model parameterizations that can generate observed electrophysiological features \emph{in principle}, the focus of the remaining sections becomes the localization of other potential sources of model/data disagreement. 
 % ie is optimizaation possible?



%1a/b Is Optimization possible?
%       1a. Construction of tests from diverse experimental sources (I wrote the neuroelectro api and its use in neuronunit, and wrote the original Allen one, but you have put in work to create runnable tests from these and other sources).  This is in a sense a method, but you can still report that these tests are runnable, even outside of optimization.
%       1b. Simulated data tests of optimization.  What works?  What doesn’t?  Why (briefly, saving some for discussion)?  NeuroElectro vs Allen also belong here, and fits in with 1a.  You should talk about model means vs means of models (or whatever we are calling it) here, if you have the results for it or think you can in 3 weeks.  

       %You can talk about rippled error surfaces — this is such a deep technical detail that I wouldn’t spend a lot of time on it.  In other words it may be important but it will be almost impossible to follow even if written well.

Two Experimental Data Types: Spike Feature Experiments, versus Electro-physiology reports.



\subsection{Useable NeuronUnit Allen Constraints}
We created an Allen cell-types API, the API allows one to create neuronunit tests from a data base of experimental cellular recording waveforms. In order to create NeuronUnit tests, one must query the cell-types database, impose a new organization on the data, extract relevant features, and convert model features to NeuronUnit tests. A similar API was created in order to render BlueBrain Project waveforms, eligible for NeuronUnit testing, such that the both the BlueBrain \cite{toledo} and Allen Cell-type data could then guide model fitting. %Tests obtained via the Allen API, were designed to execute inside a NeuronUnit and BluePyOpt frameworks.  

%experiment to NeuronUnit test

One can see the APIs being creating novel NeuronUnit tests here \url{https://github.com/fun-zoological-computing/BluePyOpt/blob/master/examples/bpo_nu_fusion/allen_efel_nu_deployed_tests_only-thesis.ipynb}
%Construction of tests from diverse experimental sources 
%My allen API is perhaps scattered. Allen API.
%Which is the best model.
In order to optimize reduced models, it was first necessary to develop optimizer-friendly implementations of these models. As described in methods, I developed four different optimizer friendly models: Izhikevich model (spanning seven regimes), Adaptive Exponential Integrate and Fire model, and General Leaky Integrate and Fire Model, additionally I included a slower traditional conductance based model, to make some comparisons.

The range of tests available for optimization in NeuronUnit was also incomplete, as it was limited to the passive membrane properties or the shapes of single action potentials.  In order for model output to reflect experimental data, it was also necessary that the timing and patterns of action potentials be assessed and tested. 

I developed the ability for EFEL features to be calculated on NeuronUnit models, as well as tests based on FISlope in Allen celltypes which were coalesced into dedicated EFEL Allen Data test sets. This required both the extraction of novel features EFEL and novel test types: [ISI, AHP\_depth, adaption ratio etc.]. As described in the introduction, knowledge of error surfaces is helpful for garunteeing a good optimization result. Through human examination of resulting error surfaces, it was found that some of these tests were also intractable, because they depended on measurements that were changeable relative to some other changeable measurement inside the cell usually $max(\frac{dV_{m}}{dt})/10$.

Multi-core and or multi-threaded optimization requires that information about model properties be shareable across various processor threads. However, some common data types contain information that is not shareable between CPUs, and they must be translated into a globally sharable framework. I created a class Data Transport Container, that could contain be used to circulate only essential data, in a more primitive but shareable form. This class was also acculated useful helpful methods such as retreiving default model parameters.

%\subsection{Standard Error of the Mean}
%For
In order to both control optimization parameters and visualize optimization results, I also developed a web application, the application requires no programming skill to use, and it allows a user of the to select among multiple cell specific constraints, and multiple model types. Once the user has specified enough parameters to define an optization job, the job can launch, and then return interactive results to the user when the optimization job is complete typically in minutes.


The range of tests available for optimization in NeuronUnit was also incomplete, especially as it focused on passive membrane properties or the shapes of single action potentials.  In order for model output to reflect experimental data, it was also necessary that the timing and patterns of action potentials be assessed and tested.  Therefore, I developed the ability for EFEL features to be calculated on NeuronUnit models. 

This required both the extraction of novel features [Allen, EFEL, etc.] and novel test types: [ISI, AHP\_depth, adaption ratio etc.] Ultimately, I still found that some tests were intractable, because they depended on measured features, relative to some other changeable measurement inside the cell usually $max(\frac{dV_{m}}{dt})/10.$

Multi-core and or multi-threaded optimization requires that information about model properties be shareable between various processor threads. However, some common data types contain information that is not shareable between CPUs, and they must be translated into a globally sharable framework. I created a class, that could translate and envelope only shareable data to solve this problem.

In order to both control optimization parameters and visualize optimization results, I also developed a web application, the application requires no programming skill to use, and it allows a user of the to select among multiple cell specific constraints, and multiple model types. 
%
%\begin{comment}


\subsubsection{Is Optimization possible?}
Not all data-sets where equally conducive towards optimization with reduced cells. In particular the cerebellum purkinje cell demands a very high current injection value to elicit a rheobase spike, $680pA$, and the reduced cells were generally not able to model this inhibited behavior.

Previously, the INCF created a model data fitting competition, were participants competed to submit models, that were best able to fit membrane voltage from a variety of suprathreshold current injection experiments. In this competition, unlike in this work the best models had to predict spike time. Within the scope of this project, post-optimization, ranking of the different models in terms of their ability to explain the most in-vivo experiments. 

There are some data sources, that no models seem to do well on, however, the majority of experiments cause good fits, and the models should compete with each other.

Over all the tests, including challenging Cerebellar Purkinje cell, and olfactory bulb mitral cell, actually the Izhikevich model was the top ranking cell, as it was able to perform the best all-round fits. However if you were to remove those two data sets, and evaluated on the remaining 6-8 out of data sets, the Adaptive Exponential model would be a better fit.


%
%\begin{comment}

%\subsection{Neocortical Layer 4/5 Pyramidal Cell Test Suite}
\subsubsection{2a}

%Direct Quote: "widening of the spike shape, decrease of the firing rate and change in the interspike interval distribution". %All these single unit waveform shapes increased their width with temperature.\cite{goldin2017temperature}

It is possible that the majority neuroelectro recordings of L5PC, spike width were conducted under room temperature as opposed to body temperature, and the relative cooling may have contracted their spike width.
\cite{goldin2017temperature}

\subsubsection{Conflicts between experimental constraints}

Conflicts between Rheobase value, and other static electrical measurements were observed in NeuroElectro and Allen Cell types measurements. When optimizing against NeuroElectro averaged measurements, and Allen Cell types single cell observations, it was common to experience a conflicted ability for the cell to satisfy all constraints simultaneously. 
Models seemed to have particular difficulty in recapitulating an accurate fit for rheobase, while simultaneously satisfying the larger set of fitness criteria: time constant, input resistance, capacitance and resting membrane potential. There was better agreement between firing rate versus current (FISlope) and the remainder of the electrical observations.

%\subsection{Section 2.1}
%
Note this is a cell from the l5 somatosensory rat, hind leg region, so it is probably not comparable to NeuroElectro Data.
\subsubsection{Performance of Layer 5 Prefrontal cortex Pyramidal Neuron on NeuronUnit tests of model data agreement}
\cite{van2016bluepyopt}
A suite of neuronunit tests containing the tests: rheobase value, membrane voltage time constant ($tau_{m})$, input resistance was computed. This multi-compartment, conductance based model served as a useful benchmark, for us to evaluate the relative performance of reduced model fits.

Significant development work went into making the model eligible to take NeuronUnit tests, by way of creating a specially dedicated NeuronUnit backend, to run this complicated conductance based multi-compartment model originating from the blue brain model \cite{markram2015reconstruction}. The intention is that by making this model interoperable with NeuronUnit the model will be able to amenable to optimization.


\subsection{The Experimental Measurements}

\begin{table}[ht]
\centering
\resizebox{\textwidth}{!}{
\begin{tabular}{lllllllll}
\toprule
name & Hippocampus CA1 pyramidal cell & Cerebellum Purkinje cell & Neocortex pyramidal cell layer 5-6 &      olf\_mit &      623960880 &      623893177 &      471819401 &      482493761 \\
\midrule
RheobaseTest                   &                      189.24 pA &                680.79 pA &                          213.85 pA &          NaN &        70.0 pA &       190.0 pA &       190.0 pA &        70.0 pA \\
InputResistanceTest            &                    107.08 Mohm &              142.06 Mohm &                        120.67 Mohm &  130.08 Mohm &  241.0 megaohm &  136.0 megaohm &  132.0 megaohm &  132.0 megaohm \\
TimeConstantTest               &                        24.5 ms &                      NaN &                           15.73 ms &     24.48 ms &        23.8 ms &        27.8 ms &        13.8 ms &        24.4 ms \\
CapacitanceTest                &                        89.8 pF &                620.27 pF &                          150.58 pF &    235.75 pF &            NaN &            NaN &            NaN &            NaN \\
RestingPotentialTest           &                      -65.23 mV &                -61.59 mV &                          -68.25 mV &    -58.14 mV &       -65.1 mV &       -77.0 mV &       -77.5 mV &       -71.6 mV \\
InjectedCurrentAPWidthTest     &                        1.32 ms &                  0.41 ms &                            1.21 ms &      1.61 ms &            NaN &            NaN &            NaN &            NaN \\
InjectedCurrentAPAmplitudeTest &                       86.36 mV &                 71.23 mV &                           80.44 mV &      68.4 mV &            NaN &            NaN &            NaN &            NaN \\
InjectedCurrentAPThresholdTest &                       -47.6 mV &                -46.89 mV &                          -42.74 mV &     -38.9 mV &            NaN &            NaN &            NaN &            NaN \\
FITest                         &                            NaN &                      NaN &                         0.05 Hz/pA &          NaN &     0.18 Hz/pA &     0.12 Hz/pA &     0.18 Hz/pA &     0.09 Hz/pA \\
\bottomrule
\end{tabular}}
\end{table}


This elaborate biophysical model includes the backpropogating dendritic action potential.
\url{https://github.com/BlueBrain/BluePyOpt/blob/master/examples/l5pc/L5PC.ipynb}
\url{https://github.com/social-hacks-for-mental-health/BluePyOpt/blob/master/examples/l5pc/L5PC.ipynb}


\begin{figure}
\begin{center}


\centering
\begin{subfigure}{.2}
  \centering
    \includegraphics[scale=0.5]{figures/correct_active_l5pc.png}
    \caption{A current injection sufficient for causing a single spike is applied for a whole second from $100ms-1100ms$}
  \label{fig:sub1}
\end{subfigure}

\centering
\begin{subfigure}{.2}
  \centering
    \includegraphics[scale=0.5]{figures/spike_shape.png}
    \caption{The spike shape is very brief in duration, and so it is worth zooming in for a closer look}
  \label{fig:sub1}
\end{subfigure}


\begin{subfigure}{.2}
  \centering
    \includegraphics[scale=0.5]{figures/correct_passive_l5pc.png}
    \caption{A current injection value of -$10pA$ is applied to the cell for the duration of $200ms-700ms$}
  \label{fig:sub2}
\end{subfigure}
\label{fig:test}
\end{center}
\end{figure}


A test suite was constructed using NeuroElectro for the layer 4/5 Prefrontal Cortex pyramidal cell, and we were able to evaluate this layer 5 PC cells against the criteria of the neuroelectro test suite. 


\begin{table}[ht]
\centering
\resizebox{\textwidth}{!}{
\begin{tabular}{lllll}
\toprule
{} & observations &   predictions & Z-Scores & SEM \\
\midrule
RheobaseTest                   &    213.85 pA &      225.0 pA &  0.06542 & 128.868981 \\
InputResistanceTest            &  120.67 Mohm &  50.7 megaohm &  -0.9013 & 27.928023 \\
TimeConstantTest               &     15.73 ms &      16.76 ms &   0.1409 & 2.562106 \\
CapacitanceTest                &    150.58 pF &     330.66 pF &    1.289 & 1.488977 \\
RestingPotentialTest           &    -68.25 mV &     -78.04 mV &   -1.499 & 44.038610 \\
InjectedCurrentAPWidthTest     &      1.21 ms &       0.15 ms &   -1.979 & 0.174910\\
InjectedCurrentAPAmplitudeTest &     80.44 mV &      89.58 mV &   0.7174 & 1.488977\\
InjectedCurrentAPThresholdTest &    -42.74 mV &     -59.57 mV &   -2.094 & 1.922930\\
\bottomrule
\end{tabular}}
\end{table}
The corresponding statistics were
$(\chi^{2},p_{value})=(13.5609360364, 0.093951963105254)$

It is worth noting that the layer 5 neocortical pyramidal neuron was very slow to dispatch relative to the reduced models developed in this thesis work. Where as a typical reduced model described here evaluated in the order of $2.5 ms$, this model on average took $5.74$s, for a single run and $34.8$s to solve for the models Rheobase, current. To be fair, the model was run without activating NEURONs variable time step cvode. However, even with variable time step applied to the differential equation solver the magnitude of the disparity is still still several $seconds:$ several $ ms$. Neuroelectro lumps together, prefrontal cortex, somatosensory cortex and V1 PC cells together into a generic frontal cortex pyramidal cell model.

%The l5pc model was pre-optimized to fit to spike times and F/I mainly, and so it should not necessarily be expected to fit other electrical characteristics of the cell. Only the rheobase test, and the time constant test seemed to fall within the range of biological plausibility. None the less, this model remains a useful benchmark for reduced neuronal models.



%It was desirable to include this extended range of Izhikevich model behavior

%However, as noted in the introductory material, it i 


%Previously I mentioned neuronal modelling competitions I have optimize every model against the same data sets in order to assess overall which model is better able to fit to diverse data sets.






%Flat regions of error surface are uninformative.
%Tests that I curated from Allen Cell types lead to some models being under-constrained about spike width. The consequences of  under-restrained were not obvious/

%they were revealed by graphs in a virtual experiments were appropriate models were elicited to spike. The lack of constraint was easily rectified, by imposing a specific spike width constraint on Adaptive exponential models, however, unexpectedly in this context, the models $\chi^{2}$ increased dramatically and biological plausibility plummeted, in all except one test. To  paraphrase, the adaptive exponential models had found an unexpected way to cheat tests, by taking advantage of a lack of constraints in an unconstrained area, ie adopting implausibly long spike widths made the AdExp models exceptionally good at passive tests.

%In type of standard, the NeuronUnit tests, themselves act as the final judge of model quality, in the absence of a spike width tests, many AdExp models were able to get very good fits on against supplied constraints, but plots of actual spike shape looked very unnatural, as spike width lasted $>=$ 6ms. Applying extra standards beyond the NeuronUnit tests creates a dilemna. As all the GLIF models, presented unusual spike shapes.

%
%Although the Rheobase was the cause of  significantly impeding error, it was not so much of a problem to include this test itself, it was more of a problem to include tests that where contingent on its value.

%error which could propogate into other tests that depend on its value, 


% From the plots below one can see that 



\subsection{Experiment Fitted Model Results on Reported Data Types} 
Because some features derived from the data were incompatible or unreliable, it was necessary to created additional NeuronUnit tests from other feature extraction libraries, as described in the Methods.
Ultimately, over four different Allen cell-type summaries, and four different different cell type  electrical reported measurements, we created eight unique data sets, and then converted these eight data sets to neuronunit test suites. Additionally, to explore if FITests, and Rheobase fitted better on their own an additional four non unique test sets where also created. The entire set of tests is presented tabular form (\ref{table:tests_derived_from_reports}) below. See \ref{sec:allen_report_data} in the Appendix for the URLS for the data sources.

%id's:623960880,623893177,471819401,482493761

\begin{table}
\resizebox{\textwidth}{!}{
\begin{tabular}{lllllllllllll}
\toprule
name & Hippocampus CA1 pyramidal cell & Cerebellum Purkinje cell & Neocortex pyramidal cell layer 5-6 &      olf\_mit &      623960880 &      623893177 &      471819401 &      482493761 &  6239608801 &  6238931771 &  4718194011 &  4824937611 \\
\midrule
RheobaseTest                   &                      189.24 pA &                680.79 pA &                          213.85 pA &          NaN &        70.0 pA &       190.0 pA &       190.0 pA &        70.0 pA &     70.0 pA &    190.0 pA &    190.0 pA &     70.0 pA \\
InputResistanceTest            &                    107.08 Mohm &              142.06 Mohm &                        120.67 Mohm &  130.08 Mohm &  241.0 megaohm &  136.0 megaohm &  132.0 megaohm &  132.0 megaohm &         NaN &         NaN &         NaN &         NaN \\
TimeConstantTest               &                        24.5 ms &                      NaN &                           15.73 ms &     24.48 ms &        23.8 ms &        27.8 ms &        13.8 ms &        24.4 ms &         NaN &         NaN &         NaN &         NaN \\
CapacitanceTest                &                        89.8 pF &                620.27 pF &                          150.58 pF &    235.75 pF &            NaN &            NaN &            NaN &            NaN &         NaN &         NaN &         NaN &         NaN \\
RestingPotentialTest           &                      -65.23 mV &                -61.59 mV &                          -68.25 mV &    -58.14 mV &       -65.1 mV &       -77.0 mV &       -77.5 mV &       -71.6 mV &         NaN &         NaN &         NaN &         NaN \\
InjectedCurrentAPWidthTest     &                        1.32 ms &                  0.41 ms &                            1.21 ms &      1.61 ms &            NaN &            NaN &            NaN &            NaN &         NaN &         NaN &         NaN &         NaN \\
InjectedCurrentAPAmplitudeTest &                       86.36 mV &                 71.23 mV &                           80.44 mV &      68.4 mV &            NaN &            NaN &            NaN &            NaN &         NaN &         NaN &         NaN &         NaN \\
InjectedCurrentAPThresholdTest &                       -47.6 mV &                -46.89 mV &                          -42.74 mV &     -38.9 mV &            NaN &            NaN &            NaN &            NaN &         NaN &         NaN &         NaN &         NaN \\
FITest                         &                            NaN &                      NaN &                                NaN &          NaN &            NaN &            NaN &            NaN &            NaN &  0.18 Hz/pA &  0.12 Hz/pA &  0.18 Hz/pA &  0.09 Hz/pA \\
\bottomrule
\end{tabular}}
\end{table}
\label{table:tests_derived_from_reports}

I then used three different models (AdEx, Izhikevich, Conductance based) and optimized them using these test suites. I attempted to also use GLIF models here, although test results often looked convincing, GLIF model waveforms looked strange.
The result is $3 \times 8 = 24$ model-data combinations.
For each each member of this $24$ element matrix we wanted to know if the fitted model behaved in a biological plausible manner, we were interested to know if fitted models were convincing mimics of \emph{in vivo} cells, at least with respect to the measurements models were trained to fit.
In the process of coalescing results the single 24 element matrix, the matrix was broken into two: one 24 element matrix %\ref{tab:main_chi2} 
and a smaller matrix consisting of only the conductance based model test combinations that executed without failure . 
%\ref{tab:HH_chi2}

\subsection{Across Model Performance Comparisons}

\begin{table}
\resizebox{\textwidth}{!}{
\begin{tabular}{cccc}
\toprule
 web-source & cell id & subset of co-listed tests & sample type  \\
\midrule
Allen & 48249376[1] & Y &     n=1 \\
Allen & 47181940[1] & Y &     n=1 \\
Allen & 623893177[1] & Y &     n=1 \\
Allen & 623960880[1] & Y &   n=1 \\  
Allen & 482493761 & N &     n=1 \\  
Allen & 471819401 & N &     n=1 \\
Allen & 623893177 & N &    n=1 \\
Allen & 623960880 & N &     n=1 \\
NeuroElectro & Olfactory Mitral Cell & N &     pooled samples \\
NeuroElectro & Neocortex pyramidal cell layer 5-6 &      N &     pooled samples \\
NeuroElectro & Cerebellum Purkinje cell & N &     pooled samples \\
NeuroElectro & Hippocampus CA1 pyramidal cell & N &     pooled samples \\
\bottomrule
\end{tabular}}

\caption[Properties of Different Data Driven Tests Used]{Properties of of the different data driven tests used. These where used to fit at or below threshold experimental cells. This table acts as a legend to assist in the interpretation of proceeding tables}

\label{tab:main_chi2}

\end{table}


\begin{table}
\resizebox{\textwidth}{!}{
\begin{tabular}{cccc}
\toprule
 model type &         experiment cell & $\chi^{2}$ & p-value  \\
\midrule
        IZHI &              4824937611 &     0.034791 &  1.000000e+00 \\
        IZHI &              4718194011 &     0.051691 &  1.000000e+00 \\
        IZHI &              6238931771 &     0.086530 &  9.999999e-01 \\
        IZHI &              6239608801 &     0.001550 &  1.000000e+00 \\
        IZHI &              482493761 &     1.292924 &  9.956362e-01 \\
        IZHI &              471819401 &     1.443618 &  9.936050e-01 \\
        IZHI &              623893177 &     1.334789 &  9.951233e-01 \\
        IZHI &              623960880 &     1.014464 &  9.981553e-01 \\
        IZHI &              Olfactory Bulb Mitral Cell &  6915.484007 &  0.000000e+00 \\
        IZHI &  Neocortex pyramidal cell layer 5-6 &     2.443396 &  9.643182e-01 \\
        IZHI &            Cerebellum Purkinje cell &    19.885113 &  1.077956e-02 \\
        IZHI &      Hippocampus CA1 pyramidal cell &     1.273070 &  9.958661e-01 \\
       ADEXP &              4824937611 &     0.000017 &  1.000000e+00 \\
       ADEXP &              4718194011 &     0.011029 &  1.000000e+00 \\
       ADEXP &                          6238931771 &     0.005062 &  1.000000e+00 \\
       ADEXP &              6239608801 &     0.000083 &  1.000000e+00 \\
       ADEXP &              482493761 &    86.949529 &  1.887379e-15 \\
       ADEXP &              471819401 &     0.370856 &  9.999575e-01 \\
       ADEXP &              623893177 &     0.273030 &  9.999870e-01 \\
       ADEXP &              623960880 &     0.140222 &  9.999990e-01 \\
       ADEXP &              olf\_mit &    10.353878 &  2.410614e-01 \\
       ADEXP & Neocortex pyramidal cell layer 5-6 &     0.013262 &  1.000000e+00 \\
       ADEXP &            Cerebellum Purkinje cell &   353.692447 &  0.000000e+00 \\
      ADEXP &      Hippocampus CA1 pyramidal cell &     0.735616 &  9.994308e-01 \\
\bottomrule
\end{tabular}}

\caption[Comparable $\chi^{2}$ for optimized results of AdEx and Izhikevich models]{Comparable $\chi^{2}$ for optimized results of the conductance based model. Not all models could be evaluated, as optimization took a long time.}

\label{tab:main_chi2}

\end{table}

\begin{table}
\resizebox{\textwidth}{!}{
\begin{tabular}{lllrr}
%\begin{tabular}{cccc}
\toprule
model type &                            exp-cell &   $ \chi^{2} $ & p-value \\
\midrule
conductance model & Hippocampus CA1 pyramidal cell & 17.21 &  0.027 \\
conductance model & olf-mit & 26487.51 &  0.0 \\
conductance model & Neo cortex pyramidal cell layer 5-6 &  2.56 & 0.95 \\
conductance model & 4824937611 &   2.17 &  0.97 \\ 
conductance model & 471819401 &  0.870 &  0.99 \\
conductance model & 482493761 &  0.036 &  0.99 \\
conductance model & 6238931771 & 1.441259 & 0.993641 \\
\bottomrule
\end{tabular} 
}
\caption[$\chi^{2}$ for Comparing Optimized Results of the Conductance Based Model]{Not all models could be evaluated as optimization took a long time or lead to failure. Of the 12 model test combinations only 6 are known.}
\label{tab:HH_chi2}
\end{table}



%See appendix:\ref{table:static_electrical_properties}
As stated in the Methods, I use the $\chi^2$ statistic as a summary of optimized model quality, with smaller values reflecting fits that better recapitulate the biological experimental data, and non-significant p-values--lack of evidence that the model disagrees with that data--as evidence of success.
The Izhikevich Model and a Conductance-Based Point Neuron Model were able to achieve such small $\chi^2$ statistics (and non-significant p-values) when seven or eight of the tests were considered together.
For example the Izhikevich model fitted to a Hippocampus CA1 pyramidal cell data achieved ($\chi^2$, p-value) = (2.13, 0.98), and the Olfactory Bulb Mitral cell achieved ($\chi^2$, p-value) = (2.02, 0.98).
A model whose every feature was exactly equal to the mean observed in the experimental data would have all Z-scores equal to 0, a $\chi^2$ statistic of 0, and a p-value of 1, by definition.
Thus an extremely high p-value (such as those above) is evidence that the optimized model is much closer to the mean of the data distribution than a random experimental neuron.
This is exactly the result one would expect from successful optimization.

\subsection{Sources of Optimization Failures}
When optimization was not successful, was this due to a fundamental inability of a given reduced model to represent the behavior of a given neuron type?
In order to make this claim, I must first had to rule out alternative possibilities.

\subsubsection{Distributions not well summarized by the mean}
The optimizer fits to the mean of a feature value, but as shown in section \ref{sec:neuroelectro}, the mean value of a feature is in some cases a misleading summary of the typical values.
For example, the olfactory bulb Mitral cell exhibited bi-modal feature distributions (possibly due to lumping with the Tufted cell, a distinction that may not have been appreciated when the data was originally collected).
Beyond that, the Neuroelectro data reflects a mean over different laboratories, animals, and recording epochs. The mean of a population can often be a robust summary, however there are circumstances when this is not true.
I plotted all the feature distributions for all the neurons used here (see Appendix), examined these by eye and made note of those where the mean was not a good summary of the typical value (due to multimodality, extreme skew, or small sample size).
Note that this only applies to the Neuroelectro data; the other data sources report features for single instances of neurons, so the model being fit is a model of that specific neuron, not of a neuron type more generally.

\subsubsection{Distributions with an uncertain mean}
The NeuroElectro data represented distributions over cells of the same nominal type.
For some cell types, reduced models were hard to optimize against these data even when the distributions are normal-like, and thus the mean and the mode are well-matched.
In order to understand whether the source of these difficulties was in the data themselves, I examined the standard error of the mean (SEM) for each feature.
Like the standard deviation, the SEM is a prediction of uncertainty in each measurement, but it reflects uncertainty about the value of the mean itself, rather than simply the variability in the measurement across neurons of the same type.
A large SEM might reflect such variability, or simply a small sample size. 
In either case, a large SEM means that the optimization target may not reflect the true properties of a typical neuron of that type.
These SEM values, for several cells and electrophysiological features, are shown in Table \ref{table:neuroelectro-sem}.

\begin{table}
\resizebox{\textwidth}{!}{
\begin{tabular}{lrrrrrrr}
\toprule
{} &  Rheobase &  SpikeThreshold &  SpikeHalfWidth &  SpikeAmplitude &  MembraneTimeConstant &  RestingMembranePotential &  InputResistance \\
Neuron Type                              &           &                 &                 &                 &                       &                           &                  \\
\midrule
Hippocampus CA1 pyramidal cell     &    122.88 &            1.85 &            0.12 &            3.68 &                  3.88 &                      0.72 &            12.57 \\
Olfactory bulb (main) mitral cell  &       NaN &            5.69 &            0.12 &            2.83 &                  5.42 &                      1.39 &            20.17 \\
Cerebellum Purkinje cell           &    419.81 &            2.00 &            0.05 &            0.57 &                   NaN &                      3.69 &            19.26 \\
Neocortex pyramidal cell layer 5-6 &    128.87 &            1.92 &            0.17 &            1.49 &                  2.56 &                      1.84 &            27.93 \\
\bottomrule
\end{tabular}
}
\caption[Standard Error of the Mean across NeuroElectro Data Sources]{A table of SEM values that describe dispersal in many of the different measurements I used to fit the optimized models to (NeuroElectro only).}
\label{table:neuroelectro-sem}
\end{table}

%\begin{comment}
%Below I show distributions for  Time Constant, Input Resistance, Capacitance, Rheobase, Resting Membrane %Potential, bimodal distributions did not apply, so we could rule out inaccurate data as a reason for poor %model performance.

%\begin{comment}

%There is no reason to believe that reduced neural models could not be made to fit inaccurate neural recordings %as well as real ones, if the fiction is just caused by noise, then it is still possible that hypothetical %spurious values would still be in reach of the Izhikevich model. The izhikevich model can be made to generate %some physiologically implausible waveform shapes. 
% a different question, of are the models arbitary waveform generators? 

%Besides even if the data was wrong, it wouldn't necessarily follow that models couldn't reproduce the %inaccurate data. Instead what we see is, that models can fit one type of experimental measurement at a time, %but they can't fit all measurements at once. This result suggests that model flexibility is the most %fundamental cause of modest, model experiment disagreement.


%In light of this result, one might ask, i
%If reduced models are not excellent at fitting data, are brain simulations really that much more realistic, %when we use data driven fitted models in the place of generic model parameters? 

% To answer this question we would need to run brain simulations. From my own experience adding in realistic %cells to pre-existing network topologies requires that the network be re-tuned to demonstrate tonic firing %with realistic CV again.

%If data driven model fits lead to models that fit one measurement better than others, which fitness criteria %will lead to the greatest consequence for network simulations?
%\end{comment}


%to answer that question we have created some large

%\subsection{Limitations of Existing Approaches} 
%Existing community supported simulated models were problem ridden, and our own custom methods were used as work arounds.
%data we were using was
% NEURON version of Izhi model is not as fast as one might expect, this may because of the way we tried to implement the model, by creating and destroying HOC module instances, that contain the model. 

%A
% Nonetheless, 

% This should not go in the general introduction, but in the intro to the appropriate section of the results where you use those models. 
\subsection{Mean model == Mean measurement?}
For a selection of three model measurements:

Over rheobase, input resistance, capacitance, and time constant, we took measurements by instancing two different models at  two different loctations in parameter space. With these two different models we took seperate measurements of all the appropriate model properties, then we averaged these measurements together, to get the 'mean measurements', secondly we created a 'mean model' by averaging the two points together in parameter space and instancing the 'mean parameter model'. We then used the mean model to create a third set of measurements.

When we have mean measurements, and also mean model measurements, it enables us to understand if bi-modal distributions of measurements in experimental data can pose a problem for optimization of cells.

Previously in methods \ref{section:nelectro} , we graphically inspected the neuroelectro data sources closely, in order to assess each measurements distribution. We revealed Bi-modal distributions in input resistance, and cell membrane capacitance, but we do not yet know if it is invalid to fit to the mean of a bi-modal distribution. Consider a  cell class which had an underlying bimodal distribution for input resistance. An individual cell from this class produced measurements for input resistance midway between two modes for input resistance. Its entirely concievable that that this mean cell would produce measurements that were also the mean of the two modes, however, we cannot assume that there this to be true, it seems equally likely that this midway cell produces a measurement for input resistance, that is significantly higher or lower than the mean of the two modes. To bolster that this non linear behavior could be a problem with in-silico models of in-vivo experiments. We created virtual experiments to expose non-linear behavior between two  close, but different sets of model parameters.

%set about establishing that it is a problem in modelling space.
%The genetic algorithm approach, of recombining model parameters to sample error surface is a similar concept. We do not naively interpolate using midway points, becuase we don't expect that small changes in model parameters to have linear effects.
%to sampling models 

We expect the Izhikitich model, and the adaptive exponential models to support "regime" change, that is we know that there are regions in parameter space that where when entered cell behavior becomes fundamentally different. For instance in the Izhikevich model, some regions support tonic-bursting and other regions support chattering. 

\begin{figure}
    \centering
    \includegraphics{figures/mean_model_mean_measure_ment_params.png}
    \caption{Caption}
    \label{fig:my_label}
\end{figure}

\begin{figure}
    \centering
    \includegraphics{figures/mean_model_mean_test.png}
    \caption{Caption}
    \label{fig:my_label}
\end{figure}

\begin{figure}
    \centering
    \includegraphics{figures/mean_model_mean_test2.png}
    \caption{Caption}
    \label{fig:my_label}
\end{figure}

explore if this was a problem for models as well as experimental cells.

%\subsection{Neocortical Layer 4/5 Pyramidal Cell Test Suite}

%Direct Quote: "widening of the spike shape, decrease of the firing rate and change in the interspike interval distribution". %All these single unit waveform shapes increased their width with temperature.\cite{goldin2017temperature}



%\subsubsection{%\subsection{Section 2.1}
%
%2. Results for several optimized models.
%    2a. First just do basic ones (like Izhikevich) for a few cell types, then you can close with L5PC.
%    2b. The app (which supports 2a).

%\subsubsection{2a}

\subsubsection{Performance of Layer 5 Pyramidal Neuron Somatosensory model on NeuronUnit tests}
%Hind-limb
\cite{van2016bluepyopt}
%To understand the validity of model re-purposing, we tested a model constrained on Layer 5 Somatosensory cortex Pyramidal neurons. 
A suite of neuronunit tests containing the tests: rheobase value, membrane voltage time constant ($tau_{m})$, input resistance was computed. This multi-compartment, conductance based model originating from the blue brain project \cite{markram2015reconstruction}. This model served as a useful benchmark, for us to evaluate the relative performance of reduced model fits. 

% somatosensory cortex, or cell from the l5 somatosensory rat, hind leg region, so it is probably not comparable to NeuroElectro Data.

Significant development work went into making the model eligible to take NeuronUnit tests, and amenable to NeuronUnit driven optimization, to make this complicated conductance based multi-compartment model interoperable with the neuronunit test judging paradigm. The intention is that by making this model inter-operable with NeuronUnit the model will be able to amenable to different, optimization. 

Before optimization:
\begin{figure}
    \centering
    \includegraphics{figures/l5pc_before_opt}
    \caption{Caption}
    \label{fig:my_label}
\end{figure}
After optimization

\begin{figure}
    \centering
    \includegraphics{figures/l5pc}
    \caption{Membrane potential versus time in the layer 5 pyramidal neuron}
    \label{fig:after_optimization}
\end{figure}

There were two points to this exercise, first, was to show that multi-compartment conductance based models, are very slow to evaluate, and the second point was, that with or without enough time, the results are not necessarily the best, in the realistic model anyway.

\begin{figure}
    \centering
    \includegraphics{figures/parameters_opt_l5pc.png}
    \caption{Caption}
    \label{fig:parameters}
\end{figure}

This elaborate biophysical model is the opposite of the reduced models focused on here.  includes the backpropogating dendritic action potential. It has different conductances in each compartment including axons, dendrites, and soma many of these parameters are fixed, but those specified in \ref{fig:parameters} are subject to optimization.


\begin{figure}
\begin{center}
\centering
\begin{subfigure}{.2\textwidth}
  \centering
   \includegraphics[scale=0.5]{figures/correct_active_l5pc.png}
    \caption{A current injection sufficient for causing a single spike is applied for a whole second from $100ms-1100ms$}
  \label{fig:sub1}
\end{subfigure}

As a reference point for understanding 
    \caption{The spike shape is very brief in duration, and so it is worth zooming in for a closer look}

\centering
\begin{subfigure}%{.2\textwidth}
  \centering
    \includegraphics[scale=0.5]{figures/L5Somatosensory_not_optimized.png}
    \caption{$V_{m}$ in $(mV)$ versus time $ms$, plots include a suprathreshold (top) and subthreshold stimulus (below)}
  \label{fig:brief_shape}
\end{subfigure}

\centering
\begin{subfigure}%{.2\textwidth}
  \centering
    \includegraphics[scale=0.5]{figures/morphology_view.png}
    \caption{This multi-compartment model is spatially extended, so a 2D depiction of its 3D form is warranted.}
  \label{fig:brief_shape}
\end{subfigure}

\begin{subfigure}%{.2\textwidth}
  \centering
    \includegraphics[scale=0.5]{figures/correct_passive_l5pc.png}
    \caption{A current injection value of -$10pA$ is applied to the cell for the duration of $200ms-700ms$}
  \label{fig:passive_properties_fine}
\end{subfigure}
\label{fig:test}
\end{center}
\end{figure}



A test suite was constructed using NeuroElectro for the non specific [cortical regions] layer 4/5 cortex pyramidal cell, and we were able to evaluate this layer 5 PC cells against the criteria of the neuroelectro test suite. Note the unnatural looking brief spike duration of the model cell spike  \ref{fig:brief_shape}. It is possible that the majority neuroelectro experiments on the layer 5 pyramidal cell were conducted under room temperature as opposed to body temperature, as there is evidence that the temperature of cortical tissue modulates spike width \cite{goldin2017temperature}, in particular cooling can contract their spike width

%%
% https://neuroelectro.org/data_table/36261/
%%
% from spike width table: 0.65 ± 0.13	1.04 ± 0.25**	0.51 ± 0.03**	0.59 ± 0.06	0.61 ± 0.03
%%
%



\subsubsection{Optimized Results}
\begin{table}[ht]
\centering
\resizebox{\textwidth}{!}{
\begin{tabular}{lllll}
\toprule
{} & observations &                 predictions &   Z-Scores \\
\midrule
RheobaseTest                   &    213.85 pA &                   213.85 pA &  2.444e-06 \\
InputResistanceTest            &  120.67 Mohm &  183.56713371523054 megaohm &     0.8102 \\
TimeConstantTest               &     15.73 ms &   0.00016898141845179352 ms &     -2.152 \\
CapacitanceTest                &    150.58 pF &    0.0009205428827686333 pF &     -1.078 \\
RestingPotentialTest           &    -68.25 mV &        -71.8621793344104 mV &    -0.5533 \\
InjectedCurrentAPWidthTest     &      1.21 ms &                    2.075 ms &      1.623 \\
InjectedCurrentAPAmplitudeTest &     80.44 mV &        63.31628969975701 mV &     -1.343 \\
InjectedCurrentAPThresholdTest &    -42.74 mV &      -44.863238963316746 mV &    -0.2646 \\
\bottomrule
\end{tabular}}
\caption{observations, predictions and Z-scores pertaining to the NeuronUnit optimized Layer 5 Pyramidal Neuron}
\label{tab:l5pc_table}
\end{table}

Due to computational limitations this model was only run for 
$12$ offspring, and $30$ generation. Actually a minimum of $MU=100$, $NGEN =100$ was prescribed by the scientists who optimized the initial model, however such a large compute job required prohibitive computational resources.


The unoptimized model had statistics:
$(\chi^{2},p_{value})=(13.5609360364, 0.093951963105254)$

The optimized model produced statistics
$(\chi^{2},p_{value})=(6.632440090005973 0.5767576828862497)$

Optimization then clearly improves the model, however, it does not bring the model  biological plausibity.





This can be improved by omitting some of the worst tests, overall, the tests are compromized.



It is worth noting that the layer 5 neocortical pyramidal neuron was very slow to dispatch relative to the reduced models developed in this thesis work. Where as a typical reduced model described here evaluated in the order of $~0.0025 seconds$, this model on average took $5.74$, for a single run and $34.8$ to solve for the models Rheobase, current.

This model was pre-optimized to fit to spike times and F/I mainly, and so it should not necassarily be expected to fit other electrical charactersistics of the cell. Only the rheobase test, and the time constant test seemed to fall within the range of biological plausibility.
None the less, this model remains a useful benchmark for reduced neuronal models. % STILL NEEDS A FEW basic results, from the appendix

%
%Test combinations that worked and did not work.
%note move the majority to the appendix
%Moved to appendix, will move back specific results



\subsection{Section 3.11}
Tests that were not always amenable to optimization:
\begin{itemize}
\item ThresholdTest
\item SpikeHalfWidth
\item Spike Amplitude
\end{itemize}
as discussed previously this is because of a threshold measurement that differs between cells. This may be more of a problem in certain regions of model parameter space, but the problem was general, it occurred in multiple models.
%Aim 1A, write something about tests overall.
%Overall the some 
Tests of static electrical properties amenable to optimization:
\begin{itemize}
\item FISlopeTests
\item Rheobase
\item Capacitance
\item Input Resistance
\item Time Constant 
\end{itemize}
%, , , , test worked but was conflicted. The tests that did not work. This is somewhere else.

Tests that worked within optimization:
Via \emph{Elephant} toolchain: FITests, Rheobase, Capacitance, Input Resistance, Time Constant, Resting Membrane Potential.
Via. 

When optimizing in the supra threshold regime Druckmann used:
(1) spike rate; (2) an accommodation index; (3) latency to first spike;(4) average AP overshoot; (5)average depth of after hyperpolarization (AHP); 
(6) average AP width similar to Druckman, when optimizing in the supra threshold regime.
When optimizing with reduced models, I found that the those 6 measurements were not enough to tightly constrain a fit, and additional constraints were helpful. In this work a minimum of 12 constraints were typically used:
\emph{EFEL} tool chain:
\begin{itemize}
\item AHP_depth
\item all_ISI_values,
\item Spikecount (similar to rate)
\item adaptation_index
\item mean_AP_amplitude  
\item min_voltage_between_spikes
\item minimum_voltage
\item peak_voltage
\item spike_half_width
\item time_to_first_spike
\item time_to_last_spike
\item voltage_base
\end{itemize}
 

%3. Study of variance between models and data.  The optimized models part of this section is predicated on result 1b (so that optimization results can be believed).  You already have the poster for this.
%I think this captures most of your results in three themes.  Other results which are really methods, like parallel rheobase search, can stay in the methods, and you will get credit for them there.

\section{Published models vs optimized models}
\label{sec:optimizing-published-models}
It is known that neural models and experimental measurements generally diverge in important ways, however, it is desirable to know the specific sources of divergence. Models and experiments may disagree for two reasons: \emph{A} the model is not flexible enough to satisfy a particular constraint simultaneously to a collection of constraints, or, \emph{B} the model was incorrectly fitted to the a type of conflicting constraints (for example model fitted to spike times at the expense of Rheobase, and FISlope). 

Since we don't yet have complete knowledge of model/experiment divergence, we don't necessarily know the best features to target with regards to model fitting. Specific knowledge of model/data disagreement facilitates the prioritized selection of features that should guide optimization. 

%\begin{figure}
%    \centering
%    \includegraphics{figures/voltage_features.png}
%    \caption{Caption}
%    \label{fig:voltage_figures}
%\end{figure}
%\begin{figure}
%    \centering
%    \includegraphics{figures/AP_Amplitude.png}
%    \caption{Caption}
%    \label{fig:features_example}
%\end{figure}

%\begin{figure}
%    \centering
%    \includegraphics{figures/AHP.png}
%    \caption{After hyperpolarisation potential }
%    \label{fig:features_example_ahp}
%\end{figure}


%\begin{figure}
%    \centering
%    \includegraphics{figures/sag_amplitude}
%    \caption{Caption}
%    \label{fig:sag_amplitude}
%\end{figure}


\subsection{Features} 

For many of these features it will be useful to refer back to the methods \ref{sec:data_sources} section, where I some key features were depicted. 

Consider a voltage recording at the location of the membrane of a neuron. Teams of researchers have already segmented voltage recordings into labelled sections, each section has a classification that is based on the shape of waveform in a limited region see figure \ref{fig:voltage_figures} for example. Rather than specifying by name each measurement it is often useful to refer collectively to these measurable shapes as "features". 

In the following multivariate analysis we analyze hundreds of such features, and we summarize important differences in a subset of this high dimensional feature space.  Below, I describe some neuronal model features that agreed well with experiments, and some features that diverged.


\subsection{Publications Associated with Model Sources}
$972$ models, $448$ experiments.
$1276$ samples. This did not include some blue brain cells. After data cleaning many data points were dropped.  $244$
$1420$


Allen Institute for Brain Science Cell Types Database \citep{celltypes} can be accessed using the SDK.
The Blue Brain Project Dataset \citep{toledo} can be obtained from the Data Navigator, or an API.

\begin{itemize}
\item Allen Institute V1 \cite{gouwens2018systematic}
\item Somatosensory Cortex \cite{markram2006blue} 
\end{itemize}

\subsubsection{Feature Extraction Libraries}
\begin{table}
\centering
%\resizebox{\textwidth}{!}{
\begin{tabular}{lll}
%\toprule
{} EFEL Ephys Feature Extraction Library & AllenSDK & Druckmann (2012) 
%\bottomrule
\end{tabular}
%}
\end{table}




%\begin{tabular}{lll}
%\toprule
%{} Injection 1 & Injection 2 & %Injection 3 \\
% at $1.0 \times$ Rheobase & at $1.5 %\times$ Rheobase & at $3.0 \times$ %Rheobase 
%\bottomrule
%\end{tabular}




%\includegraphics[]{chapters/app_tex/Allen_rush}
\begin{figure}
    \begin{center}
    \includegraphics[width=0.6\linewidth]{figures/multi_spiking_large_allen}
    \caption{A voltage recording from a supra-threshold experiment waveform used as a basis for the Allen Brain Institute cell types data base. Publication Gouwens rat \citep{gouwens2018systematic}}
    \label{fig:adaptionm}
    \end{center}
\end{figure}    

\begin{figure}  
    \begin{center}
    \includegraphics[width=0.6\linewidth]{figures/multi_spiking_large_bbp}
    \caption{Another example of a supra threshold experimental protocol. Publication Jouvanile rat \citep{toledo}}
    \label{fig:bbp_trace_adaption_late_spike}
    \end{center}
\end{figure}    

In order to identify electrical measurements or "features" that were responsible for the most variance in models and in-vitro, we performed Sparse Principle Component Analysis \citep{zou2006sparse} on the combined pool of model and \emph{in vitro} experiments. 

\subsection{Sources of Model/Experiment Disagreement}
The key advantage of using sparse PCA is the results are readily interpret able. 
A common sceanario in regular Principal Component Analysis is you may obtain a low dimensional embedding plot made from unit rotation vectors that maximize variance, but no way of relating the reduced dimensions back to the sources of variance in the system of interest. 

Sparse PCA yields an interpretable list of features, that build the principle components.
This list of features is ranked and sorted with respect to their total contributions to the Eigen Vectors. Since two Eigen Vectors where made these are.

Principle component 1 had non zero loadings for ranked (highest to lowest).
The first Eigen Vector does not facilitate discrimination between models and experiments, but the second Eigen Vector spaces models and experiments into three seperate clusters.

\begin{table}
\begin{tabular}{lll}
\toprule
{} Injection 1 & Injection 2 & Injection 3 \\
 at $1.0 \times$ Rheobase & at $1.5 \times$ Rheobase & at $3.0 \times$ Rheobase 

\end{tabular}
\caption[Model Specific, Three Step Stimulus Protocol]{Under the three step protocol rheobase current is determined uniquely for each model or experiment. Then mutiples of rheobase are applied for two increasingly stronger rheobase strengths at $\times 1.5$ and $\times 3.0 Rheobase$.
In experiments these protocols had already been run, but a re-organising of those preexisting experiments was required in order to make the experiments consistent with the model evaluation scheme.}
\end{table}

Principle component 2 had non zero loadings for ranked (highest to lowest)
\begin{table}
\begin{tabular}{llll}
\toprule
Feature Name & Feature Description & Extraction Library &  Stimulus Strength \\
 upstroke-t & Description & Allen & $1.5 \times$ Rheobase\\
 peak-t & Description & Allen & $1.5 \times$ Rheobase \\
threshold-t & Description & Allen & $1.5 \times$ Rheobase \\
fast-trough-t & Description & Allen & $1.5 \times$ Rheobase \\ 
fast-trough-t & Description & Allen & $3.0 \times$ Rheobase \\
upstroke-t & Description & Allen & $3.0 \times$ Rheobase \\ 
peak-t & Description &Allen & $3.0 \times$ Rheobase \\ 
threshold-t & Description & Allen & $3.0 \times$ Rheobase \\ 
peak-indices & Description & EFEL & $1.5 \times$ Rheobase \\
min-AHP-indices & Description & EFEL & $1.5 \times$ Rheobase \\
\bottomrule
\end{tabular}
\caption[Features of first principal component in sparse PCA]{Features of first principal component in sparse PCA. It is notable that Principle component one described variance, common to both models and experiments, it did not seperate models and experiments into different clusters.}

\end{table}

\begin{table}
\begin{tabular}{llll}
\toprule
Feature Name & Feature Description & Extraction Library  & Stimulus Strength \\
fast-trough-index & Description & Allen & $1.5 \times$ Rheobase\\
peak-index-1.5x & Description & Allen & $1.5 \times$ Rheobase \\
upstroke-index-1.5x & Description & Allen & $1.5 \times$ Rheobase \\
threshold-index-1.5x & Description & Allen & $1.5 \times$ Rheobase \\ 
fast-trough-time & Description & Allen & $1.5 \times$ Rheobase \\
fast-trough-index-3.0x & Description & Allen & $3.0 \times$ Rheobase \\ 
peak-index-3.0x & Description & Allen & $3.0 \times$ Rheobase \\
upstroke-index-3.0x & Description & Allen & $3.0 \times$ Rheobase \\ 
threshold-index-3.0x & Description & Allen & $3.0 \times$ Rheobase \\
\bottomrule
\end{tabular}
\caption[Features of second principal component in sparse PCA]{Features of second principal component in sparse PCA. It is notable that mainly these features contributed to the seperation of model and experiment clusters}
\end{table}

Adaptation index 1 and adaptation 2 are the same but adaptation 2 evaluates for when there are skipped peaks. A skipped peak is when a spike or a spiklet does not surpass the nominated threshold for spike detection. 

%The parameter \myid{spike skipf} is the fraction of skipped peaks, $k$ is the minimum of \myid{spike skipf} times $N$ and \myid{max spike skip}.

% FAIL "Minimum 4 spike needed for feature [adaptation\_index]." \- \\
The adaptation index is defined as "the Normalized average difference of two consecutive ISIs". "The adaptation index is zero for a constant firing rate and bigger than zero for a decreasing firing rate \citep{EFEL}"

  %All peaks in the time interval of \myid{stim start}$-$\myid{offset} and \myid{stim end}$+$\myid{offset} are regarded, \myid{offset} defaults to zero.
  
%pt$_0, \ldots, $pt$_{n-1} =$ peak\_time \\
%  pt$'_0$, \ldots, pt$'_{m-1}$  = $\{$ pt$_i$ | pt$_i \ge$ stim\_start $-$ offset AND pt$_i \le$ stim\_end $+$ offset $\}$ \\
%  $k = \min \{$ spike\_skipf $\cdot m$, max\_spike\_skip$\}$ \\
%  pt$''_0$, \ldots, pt$''_{l-1}$ = (pt$'_k$, \ldots, pt$'_m$) \\
%  IF $l$ < 4 THEN \+ \\
%    FAIL "Minimum 4 spike needed for feature [adaptation\_index]." \- \\
%  ENDIF \\
%  isi$_0$, \ldots, isi$_{j-1} =$ pt$''_1 -$ pt$''_0$, \ldots, pt$''_{l-1} -$ pt$''_{l-2}$ \\
%  sub$_0$, \ldots, sub$_{i-1} =$ isi$_1 -$ isi$_0$, \ldots, isi$_{j-1} -$ isi$_{j-2}$ \\
%  sum$_0$, \ldots, sum$_{i-1} =$ isi$_1 +$ isi$_0$, \ldots, isi$_{j-1} +$ isi$_{j-2}$ \\
%  APPEND $\frac{1}{i-1} \sum_{n=0}^{i-1} \frac{\mathrm{sub}_n}{\mathrm{sum}_n}$ TO adaptation\_index}
 % All peaks in the time interval of \myid{stim start}$-$\myid{offset} and \myid{stim end}$+$\myid{offset} are regarded, \myid{offset} defaults to zero.
The adaptation index is zero for a constant firing rate and bigger than zero for a decreasing firing rate:
  


%Adaptation index 2  {Normalized average difference of two consecutive ISIs}

% The parameter \myid{spike skipf} is the fraction of skipped peaks, $k$ is the minimum of \myid{spike skipf} times $N$ and \myid{max spike skip}.

\url{https://github.com/BlueBrain/eFEL/blob/master/docs/source/tex/efeatures.tex#L382}

These are the weighted features that were used to make Eigen vectors 1 and 2, are responsible for most of the variance. Interestingly these features mostly belong to the Allen SDK feature extraction set, with two exceptions: $peak-indices-1.5x$ $min-AHP-indices-1.5 \times$ belonging to EFEL efel 

%Move to Discussion 
Another observation is that a small majority of features used to create the sparse Eigen Vectors, are in the range of $1.5 \times$  rheobase, and slightly fewer are features from a $3.0\times$ rheobase experiment. A reason for this is as follows, larger spike time variability $C_{V}$ is expressed in intermediate ranges of current injection. Under the highest current injections, high frequency evenly spaced spikes are likely, although spike frequency adaption is possible, the higher current may force to spikes to occur promptly after their refractory period, and in this case you might observe diminishing amplitude of spikes with increasing stimulus duration.

At $1.5 \times rheobase $ I believe there to be more spike time variation, at $3.0 \times rheobase $  I believe there to be more spike amplitude variation.

\subsection{Sparse PCA}
Sparse PCA revealed five overlapping different groups of neuronal identity's, and three non overlapping clusters. Models and experiments clustered separately to each other in the low dimensional space. Models and data were easily separable in the direction of the 2nd Eigen Vector.

%The first principle component 
Models and experiments shared the same breadth of variability across the first principal component, with only slightly more variance in the experiments than in the data. Allen cell types experiments seemed to encompass the most variability out of models and experiments across all data sets. Models clustered tightly and varied less than in vitro experiments.



Although imputation was successfully used to avoid dropping a large number of samples, about half of all initial BBP/Allen models data types were excluded from a final analysis, because they did not all capable of meeting inclusion criteria.  

\ref{fig:reference_feature_list}
List of complete 47 features used in the analysis down from 466 after data cleaning.


There was one group of Allen cell type experiments that clustered on their own, making one set of the cluster.

sparse PCA 2nd Eigen Vector. 
Disagreement between models and in-vivo neurons may reflect limitations of model design and can be investigated by probing the key features used by classifiers to distinguish these two populations. 

\begin{figure}    
\begin{center} \includegraphics[width=1.0\linewidth]{figures/cortical_model_data_agreement_52_1}
    \caption[Is this visible]{}
    \label{fig:}
\end{center}
\end{figure}    



\begin{figure}    
    \begin{center}
    \includegraphics[scale=0.75]{figures/cortical_model_data_agreement_54_1.png}
    
    
%fast_trough_indexes : numpy array of indexes at the start of the trough (i.e. end of the spike)
%adp_indexes : numpy array of adp indexes (np.nan if there was no ADP in that ISI
%slow_trough_indexes : numpy array of indexes at the minimum of the slow phase of the trough
    \end{center}
\end{figure}    
\cite{wang2019sag}

\begin{figure}
    \centering
    \includegraphics[scale=0.75]{figures/features_that_disagree}
    \caption[Features that disagree. Slow Trough indexs, from the Allen cell types feature extraction]{The sparse PCA graph is very abstract, it's important to understand that much variance found by the sparse PCA algorithm originates from differences between model distributions and experiment distributions Simple stacked histograms of data are able to show such differences. By "Fast" in fast trough indexs, all that is meant is the beginning time of troughs occurrence, this would contrast with the "slow" or ending time of the spikes. This is arguably a bad naming scheme.
        https://allensdk.readthedocs.io/en/latest/allensdk.ephys.ephys_features.html
        }
        \label{fig:from_poster_disagree}
\end{figure}



\begin{comment}
\begin{itemize}
    \item upstroke\_t\_1.5x allen feature
    \item  peak\_t\_1.5x allen feature
    \item threshold\_t\_1.5x allen feature
    \item fast\_trough\_t\_1.5x allen feature
    \item fast\_trough\_t\_3.0x allen feature
    \item upstroke\_t\_3.0x allen feature
    \item peak\_t\_3.0x allen feature
    \item threshold\_t\_3.0x allen feature
    \item peak\_indices\_1.5x efel feature
    \item min\_AHP\_indices\_1.5x efel feature
\end{itemize}
\end{comment}


%\begin{itemize}
%    \item fast\_trough\_index\_1.5x allen feature
%    \item fast\_trough\_index\_3.0x allen feature
%    \item threshold\_index\_1.5x allen feature
%    \item peak\_index\_1.5x allen feature
%    \item upstroke\_index\_1.5x allen feature
%    \item peak\_index\_3.0x allen feature
%    \item upstroke\_index\_3.0x allen feature
%    \item threshold\_index\_3.0x allen feature
%\end{itemize}

After identifying specific sources of model and experiment divergence, it is now possible in theory to start fitting models which seek to resolve specific types of disagreement.
However, as alluded to in the introduction, it was found that were two other important factors. 

%Model repurposing is common and it is done on a network scale \cite{traub} and an individual cell scale.
%Experimental evidence is starting to reveal that model re-purposing of pyramidal neurons might not be a good idea.

%Scientific insight is well-served by the discovery and optimization of abstract models that can reproduce experimental findings. NeuroML (NeuroML.org), a model description language for neuroscience, facilitates reproducibility and exchange of such models by providing an implementation-agnostic model description in a modular format. NeuronUnit (neuronunit.scidash.org) evaluates model accuracy by subjecting models to experimental data-driven validation tests, a formalization of the scientific method. 


% After applying dimensionality reduction to this very high dimensional feature space, we show that the real (biological neurons) and simulated (model neurons) recordings are easiley and fully discriminated by eye or any reasonable classifier.  

% Are they still discernable?
%972 models, 448 experiments.


%Consequently, not a single model neuron produced physiological responses that could be confused with a biological neuron. Was this a defect of the model design (e.g. key mechanisms unaccounted for) or of model parameterization? We found that if we introduced models that were revised via optimization the revised models overlapped with the distribution of biological neurons, and were mostly classified as such. 




%
%note move the majority to the appendix
Moved to appendix, will move back specific results
%%\subsection{Neocortical Layer 4/5 Pyramidal Cell Test Suite}
\subsubsection{Performance of Layer 5 Prefrontal cortex Pyramidal Neuron on NeuronUnit tests of model data agreement}

A suite of neuronunit tests containing the tests: rheobase value, membrane voltage time constant ($tau_{m})$, input resistance was computed. This multi-compartment, conductance based model served as a useful benchmark, for us to evaluate the relative performance of reduced model fits.
    
\begin{figure}    
\begin{center}
    \includegraphics[width=0.7\linewidth]{figures/NU_BBP_fusion_L5PC_files/NU_BBP_fusion_L5PC_3_1.png}
\end{center}
\end{figure}  

It is worth noting that the layer 5 neocortical pyramidal neuron was very slow to dispatch relative to the reduced models developed in this thesis work. Where as a typical reduced model described here evaluated in the order of $~0.0025 seconds$, this model on average took $5.74$, for a single run and $34.8$ to solve for the models Rheobase, current.

\begin{table}[ht]
\centering
\resizebox{\textwidth}{!}{
\begin{tabular}{lllllll}
\toprule
{} & RheobaseTest & InputResistanceTest & TimeConstantTest & CapacitanceTest & InjectedCurrentAPWidthTest & InjectedCurrentAPThresholdTest \\
\midrule
0 &     Z = 0.07 &           Z = -0.90 &         Z = 0.14 &        Z = 1.29 &                  Z = -1.98 &                      Z = -2.09 \\
\bottomrule
\end{tabular}}
\end{table}

In this pre-optimized model, only the rheobase test, and the time constant test seemed to fall within the range of biological plausibility.
\section{Optimized Single Neurons}
\label{sec:optimized-single-neurons}
In the previous sections we saw that optimization of models to the mean features reported across many neurons, even of the same nominal type, is conceptually and empirically flawed.
The remainder of the optimization results will focus instead on the optimization of single neurons.
In other words, I will produce optimized models that used only features extracted from recordings of the same neuron, and those models will putatively represent any neuron that behaves as that one did.
This conveniently eliminates not only variability across neurons of the same type, but also variability across recordings sessions or labs, since by experimental necessity intracellular recordings are collected in a matter of minutes in a single lab.

Table \ref{tab:single-neuron-constraints} which constraints (i.e. NeuronUnit tests were used to guide this process.  
These constaints were different from those used in the NeuronUnit tests for two reasons: a) NeuronUnit contains very little data about some of these features and b) the single neuron data used here was collected using a family of suprathreshold stimuli that allow many complex spike-pattern features to be calculated.

%%%
% Note to self, 
% Table and caption represents an older approach. What is written is true, for the time that it was used, but it no longer reflects the features mainly used to guide optimization
%%%
\begin{table}
    \centering
    \resizebox{1\textwidth}{!}{
    \begin{tabular}{|c|c|c|}
    \toprule
Feature Name & Description \\
adaptation-index & Adaption Index for above $0mV$ spikes \\
adaptation-index2 & Adaption Index for a mixture of above and below $0mV$ spikes \\
time-to-first-spike & The time in seconds to first spike \\
mean-AP-amplitude & The average in AP amplitude \\
spike-half-width & Spike width measured at half amplitude \\ 
AHP-depth & After Hyperpolarization Depth \\ 
minimum-voltage & Minimum voltage in $V_{M} $ recording \\
peak-voltage & Peak voltage in $V_{M}$  recording \\
time-to-last-spike & Time elapsed until final spike. \\
AHP-depth-abs & Mean after-hyperpolarization amplitude. \\
all-ISI-values & All Inter Spike Interval Values \\
voltage-base & The base voltage of $V_{M}$ while undergoing stimulus. Often not the minimum $V_{M}$ \\
min-voltage-between-spikes & The minimum voltage between spike in $V_{M}$, while neuron undergoing stimulus \\
Spikecount & Number of spikes observed in model, given appropriate AP \\
\bottomrule
    \end{tabular}}
    \caption[Suprathreshold Features Used for Single Neuron Model Fits]{\textbf{Suprathreshold Features Used for Single Neuron Model Fits.}
    These 14 features were identified from \cite{EFEL} as useful for fitting single neuron models for which responses to suprathreshold stimuli were available.
    Unlike the features used in NeuroElectro, these features can distinguish between different temporal patterns of spiking.
    Special care was taken to minimize the impact of AHP depth and Minimum Voltage between spikes, as reduced models struggle to match these due to inherently impoverished dynamics.}
    \label{tab:single-neuron-constraints}
\end{table}


%\begin{tabular}{lr}
%\toprule\n{} &         0 \\\\\n\\midrule\nAHP\\_depth\\_1.5x                  &  0.707200 \\\\\nAHP\\_depth\\_abs\\_1.5x              &  0.413800 \\\\\nSpikecount\\_1.5x                 &  0.000000 \\\\\nadaptation\\_index2\\_1.5x          &  0.008393 \\\\\nall\\_ISI\\_values\\_1.5x             &  0.004261 \\\\\nmean\\_AP\\_amplitude\\_1.5x          &  0.387600 \\\\\nmin\\_voltage\\_between\\_spikes\\_1.5x &  0.394000 \\\\\npeak\\_voltage\\_1.5x               &  0.005194 \\\\\nspike\\_half\\_width\\_1.5x           &  0.892900 \\\\\ntime\\_to\\_first\\_spike\\_1.5x        &  0.182200 \\\\\ntime\\_to\\_last\\_spike\\_1.5x         &  0.016420 \\\\\nvoltage\\_base\\_1.5x               &  0.107100 \bottomrule
%\end{tabular}

% \end{verbatim}

%\begin{table}[]
%    \centering
%    \resizebox{1\textwidth}{!}{
%    \begin{tabular}{c|c|c}
%    \toprule
%        Feature Name & Description & injection strength \\
%        AHP-depth-abs & After Hyperpolarisation Depth absolute value & $3.0 \times $ Rheobase \\ 
%        sag-ratio2 & Difference between maximum negative deflection of $V_{M}$ and steady state $V_{M}$ during negative stimulus application &  $3.0 \times $ Rheobase \\ Input Resistance $\frac{V_{M}}/{I_{stimulus}}M \Omega$ &  $1.5 \times $ Rheobase \\ 
%        peak-voltage & time of peak voltage &  $3.0 \times $ Rheobase \\
%        voltage-base & value of base voltage (minimum $V_{M}$ during stimulus &  $3.0 \times $ Rheobase \\
%        Spikecount & Number of spikes during stimulus &  $3.0 \times $ Rheobase  \\
%        ohmic-input-resistance-vb-ssse & Input Resistance from voltage base steady state equilibrium &  $1.5 \times $ Rheobase  \\
%    \bottomrule
%    \end{tabular}}
%    \caption[Constraints for Suprathreshold Single Neuron Model Fits]{A table of constraints that were used to guide single neuron optimization using suprathreshold stimuli.
%    This set varied across optimizations, depending on which underlying stimuli were available in the recorded experimental data.
%    While features such as AHP depth and Minimum Voltage between spikes improved optimization, all things being equal, if they were given too much priority (over statistics derived from spike times or amplitudes) then fit quality was reduced.}
%    \label{tab:single-neuron-constraints}
%\end{table}

\subsection{Optimization of Blue Brain Project Neurons}
I use two data sources for these results.
First, I use a somatic current injections conducted on a rat somato-sensory hind limb neurons as part of the Blue Brain Project.

Figure \ref{fig:B95Adexp} shows an example of an AdEx model optimized against data from such a neuron, which provided a challenging response (to somatic current injection) to fit.
For example, while spike rate adaptation can be observed in this neuron, it is not monotonic: while the inter-spike-intervals (ISIs) are generally increasing, the final ISI is actually slightly shorter than the penultimate one.
Nonetheless, the optimizer achieves a reasonably good match to this pattern of spikes, as well as to the resting potential and the spike threshold.
It performs less well at capturing the hyperpolarization between spikes.
Similar results are shown using an Izhikevich model (Figure \ref{fig:B95_IZHI}).
The AdEx model is somewhat better at matching spike times than the Izhikevich model, consistent with the literature \citep{rossant2011fitting}. 

\begin{figure}
    \centering
    \includegraphics[scale=0.75]{figures/bbp_multispiking_fit.png}
    \caption[Optimized AdEx model from BBP]{\textbf{AdEx model optimized against BBP data}. An Adaptive Exponential (AdEx) model was optimized to both the spike times and shapes from a single neuron in the Blue Brain Project (BBP) Microcircuit Portal data (taken from animal ID B95).
    The real biological neuron is the orange trace, and the simulated trace is shown in blue.
    %\url{http://microcircuits.epfl.ch/#/animal/8ecde7d1-b2d2-11e4-b949-6003088da632}
    While there as some ``misses" in the subthreshold behavior in between spikes, and in the amplitudes of the spikes, due to fundamental limitations in the dynamics of the model, the basic pattern of spiking is successfully recapitulated.}
    \label{fig:B95Adexp}
\end{figure}

\begin{figure}
    \centering
    \includegraphics{figures/IZHI_B95.png}
    \caption[Optimized Izhikevich Model from BBP]{\textbf{Izhikevich Model Optimized against BBP Data}. Similar to Fig. \ref{fig:B95Adexp}, but instead using an Izhikevich model.
    Here optimisation is a bit worse, but not poorer than the best out-of-sample predictions of spike timing from previous efforts.}
    \label{fig:B95_IZHI}
\end{figure}

\subsection{Allen Cell Types Database Neurons}
I then used similar data for mouse neurons taken from primary visual cortex and optimized the AdEx model (which gave the best results for the BBP data considered above).
These results were improved over those using the BBP neuron data.
As shown in Figures \ref{fig:specimen_476053392_A}, \ref{fig:specimen_476053392_B} and \ref{fig:specimen_325479788}, optimized models of this flavor were capable of reproducing patterns of spikes and (to a better extent than using the BBP data) subthreshold dynamics.
The same optimized model was also capable of reproducing the patterns of spikes in response to different amplitudes of somatic current injection (Figs \ref{fig:specimen_476053392_A} and \ref{fig:specimen_476053392_B}).

\begin{figure}
    \centering
    \includegraphics[scale=1]{figures/adexp_fit_allen_spec_id_476053392.png}
    \caption[Optimized AdEx Model from Allen Cell Types (A)]{\textbf{AdEx Model Optimized against Allen Cell Types Data}.
    An AdEx Model was fitted to suprathreshold responses from Allen Cell Types cell id 476053392.
    The optimized model matches the spikes times of the experimental data in response to the same stimulus.}
    \label{fig:specimen_476053392_A}
\end{figure}

\begin{figure}
    \centering
    \includegraphics[scale=1]{figures/28spikesB95}
    \caption[Optimized AdEx Model from Allen Cell Types (B)]{\textbf{Optimized AdEx Model from Allen Cell Types (B).} Same as above, but for a somatic current injection of larger magnitude.}
    \label{fig:specimen_476053392_B}
\end{figure}

%Similar to Druckmann Suprathreshold depolarizing step currents \cite{druckmann2008evaluating}.
%\begin{figure}
%    \centering
%    \includegraphics[scale=1]{figures/adexp_fit_allen_specid_325479788.png}
%    \caption[Optimized AdEx model from Allen Cell Types (C)]{\textbf{AdEx model optimized against Allen Cell Types Data}.
%    Same as above, but for cell ID 325479788}
%    \label{fig:specimen_325479788}
%\end{figure}

\subsection{Comparison to Best-case Scenarios}
Since these result now concern fits to single neurons, I cannot use the $\chi^2$ measure of agreement between the optimized model and the distribution of the data; the distribution now has only a single member per optimization.
Instead, I will compare the quality of these results to best-cases scenarios from prior research.
I will use the ``variance explained ratio" between the experimentally recorded data and the model simulation in response to the same stimulus.

Previously, a competition was held \citep{incf_multi} inviting participants to predict spike times (in response to specific patterns of somatic current injection) for well-characterized cortical neuron.
The best models in this competition were only able to predict $\sim86\%$ of spike times (Fig. \ref{fig:rossant-fits}), and this under conditions with fluctuating somatic currents that actually produce more spike-timing regularity than square currents \citep{mainen1995reliability}.
Consequently, it is unlikely that \emph{in silico} models can be expected to produce a perfect match of spike times in response to square wave current injection, tempering expectations for my optimizer.
As noted in the previous optimization efforts \citep{druckmann2007novel}, it is probably mistaken to fit models to precise spike times; real cortical neurons produce and receive noisy currents, and likely rarely exhibit identical spike trains even in response to the same nominal stimulus.
Optimization on derived features of spike timing and rate, such as statistics of distributions of ISIs, of F-I curves, is a more promising approach.

\begin{figure}
    \centering
    \includegraphics[scale=5.0]{figures/IZHIkevich_fit_60Adexp_80.jpg}
    \caption[Model Fits in Prior Work]{\textbf{Conflicts between Spike Timing and Spike Shape}.
    This figure from  \cite{rossant2011fitting} shows how reduced models that can fit spike times typically cannot also fit spike shape.
    Several models are shown in this figure including the Izhikevich model used in current work and a ``MAT2" model, which is a more well-equipped version of the AdEx model.
    The MAT2 model correctly predicts $86\%$ of experimental spike times, but exhibits a spike threshold (a components of spike shape) that is much higher in the model compared to the experiment.
    The Izhikevich model only predicts $62\%$ of spike times (this is also consistent with the results in this thesis), but it displays richer subthreshold and more accurate peri-threshold behavior than the more mercenary MAT2 model.} 
    \label{fig:rossant-fits}
\end{figure}

%\begin{comment}
%\begin{figure}
%    \centering
%    \begin{subfigure}[t]
%        \centering
%        \includegraphics[width=0.5\linewidth]{example-image-a.pdf} 
%        \caption{Generic} \label{fig:timing1}
%    \end{subfigure}
%    \hfill
%    \begin{subfigure}[t]
%        \centering
%        \includegraphics[width=0.5\linewidth]{example-image-b.pdf} 
%        \caption{Competitors} \label{fig:timing2}
%    \end{subfigure}
%
%    \vspace{1cm}
%    \begin{subfigure}[t]
%        \centering
%        \includegraphics[width=0.5\linewidth]{example-image-c.pdf} 
%        \caption{Price regulation} \label{fig:timing3}
%    \end{subfigure}
%    \hfill
%    %\begin{subfigure}[t]%{0.45\textwidth}
%    %    % just an empty subfigure to shift C below A
%    %\end{subfigure}
%    \caption{Some general caption of all the figures. In (\subref{fig:timing1}) %you can see a  green square....}
%\end{figure}
%\end{comment}

% TODO make multi panel.




% Interesting direct qoute from Druckmann:
%"In experiments, intrinsic noise gives rise to a large variability (e.g., in firing pattern) in the voltage responses to repetitions of the exact same input. Thus, the common approach of fitting models by attempting to perfectly replicate, point by point, a single chosen trace out of the spectrum of variable responses does not seem to do justice to the data."

%In experiments, however, when the exactly same stimulus is repeated several times, the voltage traces elicited differ among themselves to a significant degree (Mainen and Sejnowski, 1995; Nowak et al., 1997).










%\section{Limitations of Existing Approaches} 
Existing community supported simulated models are problem ridden.

% NEURON version of Izhi model is not as fast as one might expect, this may because of the way we tried to implement the model, by creating and destroying HOC module instances, that contain the model. 

% Nonetheless, 
Our version of the NEURON-Izhi-model which was derived from a J-NeuroML translation, could not reproduce all of the Izhi original publication figures, because it had two different conflicting sources of capacitance.\\
\\
From JNeuro-ML we created NEURON version of Izhikich model, however we were not able to make this model execution times brief enough to be useful for optimization. Although the NEURON simulator is designed to be fast, their may be a cost associated with reloading the NEURON environment many times in fast succession.

This should not go in the general introduction, but in the intro to the appropriate section of the results where you use those models.

\section{A Web Application for Optimization and Visualization}
An application was developed using the reduced model neural model optimizer as flexible backend. %Since the optimizer essentially acts like the backend of a program, creating an application simply meant writing a "front-end".\\
\\
A python library "streamlit" facilitates the construction of web applications. Fortunately for the developer, coding in this web frame work requires no handling of html elements, and streamlit has tools for converting python data into interactive tables and plots. By abstracting html out of the workflow streamlit makes the front-end development accessible to time-poor backend developers.\\
\begin{figure}
\begin{center}

\includegraphics[scale=1]{chapters/app_tex/web_app_thesis}
\caption{The side-pane of the web application provides users with a choice of three models, and four data sets that can be used to fit data.
}
\end{center}

\end{figure}
\begin{figure}
\begin{center}

\includegraphics[scale=1]{chapters/app_tex/app_results}
\end{center}

\end{figure}

When optimization is complete, the user sees the $\chi^{2}$ statistic, the $p$-$value$. The user is then able to follow a link to download the "model". Later the user will be able to download the full NeuroML model specification.

\begin{figure}
\begin{center}

\includegraphics[scale=1]{chapters/app_tex/more_app_results}
\end{center}
\caption{An accompanying interactive visualisation of the optimized model neuron firing under rheobase firing, and also under passive conditions is supplied.}

\end{figure}

\begin{figure}
\begin{center}

\includegraphics[scale=1]{chapters/app_tex/Screenshot from 2020-09-19 10-46-32}
\end{center}

\end{figure}


For each sciunit score in a suite, the Z-scores are visualized as an appropriately placed point on a bell curve. In addition to the $\chi^{2}$ test, this enables users to see how well the model does per test, on features that they may care more about.

The user then has the option of consulting displayed Z-scores, for each test.
\begin{figure}
\begin{center}
\includegraphics[scale=1]{chapters/app_tex/Screenshot from 2020-09-19 10-47-27}
\caption{In principle the web application is compatible with the approach of fitting models to the supra threshold multi-spiking experiments approach but this functionality does not exist at the time of writing}
\end{center}

\end{figure}



%\includegraphics[]{chapters/app_tex/Screenshot\ from\ 2020-09-19\ 10-47-31}


%Test combinations that worked and did not work.
%note move the majority to the appendix
%Moved to appendix, will move back specific results

% Put some of the below (whatever is worthwhile) into Results section 1b

