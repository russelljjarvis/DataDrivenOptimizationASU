
\section{Limitations of Optimization using Experimental Data}
\label{sec:limitations-of-optimization}

\subsection{Is Optimization Always Possible?}
Not all of the available data-sets were conducive to model fitting on the reduced cells. In particular the cerebellum Purkinje cell, demands a very large current stimulus to elicit a rheobase spike: $680pA$, and the reduced cells were generally not able to match this high rheobase value. The physical determinants of the large rheobase current are of interest, Cerebellum Purkinje cells have a very large surface area, in particular their dendrite supports on average 100,000-plus inputs from parallel fibers. The Purkinje cell dendrites have a large surface area to support the large number of post synaptic receptors. The soma and apical dendrite trunk of the purkinje are both also very large. Owing to having both large volume and large surface area, there are a lot of different places electron current can disperse too, besides traveling directly to the axon hillock. Additionally although current injection is applied to the soma, the dendrite is coupled to the soma, so that the dendrite may absorb some of the current injected at the cell soma. 

% so it makes sense that Purkinje cells should be insensitive to minor fluctuations of stimulus current, 
Owing to claims above, Purkinje cells should be expected to absorb large currents without firing an AP. Generally the reduced models could not fit too the Purkinje cells demand for a high rheobase current. 

Some reduced models could achieve rheobase currents as strong as $350-400pA$, but this was usually at the expense of compromizing all the other fitted measurements. Fitted models of the cerebellum purkinje cell, always failed to be biologically plausible. The p-value of the $\chi^{2}$ statistic always gave reason to reject the null hypothesis.

Just like the Purkinje cell, mitral cell olfatory bulb measurements could not be recapitulated in models. olfactory mitral cells have relatively high membrane capacitance $235pF$, and high input resistance $130MOhm$. Generally reduced models could not reproduce these values.  

Over all the tests, including two of the most challenging cases: the Cerebellar Purkinje cell, and olfactory bulb mitral cell, the Izhikevich model was achieved the lowest overall $\chi^{2}$ p-value, as it was able to perform the best all-round fits. %However if you were to remove those two data sets, and evaluated on the remaining 6-8 out of data sets, the Adaptive Exponential model would be a better fit.

\subsection{Conflicts between experimental constraints}
Conflicts between Rheobase value, and other static electrical measurements were observed in NeuroElectro and Allen Cell types measurements. When optimizing against NeuroElectro averaged measurements, and Allen Cell types single cell observations, it was common to experience a conflicted ability for the cell to satisfy all constraints simultaneously. 
Models seemed to have particular difficulty in recapitulating an accurate fit for rheobase, while simultaneously satisfying the larger set of fitness criteria: time constant, input resistance, capacitance and resting membrane potential. There was better agreement between firing rate versus current (FISlope) and the remainder of the electrical observations.


%The l5pc model was pre-optimized to fit to spike times and F/I mainly, and so it should not necessarily be expected to fit other electrical characteristics of the cell. Only the rheobase test, and the time constant test seemed to fall within the range of biological plausibility. None the less, this model remains a useful benchmark for reduced neuronal models.
%It was desirable to include this extended range of Izhikevich model behavior
%However, as noted in the introductory material, it i 
%Previously I mentioned neuronal modelling competitions I have optimize every model against the same data sets in order to assess overall which model is better able to fit to diverse data sets.

%
%\begin{comment}
%\subsection{Neocortical Layer 4/5 Pyramidal Cell Test Suite}
%\subsection{%2a}
%Direct Quote: "widening of the spike shape, decrease of the firing rate and change in the interspike interval distribution". %All these single unit waveform shapes increased their width with temperature.\cite{goldin2017temperature}

%1a/b Is Optimization possible?
%       1a. Construction of tests from diverse experimental sources (I wrote the neuroelectro api and its use in neuronunit, and wrote the original Allen one, but you have put in work to create runnable tests from these and other sources).  This is in a sense a method, but you can still report that these tests are runnable, even outside of optimization.
%       1b. Simulated data tests of optimization.  What works?  What doesn’t?  Why (briefly, saving some for discussion)?  NeuroElectro vs Allen also belong here, and fits in with 1a.  You should talk about model means vs means of models (or whatever we are calling it) here, if you have the results for it or think you can in 3 weeks.  
       %You can talk about rippled error surfaces — this is such a deep technical detail that I wouldn’t spend a lot of time on it.  In other words it may be important but it will be almost impossible to follow even if written well.


%During optimization knowledge of error surfaces should not be mandatory but it can help to solidify good optimization outcomes.  Through human examination of resulting error surfaces, it was found that some of the novel test sets were not helpful to the optimization framework.  Interestingly some types of tests had a propensity to amplify errors originating from elsewhere.

% depended on current stimulus values that were not fixed between models, but instead where contingent on the changeable state of the model cell, this measurement was usually the derivative of membrane potential: $max(\frac{dV_{m}}{dt})/10$. For more on this see \ref{sec:Optimization Pitfalls} 
% duplication
%I found that some fraction of these new tests were because they depended on measured features relative to some other changeable measurement inside the cell usually $max(\frac{dV_{m}}{dt})/10.$ As I discuss in methods, had the propensity of amplifying errors that propogated from elsewhere. 
%This required both the extraction of novel features [EFEL: ISI, AHP-depth, adaption ratio etc.] on new data types types.

%This class was also acculated useful helpful methods such as retrieving default model parameters.
