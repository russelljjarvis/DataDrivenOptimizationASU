

%Flat regions of error surface are uninformative.
%Tests that I curated from Allen Cell types lead to some models being under-constrained about spike width. The consequences of  under-restrained were not obvious/

%they were revealed by graphs in a virtual experiments were appropriate models were elicited to spike. The lack of constraint was easily rectified, by imposing a specific spike width constraint on Adaptive exponential models, however, unexpectedly in this context, the models $\chi^{2}$ increased dramatically and biological plausibility plummeted, in all except one test. To  paraphrase, the adaptive exponential models had found an unexpected way to cheat tests, by taking advantage of a lack of constraints in an unconstrained area, ie adopting implausibly long spike widths made the AdExp models exceptionally good at passive tests.

%In type of standard, the NeuronUnit tests, themselves act as the final judge of model quality, in the absence of a spike width tests, many AdExp models were able to get very good fits on against supplied constraints, but plots of actual spike shape looked very unnatural, as spike width lasted $>=$ 6ms. Applying extra standards beyond the NeuronUnit tests creates a dilemna. As all the GLIF models, presented unusual spike shapes.

%
%Although the Rheobase was the cause of  significantly impeding error, it was not so much of a problem to include this test itself, it was more of a problem to include tests that where contingent on its value.

%error which could propogate into other tests that depend on its value, 


% From the plots below one can see that 



\subsection{Experiment Fitted Model Results on Reported Data Types} 
Because some features derived from the data were incompatible or unreliable, it was necessary to create additional NeuronUnit tests from other feature extraction libraries, as described in the Methods.
We used data from four different distinct cell types with data obtained from NeuroElectro and four different single cortical cells from the Allen Cell Types database, for a total of eight sets of feature data to optimize.
We then optimized against this data using a suite of nine NeuronUnit tests. Additionally, to explore if FITests, and Rheobase fitted better on their own, an additional four non-unique test sets where also created. The entire set of tests is presented in tabular form (\ref{table:tests_derived_from_reports}). See \ref{sec:allen_report_data} in the Appendix for the URLS for the data sources.

%id's:623960880,623893177,471819401,482493761

\begin{table}
\resizebox{\textwidth}{!}{
\begin{tabular}{|c|c|c|c|}
\toprule
 Source & Cell ID & Subset of already listed tests & Sample type  \\
\midrule
Allen & 48249376[1] & Y &     Single cell \\
Allen & 47181940[1] & Y &     Single cell \\
Allen & 623893177[1] & Y &     Single cell \\
Allen & 623960880[1] & Y &   Single cell \\  
Allen & 482493761 & N &     Single cell \\  
Allen & 471819401 & N &     Single cell \\
Allen & 623893177 & N &    Single cell \\
Allen & 623960880 & N &     Single cell \\
NeuroElectro & Olfactory Mitral Cell & N &     Mean of cells \\
NeuroElectro & Neocortex pyramidal cell layer 5-6 &      N &     Mean of cells \\
NeuroElectro & Cerebellum Purkinje cell & N &     Mean of cells \\
NeuroElectro & Hippocampus CA1 pyramidal cell & N &     Mean of cells \\
\bottomrule
\end{tabular}}
\caption[Properties of Different Experimental Data Used]{Properties of the experimental data used in this section.
Two data sources (NeuroElectro and the Allen Cell Types database) were used to fit neuron models.
All Allen Cell Types data come from cortical neurons of various types.
Allen Cell Types data correspond to individual recorded cells, and NeuroElectro data come from mean feature values reported across many neurons of the same nominal type.}
\label{tab:main_chi2}
\end{table}

\begin{table}
\resizebox{\textwidth}{!}{
\begin{tabular}{lllllllllllll}
\toprule
name & CA1 pyramidal & Purkinje & NCP Layer 5-6 & Mitral &      623960880 &      623893177 &      471819401 &      482493761 &  6239608801 &  6238931771 &  4718194011 &  4824937611 \\
\midrule
RheobaseTest                   &                      189.24 pA &                680.79 pA &                          213.85 pA &          NaN &        70.0 pA &       190.0 pA &       190.0 pA &        70.0 pA &     70.0 pA &    190.0 pA &    190.0 pA &     70.0 pA \\
InputResistanceTest            &                    107.08 Mohm &              142.06 Mohm &                        120.67 Mohm &  130.08 Mohm &  241.0 megaohm &  136.0 megaohm &  132.0 megaohm &  132.0 megaohm &         NaN &         NaN &         NaN &         NaN \\
TimeConstantTest               &                        24.5 ms &                      NaN &                           15.73 ms &     24.48 ms &        23.8 ms &        27.8 ms &        13.8 ms &        24.4 ms &         NaN &         NaN &         NaN &         NaN \\
CapacitanceTest                &                        89.8 pF &                620.27 pF &                          150.58 pF &    235.75 pF &            NaN &            NaN &            NaN &            NaN &         NaN &         NaN &         NaN &         NaN \\
RestingPotentialTest           &                      -65.23 mV &                -61.59 mV &                          -68.25 mV &    -58.14 mV &       -65.1 mV &       -77.0 mV &       -77.5 mV &       -71.6 mV &         NaN &         NaN &         NaN &         NaN \\
InjectedCurrentAPWidthTest     &                        1.32 ms &                  0.41 ms &                            1.21 ms &      1.61 ms &            NaN &            NaN &            NaN &            NaN &         NaN &         NaN &         NaN &         NaN \\
InjectedCurrentAPAmplitudeTest &                       86.36 mV &                 71.23 mV &                           80.44 mV &      68.4 mV &            NaN &            NaN &            NaN &            NaN &         NaN &         NaN &         NaN &         NaN \\
InjectedCurrentAPThresholdTest &                       -47.6 mV &                -46.89 mV &                          -42.74 mV &     -38.9 mV &            NaN &            NaN &            NaN &            NaN &         NaN &         NaN &         NaN &         NaN \\
FITest                         &                            NaN &                      NaN &                                NaN &          NaN &            NaN &            NaN &            NaN &            NaN &  0.18 Hz/pA &  0.12 Hz/pA &  0.18 Hz/pA &  0.09 Hz/pA \\
\bottomrule
\end{tabular}}
\caption[Reported Cell Features]{Reported Cell Features for 12 cells.
The first 4 are taken from NeuroElectro and the remaining 8 from the Allen Cell Types database.
Unreported data values are encoded as ``NaN".}
\label{table:tests_derived_from_reports}
\end{table}

I then used three different models (AdEx, Izhikevich, and a conductance-based point neuron) and optimized them using these test suites. I attempted to also use GLIF models here, although test results often looked convincing, GLIF model waveforms looked strange.
The result is $3 \times 8 = 24$ model-data combinations.
For each each member of this $24$ element matrix we wanted to know if the fitted model behaved in a biological plausible manner, we were interested to know if fitted models were convincing mimics of \emph{in vivo} cells, at least with respect to the measurements models were trained to fit.
In the process of coalescing results the single 24 element matrix, the matrix was broken into two: one 24 element matrix (Table \ref{tab:main_chi2})
and a smaller matrix consisting of only the conductance based model test combinations that executed without failure (Table \ref{tab:HH_chi2}).

\subsection{Across Model Performance Comparisons}
\begin{table}
\resizebox{\textwidth}{!}{
\begin{tabular}{|c|c|c|c|}
\toprule
Model Type & Experimental Data ID & $\chi^{2}$ & p-value  \\
\midrule
        IZHI &              4824937611 &     0.034791 &  1.000000e+00 \\
        IZHI &              4718194011 &     0.051691 &  1.000000e+00 \\
        IZHI &              6238931771 &     0.086530 &  9.999999e-01 \\
        IZHI &              6239608801 &     0.001550 &  1.000000e+00 \\
        IZHI &              482493761 &     1.292924 &  9.956362e-01 \\
        IZHI &              471819401 &     1.443618 &  9.936050e-01 \\
        IZHI &              623893177 &     1.334789 &  9.951233e-01 \\
        IZHI &              623960880 &     1.014464 &  9.981553e-01 \\
        IZHI &              Olfactory Bulb Mitral Cell &  6915.484007 &  0.000000e+00 \\
        IZHI &  Neocortex pyramidal cell layer 5-6 &     2.443396 &  9.643182e-01 \\
        IZHI &            Cerebellum Purkinje cell &    19.885113 &  1.077956e-02 \\
        IZHI &      Hippocampus CA1 pyramidal cell &     1.273070 &  9.958661e-01 \\
\bottomrule
\end{tabular}}
\caption[Fit Quality for AdEx and Izhikevich Models]{Fit quality for two model types optimized against the data in Table \ref{table:tests_derived_from_reports}.
Lower $\chi^2$ and higher p-value represents a better fit.
p-values > 0.05 lack evidence to reject the null hypothesis that the optimized model comes from the same distribution as the experimental data.
p=1 indicates perfects agreement between the optimized model and the experimental data on all measured features.}
\label{tab:main_chi2-izhi}
\end{table}

\begin{table}
\resizebox{\textwidth}{!}{
\begin{tabular}{|c|c|c|c|}
\toprule
 Model Type & Experimental Data ID & $\chi^{2}$ & p-value  \\
\midrule
       ADEXP &              4824937611 &     0.000017 &  1.000000e+00 \\
       ADEXP &              4718194011 &     0.011029 &  1.000000e+00 \\
       ADEXP &                          6238931771 &     0.005062 &  1.000000e+00 \\
       ADEXP &              6239608801 &     0.000083 &  1.000000e+00 \\
       ADEXP &              482493761 &    86.949529 &  1.887379e-15 \\
       ADEXP &              471819401 &     0.370856 &  9.999575e-01 \\
       ADEXP &              623893177 &     0.273030 &  9.999870e-01 \\
       ADEXP &              623960880 &     0.140222 &  9.999990e-01 \\
       ADEXP &              Olfactory Bulb Mitral Cell &    10.353878 &  2.410614e-01 \\
       ADEXP & Neocortex pyramidal cell layer 5-6 &     0.013262 &  1.000000e+00 \\
       ADEXP &            Cerebellum Purkinje cell &   353.692447 &  0.000000e+00 \\
      ADEXP &      Hippocampus CA1 pyramidal cell &     0.735616 &  9.994308e-01 \\
\bottomrule
\end{tabular}}
\caption[Fit Quality for AdEx models]{Same as Table \ref{tab:main_chi2-izhi}, but for AdEx models.}
\label{tab:main_chi2-adex}
\end{table}


\begin{table}
\resizebox{\textwidth}{!}{
\begin{tabular}{|l|l|l|r|r}
%\begin{tabular}{cccc}
\toprule
Model type & Experimental Data ID & $\chi^{2}$ & p-value \\
\midrule
conductance model & Hippocampus CA1 pyramidal cell & 17.21 &  0.027 \\
conductance model & olf-mit & 26487.51 &  0.0 \\
conductance model & Neo cortex pyramidal cell layer 5-6 &  2.56 & 0.95 \\
conductance model & 4824937611 &   2.17 &  0.97 \\ 
conductance model & 471819401 &  0.870 &  0.99 \\
conductance model & 482493761 &  0.036 &  0.99 \\
conductance model & 6238931771 & 1.441259 & 0.993641 \\
\bottomrule
\end{tabular} 
}
\caption[Fit Quality for Condutance-Based Models]{Same as Table \ref{tab:main_chi2-izhi}, but for a single-compartment conductance-based model.}
\label{tab:HH_chi2}
\end{table}


%See appendix:\ref{table:static_electrical_properties}
As stated in the Methods, I use the $\chi^2$ statistic as a summary of optimized model quality, with smaller values reflecting fits that better recapitulate the biological experimental data, and non-significant p-values--lack of evidence that the model disagrees with that data--as evidence of success.
The Izhikevich Model and a Conductance-Based Point Neuron Model were able to achieve such small $\chi^2$ statistics (and non-significant p-values) when seven or eight of the tests (excluding rheobase) were considered together.
For example the Izhikevich model fitted to a Hippocampus CA1 pyramidal cell data achieved ($\chi^2$, p-value) = (2.13, 0.98), and the Olfactory Bulb Mitral cell achieved ($\chi^2$, p-value) = (2.02, 0.98).
A model whose every feature was exactly equal to the mean observed in the experimental data would have all Z-scores equal to 0, a $\chi^2$ statistic of 0, and a p-value of 1, by definition.
Thus an extremely high p-value (such as those above) is evidence that the optimized model is much closer to the mean of the data distribution than a random experimental neuron.
This is exactly the result one would expect from successful optimization.

\subsection{Sources of Optimization Failures}
When optimization was not successful, was this due to a fundamental inability of a given reduced model to represent the behavior of a given neuron type?
In order to make this claim, I first had to rule out alternative possibilities.

\subsubsection{Distributions not Well-summarized by the Mean}
The optimizer fits to the mean of a feature value, but as shown in section \ref{sec:neuroelectro}, the mean value of a feature is in some cases a misleading summary of the typical values.
For example, the olfactory bulb Mitral cell exhibited bi-modal feature distributions (possibly due to lumping with the Tufted cell, a distinction that may not have been appreciated when the data was originally collected).
Beyond that, the Neuroelectro data reflects a mean over different laboratories, animals, and recording epochs. The mean of a population can often be a robust summary, however there are circumstances when this is not true.
I plotted all the feature distributions for all the neurons used here (see Appendix), examined these by eye and made note of those where the mean was not a good summary of the typical value (due to multimodality, extreme skew, or small sample size).
Note that this only applies to the Neuroelectro data; the other data sources report features for single instances of neurons, so the model being fit is a model of that specific neuron, not of a neuron type more generally.

\subsubsection{Distributions with an Uncertain Mean}
The NeuroElectro data represented distributions over cells of the same nominal type.
For some cell types, reduced models were hard to optimize against these data even when the distributions are normal-like, and thus the mean and the mode are well-matched.
In order to understand whether the source of these difficulties was in the data themselves, I examined the standard error of the mean (SEM) for each feature.
Like the standard deviation, the SEM is a prediction of uncertainty in each measurement, but it reflects uncertainty about the value of the mean itself, rather than simply the variability in the measurement across neurons of the same type.
A large SEM might reflect such variability, or simply a small sample size. 
In either case, a large SEM means that the optimization target may not reflect the true properties of a typical neuron of that type.
These SEM values, for several cells and electrophysiological features, are shown in Table \ref{table:neuroelectro-sem}.

\begin{table}
\resizebox{\textwidth}{!}{
\begin{tabular}{lrrrrrrr}
\toprule
{} &  Rheobase &  SpikeThreshold &  SpikeHalfWidth &  SpikeAmplitude &  MembraneTimeConstant &  RestingMembranePotential &  InputResistance \\
Neuron Type                              &           &                 &                 &                 &                       &                           &                  \\
\midrule
Hippocampus CA1 pyramidal cell     &    122.88 &            1.85 &            0.12 &            3.68 &                  3.88 &                      0.72 &            12.57 \\
Olfactory bulb (main) mitral cell  &       NaN &            5.69 &            0.12 &            2.83 &                  5.42 &                      1.39 &            20.17 \\
Cerebellum Purkinje cell           &    419.81 &            2.00 &            0.05 &            0.57 &                   NaN &                      3.69 &            19.26 \\
Neocortex pyramidal cell layer 5-6 &    128.87 &            1.92 &            0.17 &            1.49 &                  2.56 &                      1.84 &            27.93 \\
\bottomrule
\end{tabular}
}
\caption[Standard Error of the Mean Across NeuroElectro Data Sources]{The standard error of the mean (SEM) described the uncertainty that the sample mean value of a feature (across neurons) is close to the population mean.
The SEM values describe that uncertainty for the sample means of various features for each of 4 cell types in NeuroElectro.}
\label{table:neuroelectro-sem}
\end{table}

%\begin{comment}
%Below I show distributions for  Time Constant, Input Resistance, Capacitance, Rheobase, Resting Membrane %Potential, bimodal distributions did not apply, so we could rule out inaccurate data as a reason for poor %model performance.

%\begin{comment}

%There is no reason to believe that reduced neural models could not be made to fit inaccurate neural recordings %as well as real ones, if the fiction is just caused by noise, then it is still possible that hypothetical %spurious values would still be in reach of the Izhikevich model. The izhikevich model can be made to generate %some physiologically implausible waveform shapes. 
% a different question, of are the models arbitary waveform generators? 

%Besides even if the data was wrong, it wouldn't necessarily follow that models couldn't reproduce the %inaccurate data. Instead what we see is, that models can fit one type of experimental measurement at a time, %but they can't fit all measurements at once. This result suggests that model flexibility is the most %fundamental cause of modest, model experiment disagreement.


%In light of this result, one might ask, i
%If reduced models are not excellent at fitting data, are brain simulations really that much more realistic, %when we use data driven fitted models in the place of generic model parameters? 

% To answer this question we would need to run brain simulations. From my own experience adding in realistic %cells to pre-existing network topologies requires that the network be re-tuned to demonstrate tonic firing %with realistic CV again.

%If data driven model fits lead to models that fit one measurement better than others, which fitness criteria %will lead to the greatest consequence for network simulations?
%\end{comment}


%to answer that question we have created some large

%\subsection{Limitations of Existing Approaches} 
%Existing community supported simulated models were problem ridden, and our own custom methods were used as work arounds.
%data we were using was
% NEURON version of Izhi model is not as fast as one might expect, this may because of the way we tried to implement the model, by creating and destroying HOC module instances, that contain the model. 

%A
% Nonetheless, 

% This should not go in the general introduction, but in the intro to the appropriate section of the results where you use those models.