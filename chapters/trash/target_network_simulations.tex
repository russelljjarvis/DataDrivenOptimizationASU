


In reduced neuronal It was found that the electrical properties of cells:
 (Rheobase, input resistance, membrane time constant,Resting Potential, Capicitance)

 did not encode the above threshold spiking behavior of neurons, when neurons were undergoing a multispiking stimulus regime.

 forinstance in Allen cell '482493761' injecting a current of 
 specified by $ current = {'amplitude':110*pq.pA,'duration':1100*pq.ms,'delay':100*pq.ms} $
 should elicit a spike count of $3$, in an optimized Izhikevich model that has good agreement with all values except for Rheobase these values it produces 33 spikes.


 You can imagine that in cells, that have been optimized to rheobase only, that current spike count relationship might be more 
 accurate and faithful to the experiment.

In cell 471819401 , 290 pA causes 20 spikes,
in the optimized Adaptive Exponential model 290
specified by $ current = {'amplitude':290*pq.pA,'duration':1100*pq.ms,'delay':100*pq.ms} $
caused only $5$$ spikes. 

and in the Izhikevich model $59$
RheobaseTest	TimeConstantTest	RestingPotentialTest	InputResistanceTest
observations	70.0 pA	24.4 ms	-71.6 mV	132.0 megaohm
predictions	234.67 pA	31.31 ms	-72.08 mV	130.26 megaohm
Z-Scores	3.98	0.09	0	0.01

Another way of writing this is that fitting a model to electrical properties of cells, does not
garuntee that the cell is fitted to multispiking behavior, at current injection virtual experiments with larger amounts of injected current.
A model fitted to one "sweep" is a model of only that sweep. 

If you apply the sweep fitted model to a different current injection value, you will not get the expected waveform shape fitted.
because even though one may have optimized to the shape of a single spike, and the membrane time constant, and input resistance should have a bearing
 on spike width and spike height respectively. Since spike width is dependant on membrane time constant, and input resistance, you 
 would expect that spike frequency and therefore firing rate, should be proportionately related to 
 to the membrane spike width, and therefore membrane capacitance. However, in reduced neuronal models, this was found
 to be not strictly the case.

 The Rheobase value is the biggest determinant of how current injection values map onto spike counts.


 Arguably then only the rheobase current injection value is important for fitting to Reduced Neuronal Models in order to make networks of Reduced Neuronal cells behave more realistically.
 It is unclear, however, what the relationship is between synaptic current and electrode current. Synaptic current is distributed, but electrode current is focused on a single point.
 If both are distal to the axon hillock, both synaptic current, and injected currents may behave in a similar way.


 These model parameters do not encode the spike frequency.
you didn't optimize to the spike count at different injection values.

Therefore in order to make realistic network models that make use of reduced models, 
one should have a firing rate, and some sweep data, of cellular behavior with the same spike frequency.
one can then fit simple multi-spiking waveform measurements to the models.

Surprisingly some types of spiking information, like spike rate, and multispiking height, spike adaption.
Are easier for reduced models to fit. I show this in some simulated data, virtual experiments.

With spike counts, and spike frequencies, 
