\documentclass{report}
\usepackage[utf8]{inputenc}
\usepackage{graphicx}
\usepackage{enumitem}


\renewcommand{\labelitemi}{$\bullet$}
\renewcommand{\labelitemii}{$\cdot$}
\renewcommand{\labelitemiii}{$\diamond$}
\renewcommand{\labelitemiv}{$\ast$}
\addcontentsline{toc}{part}{REFERENCES}
\bibliographystyle{chapters/minimal}
\bibliography{dis}

\begin{document}

%\begin{itemize}
%\item Also, you will want both a label general introduction and label general discussion, but also within each chapter you will need a mini-introduction that explains what you are aiming to do or show in that chapter and why. 

%\item That mini-introduction can be essentially like the content of the first few cells of some of your notebooks, but you will want to make sure that every sentence is comprehensible to someone who has read and understood the general introduction but otherwise doesn’t have much special expertise other than being a neuroscientists of some kind.


% Explanatory comment Name of section, above call to include.
\chapter*{Introduction}

\section{Introduction}

\subsubsection{Motivation}
\begin{itemize}
Need to explain two things:
- why optimize, but also why reduced models. Probably why models first.

\item Biophysically accurate models take a significant time and resources to evaluate. A different class of neuronal model, known as a "Reduced" model is comparatively fast to solve especially at a large scale. Speed of simulation is important for learning about the brain. Even in the many valid instances when the complete 3D shape is an integral part of a cortical brain simulation, such simulations are often enriched, by encasing a "core" of biophysically accurate models inside a "shell" of simple fast and reduced Izhikitch, GLIF, or adaptive exponential \cite{brette2005adaptive} Wulfram Gerstner models. Further more Izhikivitch, AD

\section{Introduction}

\subsection{Model  Optimization with NeuronUnit}
Some neural properties can’t be easily measured in experiments. These unknown properties hamper modeling accuracy and require parameter fitting. For example, a common approach for approximating unknown ion channel densities is to ‘optimize’ the governing equations to match known waveforms. The process of optimization involves what is known as an ‘inverse’ problem where we efficiently and sparsely search for the ‘optimal’ value of an parameter that satisfies the system of equations. Often an optimal value is corresponds to a global minimum or maximum value of a cost function.\\*

Computational optimization techniques are generally specific to a particular type of problem rather than being generalized. However, several notable algorithms have solved a wide range of problems including genetic and algorithms and stochastic gradient descent (SGD). The popularity of these two algorithms is due to their robustness. NSGA2 and SGD are able to avoid falsely reporting a local minimum when a more optimal solution is available. 
\Subsection{Multiobjective optimization} Multi objective optimization problems are a subset of optimization problems, where model fitness is evaluated against multiple independent constraints, rather than just one error. It is often possible to reduce multiple constraints into one constraint by summing the outputs of objective functions together, however the price of reducing multiple errors into one error, is that where multiple and diverse models give satisfactory solutions to the provided constraint, reducing error leads to a situation where a constraint that is easier to satisfy, rapidly drags down the error score and dominates the by contributing lower errors to the sum of error scores.

However, of SGD and NSGA2, only NSGA2 is a natural choice for tackling multi-objective optimization problems. Default implementations of SGD are not able to utilize the principle of non-domination as an optimization strategy.%%
%%
There is a great diversity of real biological neurons, all of which differ substantially in their electrical behavior. There are a few different classes of general purpose neuronal models, that can reproduce these different types of electrical behaviours, given appropriate parameterizations of the models.\newline
\newline
An exisiting class of neuron model type, called The Izhikevich model was published with parameter sets believed to make the model outputs accurately align with a variety of real biological cell outputs. However since publication much very specific electro physiological recordings have accumulated, that in someways undermine model/experiment agreement. However it is now possible to constrain the Izhikevich model and find new parameterizations that more allow us to more accurately reproduce more recently published experimental data.\newline
\newline
NeuronUnit easily converts a quantitative measure of model/data agreement into a useful error signal. A very natural application of this signal is to guide the process of optimization. We have used Neuronunit to guide optimization by taking a flexible model type such as a generalized linear integrate and fire model or the Izhikevich model and constraining the model against relevant experimental data. As an example, NSGA2 was used to optimize models in conjunction with data driven tests based on pooled data from NeuroElectro.org. A variety of compact and fast single compartment models were used to explore model optimization. Figure 4 demonstrates test error at the beginning of the optimization process for models with randomly sampled parameters and the smaller error following optimization. Figure 5 shows the evolution of the error during the optimization process. \newline
\newline
Optimized neuron models may vary from their neuron counterparts for several reasons. Table 3 shows an example where optimizing the model with respect to the rheobase test comes into conflict with minimizing with respect to input resistance. The solution to the optimization problem consists of two sets of model parameters, which can resolve this conflict differently. Examining the experimental data that these tests were derived from suddenly becomes important. By examining the data, we can see if the rheobase currents and the distributions of input resistance are bi-modal and uniformly distributed. If the data is treated as uni-modal, and the uni-modal mean is used to optimize then the model, then the model is not able to satisfy both constraints simultaneously. In this case, the measurements don’t correspond to neuron data, and the model can’t produce the artificial behavior. When comparing complex data and simple models we find that solutions are better represented using a combination of two optimization solutions.\newline
\newline


\subitem optimization is an interaction between models and constraints which guides a fitting process. 

\subitem if the combination of models and constraints is bad, then then a tractible error surface will not result.  

\subsubitem Unfortunately, it is not always possible to tell without trying which combinations of A: neural models, and B, constraints will lead to the Genetic Algorithms ability to converge on around the minimal error. 


I describe some code implementation experiments were the model/constraint combination lead to DEAP genetic algorithms matching model parameters to constraints and model/constraint selections that lead to optimizer performing no better than a random search of parameter space.


\item successful optimization means model objective choices. How model constraint combinations interact cannot always be known in advance, and the interaction has to be explored experimentally.

Efficient model examples: (generalized leaky integrate and fire model) GLIF, Izhikitch. Adaptive Exponential Integrate and fire model, single compartment conductance based model. 

\item You could make biophysically accurate models faster, or you could make reduced models more accurate. To make reduced models more accurate, you would find parameterizations of the models that let the models act as better mimics of experiments.
\item Herein we investigate how well Faster models can match experimental recording waveform shapes.
\item The reason why we want to investigate the match:
\item Large scale simulations cant evaluate on a timescale that is meaningful, unless a large ratio of modelled cells are "reduced models"
\item Reduced Models already enjoy wide spread usage. We want to investigate if reduced models can be made to be more realistic, by checking if they can mimic data better.
\item If reduced models can't be made more realistic (herein we show only marginal improvements), we need to show the limitations of reduced cells, with regards to a particular set of tests.
\item We need to document the approach used, and how the approach contrasts with spike time approaches to model fitting.
\end{itemize}
\chapter*{Methods}
  

\begin{figure}    
\begin{center}
\includegraphics[width=1.1\linewidth]{figures/software_architecture.png}
\caption{In the process of performing the analysis in this work, we expanded an existing neuron model optimisation frame work BluePyOpt \cite{bluepyopt}, by adding in workflows to handle different and elaborate constraint functions and different and elaborate models. and found the minimal current injection value that would cause only one spike. The normal structure of this algorithm is a binary search, however we modified the algorithm so it would map onto multiple processors at once. This lead to significant speed ups for multicompartment NEURON models}

\end{center}
\end{figure}

    

 \section{Model Implementations}   

A constant error warning plagued brian2 investmentallations
\begin{verbatim}
Brian2 causes error:
 ERROR      Brian 2 encountered an unexpected error. If you think this is bug in Brian 2, please report this issue either to the mailing list at <http://groups.google.com/group/brian-development/>, or to the issue tracker at <https://github.com/brian-team/brian2/issues>. Please include this file with debug information in your report: /tmp/brian_debug_t0acbm4l.log  Additionally, you can also include a copy of the script that was run, available at: /tmp/brian_script_juzhsbph.py Thanks! [brian2]
Traceback (most recent call last):
\end{verbatim}

To make optimization of models tractable it was important to do ongoing feasibility testing. For example its important to evaluate the the utility of established model implementations, as using these models to optimize may not in fact be feasible.\\ 
\\
Despite an a large number of choices of FOSS reduced model
implementations, many off the shelf implementations were not useful, or significant intervention was required to make some established implementations workable inside an optimization framework. \\
\\
In two classes of model a feasible choice of implementation did not exist and it was easier to re-implement those models. The two models I re-implemented were
the Adaptive Exponential Integrate and fire Model, and also the IZHI
model.  In the work below, I profile existing model implementations, and
justify the reasons for re-implementing.\\
\\
This is in contrast to the brian2/neuraldynamics AdExp model, which took
between 2 or 3 times longer to find a rheobase current injection value. However the slowness is not caused by the simulation backend (brian2 which is relatively fast and efficient). The slow down is caused by the way the model is defined. Specifically the
model is defined in a middle code layer neurodynamics\cite{gerstner2014neuronal}.\\
\\
It is very likely, that the model implementation is correct, since Gerstner is an author of one of the original adaptive exponential publications, and the neural dynamics book that the brian2 code is strongly affiliated with. Since Integrate and Fire models don't formally include spikes when an implementation does include spikes, it is an optional add on.\\
\\
The AdExp neurodynamics models default implementation causes spikes with peaks below $0mV$, since the IZHI model like all integrate and fire models do not explicitly include spikes\\
\\
This is not technically wrong, but it violates
assumptions in the \emph{NeuronUnit} feature extraction protocol. The default spiking behavior, looks odd, and it is simply this poor model definition that is causing a slower optimizer performance. The optimizer takes an unusual waveform shape, and searches for longer in distant
parameter regions to find a good fit.\\
\\
Over the course of evaluating the brian neural dynamics model \cite{gerstner2014neuronal}. I experienced some phenomena that only occurred in the context of genetic algorithm optimization. The reason why optimization provides a different evaluation context is because, in optimization simulation objects are required to be created and destroyed rapidly and on mass. Brian2 is designed to be an efficient network simulator, the case of being designed for network simulation, may assume you will want to create a lot of neural models that persist efficiently together in memory (this was also a problem with PyNN models). Therefore you might see below, that while only one brian model exists in memory, performance is okay, but when creating and destroying models rapidly and on mass a slow down occurs.\\
\\
Below I have implemented a python integrator for the Adaptive
Exponential Integrate and fire model. This solver lead to faster
evaluations of current injection experiments. The integrator I developed
had a $0mV$ spiked when evaluated at default
parameter values.\\


Brian2 and sciunits sometimes collided in name space, and logging.
%\href{https://github.com/scidash/sciunit/pull/124/files/83907ba68740642178ebb91084f6e382e06a43c4#diff-d68791d2ed5dfaa96a900be6180bd950}

\section{Profiling the JIT enabled AdExp Model}
Mean time taken on single model evaluation:$ 0.0012554397583007812s $
Mean time taken to compute rheobase:
$0.183s $


Even faster than Izhikevich implementation which was: $ 0.462s $ $  0.002 s$
Rheobase takes a mean of 15 model evaluations.
\begin{figure}    
\begin{center}
\includegraphics[width=0.25\linewidth]{figures/backend_check_files/backend_check_6_2.png}
\caption{}

\end{center}
\end{figure}

%\begin{verbatim}
%    251 ms +- 5.02 ms per loop (mean +- std. dev. of 2 runs, 1 loop each)
%    240 ms +- 11.1 ms per loop (mean +- std. dev. of 2 runs, 1 loop each)
%    223 ms +- 12.5 ms per loop (mean +- std. dev. of 2 runs, 1 loop each)
%\end{verbatim}

\begin{verbatim}
    922 ms +- 12.7 ms per loop (mean +- std. dev. of 7 runs, 1 loop each)
\end{verbatim}

\subsection{Compare parallel to serial speeds, and accuracy}

Below is the Brian2/NeuralDynamics AdExp model. In-order to make the spike height greater than $0mV$ it was easier to use computer code to schedule waveform modifications that occur straight after the the brian2 simulation, these scheduled waveform modifications can be considered part of a peripheral shell of simulation code. In postprocessing
the waveform data type is a Neo Wave form object that is artificially the algorithm of determining rheobase and displaying results. The time of this model is determined on multiple factors, as discussed elsewhere, execution time is not uniform across model parameterizations. Models with multispiking behavior will take longer to solve.

Simulation times for this model vary, dramatically possibly because of
lazy evaluation, the simulation times may vary according to what else
you are running on your computer. Not all models experience a speed up when executed in parallel, however
this model was faster in the parallel Rheobase determination algorithm. Some common times are: $3.92,6.75,4.48,5.17$. Mean time was:

\subsection{Comparison of Times Taken to Find Rheobase}
Custom implementation JIT enabled implementation: $4.0s$. 
Brian2 taken to find Rheobase: $4.40s$ (serial), $3.976s$ (parallel).

The evaluation times between Brian2 and the custom written
integrator are similar. Both have average rheobase solution times of approximately 4 seconds, however the spike shape derived from the custom written integrator look more realistic under default paramaterizations. The biological plausibility of default model paramaterization has consequences for model optimization speed, because when  models undergo mutation and cross-over the mean of random models regressors towards the default model initialization, and if the default model is a bad fit to data, the average model sampled by the genetic algorithm will also be bad to data.\\

\begin{center}
\begin{figure}
\includegraphics{figures/backend_check_files/backend_check_4_2}
\caption{Default model parameterization of the custom written integrator}
\end{figure}
\end{center}


\begin{center}
\begin{figure}

\includegraphics{figures/backend_check_files/backend_check_12_10}
\caption{Model parameterization of the brian2 simulator with the customization: interpolated spike height, forced to be above $0mV$}
\end{figure}

\end{center}
    
\begin{verbatim}
    272 ms +- 66.5 $/mu$s per loop (mean +- std. dev. of 2 runs, 1 loop each)
\end{verbatim}

The next model to be evaluated is the NEURON Izhi model. The NEURON Izhi model has various draw backs. 1. It depends on an external file which must be recompiled each time this project is recreated. 2. The build environment of NEURON is non-trivial, and only a super dedicated NEURON modeller would install it on their system. Any performance advantage of using NEURON investment does not exceed the installation cost of installing the program. 3. The model implementation code is less generalizable than than the published Izhi model itself. Where the standard NEURON-NeuroML code only covers the Regular-Spiking model * This is likely due to a name space conflict between Capacitance. Neuron has a `capacitive' mechanism inside modelled Neurons, this particular model has section capacitance as well as an introduced capacitive term inside a C-compiled mechanism. Both contribute to a the membrane
potential calculation. * The NEURON Izhi model took $78$ seconds to find the rheobase current injection value $ 51.79367065 * pA $.

    
%\begin{center}
 %   \includegraphics[width=0.7\textwidth,]{chapters/figures/backend_check_files/backend_check_14_2.png}
%    \caption{where is picture}
%\end{center}


%\begin{figure}
%    \centering
%    \includegraphics{chapters/normal_distribution}
%    \caption{This is your image%}
%    \label{fig:my_label}
%\end{figure}



% https://www.overleaf.com/learn/how-to/Images_not_showing_up 
        
    The forward Euler python IZhi model is very fast. The forward euler
implementation utilized Numba JIT. Rheobase is found in under a second,
and in many cases close 0.5 seconds. This represents a very dramatic
speed up.

Unlike the NEURON NeuroML implementation of the izhikitich equation,
this implementation is just as generalizable as the original MATLAB
implementation of the izhikitich model.


\begin{verbatim}
  time taken on
  block 0.6859951019287109 \textbackslash{}n3.3 ms +- 9.79 $\mu$s per loop (mean +- std. dev. of 2
  runs, 100 loops each)\textbackslash{}n3.32 ms +- 30.9 us per loop (mean +- std. dev. of 2 runs,
  100 loops each)\textbackslash{}n3.19 ms +- 10.9 us per loop (mean +- std. dev. of 2 runs, 100

\end{verbatim}
        
\section{Python/LEMS and NEURON versions of single compartment Conducance Model.}

Conductance based models took approximately the same amount of
time to evaluate the Rheobase search algorithm as the python
implementation.

%This problem in the default parameterization of the python model was later located in the scale or units of capacitance, if default capacitance parameterization is multiplied by 100.0 the problem goes away.

time taken on block $ 12.6s $

\begin{center}
\includegraphics{figures/backend_check_files/backend_check_22_2}
\end{center}

$ 1.40762329 * pA $


\subsection{NEURON versions of single compartment Conducance
model.}

Hodgkin Huxley Conductance based channels models took approximately the same amount of time to evaluate the Rheobase search algorithm as the python implementation.

%The NEURON implementation was slightly faster, and the default parameterization of the model lacked `ringing'', or below threshold oscillations that the Python ODE version had under default conditions.

%This problem in the default parameterization of the python model was later located in the scale or units of capacitance, if default capacitance parameterization is multiplied by 100.0 the problem goes away.

    \begin{verbatim}
time taken on block 8.573923826217651
    \end{verbatim}


    %\graphicspath{ {../figures/} }
    \begin{center}
    \begin{figure}
    \includegraphics{figures/backend_check_files/backend_check_26_2}
    %kend_check_files/backend_check_26_2.png}
    \end{figure}
    
    \end{center}
\begin{verbatim}
112.5 pA
'value': array(1.40645904) * pA
\end{verbatim}


\begin{verbatim}
\{'El\_reference': -0.07016548013687134, 'C': 3.990452661875942e-10,
'init\_threshold': 0.02964956889477108, 'th\_inf': 0.02964956889477108,
'spike\_cut\_length': 109.5, 'init\_voltage': -35.0, 'R\_input': 910258965.9792937\}
time taken on block 0.23476457595825195
\end{verbatim}

    


$ 112.5 pA $
$0.0 mV$ $-0.065 mV$

    \begin{verbatim}
    \{'value': array(183.33333333) * pA\}
    \end{verbatim}

\begin{verbatim}
array(112.5) * pA
\end{verbatim}


\begin{verbatim}
    0.017240506310425608 mV -0.08583939747094235 mV
    0.017240506310425608 mV -0.08583939747094235 mV
\end{verbatim}

    \begin{center}
    \includegraphics{figures/backend_check_files/backend_check_32_2.png}
    \end{center}



\section*{Technical Details of the Optimizer}
\begin{itemize}

\item Experimental Recording Features from Neuroelectro. And some from the Allen SDK.
\item I am using selBest and NSGA2 to optimize currently.
\item Python, NeuroUnit, 

* Model/Test combinations

Contrast with other optimization approaches:

\item Other optimizers, rheobase current injection used is a hard constraint. However, this work differs from other work in that 
\item Rheobase value as a soft constraint. This means that current injection values are a variable model parameter. The exact value of current used is determined via exploring the model response to current injections of varying amplitude. The exact value, its value is determined by other model parameters.

\item spike shape measurements and error functions. We used a set of 8 different experimental measurements NeuronUnit.
\end{itemize}

\chapter*{Results}

\begin{itemize}
\item  $N-free-model-parameters << N-constraints$
\item  In our specific design this expands to:
\item  $(model parameters + current-injection-value-parameter) << N (independent and uncorrelated)$ constraints.
\item  For the different classes of Reduced Model we show that the optimizer converges when data is simulated.
\end{itemize}

In a simulated experiment, existing models were instantiated using a randomly chosen model parameters.

In the class of reduced neural models we are optimizing, are not arbitrary waveform generators. The models have intrinsic restrictions that prevent them from matching perfectly with experimental waveforms.

When constraints are derived from model measurements, intrinsic model restrictions no longer apply. Optimized models should match perfectly with the simulated experiments. 

* Failure to match is indicative of: -- Failure to setup tractable optimization problems. Too higher a dimension.

- When inverting linear equations, finding a unique solution requires that the number of constraining equations is greater than the number of free variables you are solving for:

\section{Pitfalls}
Choosing optimizer constraints, that cause visible ripples in error surface.
To protect against a situation where the collection of error sources guiding optimization are too correlated with each other, to act as 


\section{Verification}
Ground truths are model solutions that we know are correct independently from the optimizer. One way to establish ground truths is to identify the global minima by exhaustively searching the solution space. An exhaustive search is a reasonable approach when you consider only one or two model parameters are free parameters, however if one does 100 samples in each of N dimensions then one must make samples $100^{N}$ total samples to be sure of ehaustively searching, assuming the most efficient code, hardware and development time $N=3$, may be the highs.  Also the choice of 100 samples is nominal, 100 samples could be either too fine or too sparse, depending on how if the parameter being searched exhibits 2nd order sensitivity.

It is more computationally efficient to obtain ground truths by simulating constraining data using digital models. It is easy to simulate constraining data, all that is required is that you take a neuronal model and measure its behavior in response to carefuly chosen current injection values. Measured behavior can then be used to construct NeuronUnit tests, were the measurements become "observations", or observed behaviors. To make the simulated data cover a range of circumstances, one can make different NU measurements by randomly choosing different parameter values of models to find.

It was important to be able to establish ground truths that were always possible for the optimizer to match exactly. Often experimental data implies waveform shapes that are beyond the capabilities of the model that is to be fitted. Simulating experimental measurements meant, that model limitations can be understood separately from optimizer limitations.

\section{Optimizer Limitations}


\begin{center}
    \begin{figure}
    
        \includegraphics[width=\linewidth]{figures/correlated_errors.png}
    
    	\caption{Test caption}
    \end{figure}
\end{center}

\subsection{chapter over view}
\begin{itemize}
\item to assess the accuracy of current neural modelling work we explored:
\item variance in models
\item variance in experiments,  
\item variance in the combined set of models and experiments.
\item combine this discussion with model re-purposing discussion.
%\item model re-purposing?

\end{itemize}

%Druckmann \cite{druckmann2008evaluating}

%\cite{buil}


\subsubsection{Cortical Model and Cortical Experiment Agreement}

NeuroML-DB \cite{birgiolas2016rapid} catalogues over 1,500 published models obtained in NeuroML format from Open Source Brain [5]. Complementing OSB, NeuroML-DB provides systematic characterizations of model complexity, electrophysiology, and morphology, making it easy to find, evaluate, and reuse models and their components.\\
\\
It is known that generally that neural models and experimental measurements diverge in some respects,  however, we needed to locate specific sources of divergence. Specific knowledge of model/data disagreement informs the question: "In  cortical neuron models which aspects of a voltage recordings should be prioritized as optimization constraints?"

\subsubsection{Features} Consider a voltage recording at the location of the membrane of a neuron. Teams of researchers have already segmented voltage recordings into labelled sections, each section has a classification that is based on the shape of waveform in a limited region. Rather than specifying by name each measurement it is often useful to refer collectively to these measurable shapes as "features". In a multivariate analysis we analyze hundreds of such features, and we summarize important differences in a subset of this high dimensional feature space. 

Below, I describe some neuronal model features that agreed well with experiments, and some features that diverged.


\subsubsection{Publications Associated with Model Sources}

\begin{table}[ht]
\centering
\resizebox{\textwidth}{!}{
\begin{tabular}{lll}
\toprule
{} & Large Scale Model &   Publication 
Allen Institute V1 Model &  \cite{gouwens2018systematic}
Somatosensory Cortex & \cite{markram2015reconstruction}
\bottomrule
\end{tabular}}
\end{table}

\subsubsection{Feature Extraction Libraries}
\begin{table}[ht]
\centering
\resizebox{\textwidth}{!}{
\begin{tabular}{lll}
\toprule
{} & EFEL Ephys Feature Extraction Library & AllenSDK & Druckmann (2012) 
\bottomrule
\end{tabular}}
\end{table}


\subsubsection{Virtual Experiment Three Step Protocol Stimulate for 2s}

\begin{table}[ht]
\centering
\resizebox{\textwidth}{!}{
\begin{tabular}{lll}
\toprule
{} Injection 1 & Injection 2 & Injection 3 &
 at 1.0×Rheobase & at 1.5×Rheobase & at 3.0×Rheobase 
\bottomrule
\end{tabular}}
\end{table}



%\includegraphics[]{chapters/app_tex/Allen_rush}
\begin{figure}
    \begin{center}
    
    \includegraphics[width=0.6\linewidth]{figures/multi_spiking_large_allen}
    \caption{A voltage recording from a suprathreshold experiment waveform used as a basis for the Allen Brain Institute cell types data base. Publication Gouwens rat \cite{gouwens2018systematic}}

    \end{center}
    
\end{figure}    

\begin{figure}  
    \begin{center}
    
    \includegraphics[width=0.6\linewidth]{figures/multi_spiking_large_bbp}
    \caption{An example of a multispiking waveform used as a basis for the Blue Brain Project. Publication Jouvanile rat \cite{toledo}}


    \end{center}
\end{figure}    



%\end{tcolorbox}
\begin{figure}    
\begin{center}

\includegraphics[width=1.0\linewidth]{figures/cortical_model_data_agreement_52_1.png}
\caption{}

\end{center}
\end{figure}    
\begin{figure}    
\begin{center}
\includegraphics[width=1.0\linewidth]{figures/cortical_model_data_agreement_54_1.png}
\caption{}
\end{center}
\end{figure}    
\cite{wang2019sag}
\begin{itemize}
    \item upstroke\_t\_1.5x allen feature
    \item  peak\_t\_1.5x allen feature
    \item threshold\_t\_1.5x allen feature
    \item fast\_trough\_t\_1.5x allen feature
    \item fast\_trough\_t\_3.0x allen feature
    \item upstroke\_t\_3.0x allen feature
    \item peak\_t\_3.0x allen feature
    \item threshold\_t\_3.0x allen feature
    \item peak\_indices\_1.5x efel feature
    \item min\_AHP\_indices\_1.5x efel feature
\end{itemize}


\begin{itemize}

    \item fast\_trough\_index\_1.5x allen feature
    \item fast\_trough\_index\_3.0x allen feature
    \item threshold\_index\_1.5x allen feature
    \item peak\_index\_1.5x allen feature
    \item upstroke\_index\_1.5x allen feature
    \item peak\_index\_3.0x allen feature
    \item upstroke\_index\_3.0x allen feature
    \item threshold\_index\_3.0x allen feature
\end{itemize}



\chapter*{Discussion}

{\singlespace
% Making the references a "part" rather than a chapter gets it indented at
% level -1 according to the chart: top of page 4 of the document at
% ftp://tug.ctan.org/pub/tex-archive/macros/latex/contrib/tocloft/tocloft.pdf
%\addcontentsline{toc}{part}{REFERENCES}
\addcontentsline{toc}{part}{REFERENCES}
\bibliographystyle{asudis}
\bibliography{chapters/dis}}



\end{document}
