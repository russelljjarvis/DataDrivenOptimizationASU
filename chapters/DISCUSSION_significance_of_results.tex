\section{Discussion}

\subsection{Poor Ability of models to Fit Data.}
\begin{itemize}


\item Within a reasonable parameter range conductance based models are usually close to experimentally observed measurements.

\section{A pattern to model fitting inability in Phenomenological Models.}
* many reduced neural models are far from optimal at any point in parameter space. Optimizing does not significantly improve agreement. Optimizing does not bring model and experiment close to `Z=0`.

\item There is an order of magnitude difference. Between agreement in Rheobase values between the conductance based models and the experimental models. 

\end{itemize}

\section{What does it mean.}
\begin{itemize}

\item  In methods I discussed an approach to verifying the optimizer.
- Specifically, we showed efficient optimizer convergence when constraints were derived from simulating experiments.

When the optimizer setup is cogent because appropriate models and test combinations have been chosen, and because 
\begin{equation}

N free_params <= N constraints
\end{equation}

The optimizer finds a varied set of compromised solutions.

\item  In order to verify our approach we also re-implemented our code using BluePyOpts select best algorithm, and NSGA2.
\end{itemize}

Model constraint combinations give rise to differences in how correlated errors are during gene evolution. 