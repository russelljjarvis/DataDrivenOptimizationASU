\section*{Technical Details of the Optimizer}
\begin{itemize}
\item Experimental Recording Features from Neuroelectro. And some from the Allen SDK.
\item 
\end{itemize}

Four factors needed to be controlled for, before we locate the cause of poor model/experiment fits. Those factors were: the informativeness of measurement errors, the performance of the optimizer, data quality, and model quality.

The digital models we used were known to not be perfectly faithful to real cells, they are not flexible enough to be fitted to all experimental waveforms. Additionally the data sources may also have some fidelity problems, it was necessary to create data source independent "ground truths", by synthesising plausible data, with the digital models, and optimizing against these ground truths. Because the DEAP genetic algorithm, has been shown to be able solve Rastrigrins function, we expect that our derivative frame work, should be able to fit to simulated data sources, with a high degree of precision, and that is what we found.



We had to show that the optimizer was capable of producing perfect fits under ideal circumstances. We showed that the measurements the genetic algorithm was using were able to act as informative guides. 

Since we are confident about the optimizers ability to fit to synthesized data

It is possible that some selections of experimental data, might be supurious or compromised in some way.  

The algorithms IBEA and NSGA2 seem to result in different optimization convergence speeds, and similar solution quality, but this varied in a problem specific manner. Generally it was found that IBEA produced better results faster, although, often it could produce solutions that were significantly dominated in one error score.

Typical model parameters were number of generations $(NGEN=150$, population size:$MU=25$. There was plenty of opportunities to change these parameters,  values of $NGEN$ and $MU$ as small as $(10,10)$ were tolerable under some limited circumstances, and also multiobjective optimizations with $NOBJ>25$ demanded $NGEN=200$, and $MU=50$ in order to unearth good results.

As you can see in the graph below when the measurements:time constant, capacitance, Rheobase, Resting potential and Input resistance are used to constrain optimization. 

\begin{longtable}{llll}
\toprule
{} &                     observations &    predictions & Z-Scores \\
\midrule
\endhead
\midrule
\multicolumn{4}{r}{{Continued on next page}} \\
\midrule
\endfoot

\bottomrule
\endlastfoot
\textbf{RheobaseTest                  } &                         225.0 pA &       225.0 pA &        0 \\
\textbf{TimeConstantTest              } &                          1.81 ms &        1.75 ms &     0.01 \\
\textbf{RestingPotentialTest          } &                        -62.88 mV &      -62.88 mV &        0 \\
\textbf{InputResistanceTest           } &  34339711.37 kg*m**2/(s**3*A**2) &  35.73 megaohm &     0.01 \\
\textbf{CapacitanceTest               } &          0.0 s**4*A**2/(kg*m**2) &       48.97 pF &     0.03 \\
\end{longtable}

AP width, amplitude, and AP threshold errors, did not participate in guiding optimization, because as I have explained, those measurements are not objective between model instances, and thus they lead to misleading error measurements. These tables show that the optimizer can re-cover ground truths that are derived from simulated data. This result allows us to confidently argue that in fitted models, the cause of model/experiment disagreement must be located in either the data, or the models, but not the optimization process, or the choice of error signals which are demonstrably sound.

\begin{longtable}{llll}
\toprule
{} &                     observations &    predictions & Z-Scores \\
\midrule
\endhead
\midrule
\multicolumn{4}{r}{{Continued on next page}} \\
\midrule
\endfoot
\textbf{InjectedCurrentAPWidthTest    } &                            0.0 s &         0.7 ms &     0.17 \\
\textbf{InjectedCurrentAPAmplitudeTest} &                        143.62 mV &       80.55 mV &     0.19 \\
\textbf{InjectedCurrentAPThresholdTest} &                        -39.98 mV &      -39.71 mV &        0 \\
\end{longtable}

\includegraphics[]{figures/simulated_convergence_performance.png}

\includegraphics[]{figures/simulated_data_convergence_passive_fits.png}
%\item Python, NeuroUnit, 

\subsection{Model/Test combinations}

Contrast with other optimization approaches:
There are two different strategies, one strategy involves evaluating goodness of fit, only at one current injection strength for every gene. Another approach is a bit more generous, in that, rather than allowing models to instantly fail to evoke a spike at the appropriate current injection, that models specific rheobase value is found, and all other tests are evaluated at the current injection value. The consequence of accomodating poorer fits on the Rheobase test, is allows models to compete on the other seven or more different fitness criteria.

\subsection{Elephant Test Suite}
At a count of eight tests (five useable) the elephant derived suite was relatively compact. The elephant tests suite is concerned with core ephysiology measurements, the particular measurements chosen were important because, for example capacitance, and input resistance directly affect the evolution of reduced model governing equations. Taken as a collection these ephysiological properties as a collection also determine spiking waveform shape. 

Capacitance, Rheobase, Time Constant, Input Resistance, Resting Potential

%\item 
Result It when models were fitted using the following measurements alone, the resulting models were also fitted to spike half-width, arguably because the membrane time-constant, and membrane capacitance together encode spike-half width.

\subsection{Electrophysiology Feature Extraction Library}

Many of the EFEL measurements, and Allen SDK Feature measurements pertain to spike train statistics, because model fitting to spike train statistics is a productive and well understood domain, we did not utilize spike train statistics in model fitting. Instead EFEL spike shape and electrophysiology measurements were applied.\\
\\
Through trial and error experimentation the following EFEL features were demonstrated to result in helpfull and tractible error surfaces. In simulated constraint paradigms using the aforementioned EFEL fitness criteria, lead to recovery of optimal models. This finding did not apply to the entire collection of EFEL features however.\\
\\
In a large scale analysis of variance between models and data see section large\_scale\_variance

%\href{run:./RESULTS_large_scale_variance}{RESULTS_large_scale_variance}

\begin{verbatim}
'AHP_depth_abs_3.0x','sag_ratio2_3.0x','ohmic_input_resistance_1.5x','sag_ratio2_1.5x','peak_voltage_3.0x','peak_voltage_1.5x','voltage_base_3.0x','voltage_base_1.5x','Spikecount_1.5x','Spikecount_3.0x','ohmic_input_resistance_vb_ssse_1.5x'
\end{verbatim}


%\item 
Unlike in BluePyOpt \cite{Werner} and Allen Institute optimization workflows \cite{gouwens} rheobase current injection was applied is a soft constraint. However, this work differs from other work in that rheobase current injection was found for every different model paramerisation.
%\item 
Rheobase value as a soft constraint. This means that current injection values are a variable model parameter. The exact value of current used is determined via exploring the model response to current injections of varying amplitude. The exact value, its value is determined by other model parameters.

spike shape measurements and error functions. We used a set of 8 different experimental measurements NeuronUnit.


\begin{figure}
	\includegraphics[]{figures/normal_distribution}
    \caption{Error functions were evaluated with the assistance of a library: \emph{NeuronUnit} 
    were based on finding a normal distribution on electro physiology measurements, 
    and then measuring model outputs and mapping the model behavior onto
     a place on the experimental normal distribution. Scores that where closer to the
      experimental mean where deemed to be low in error.
	Z-scores obtained via NeuronUnit can be thought of as  }
	%\label{figure\arabic{figurecounter}
	
\caption{Caption}
	
\end{figure}
