
\section{Introduction}

\subsubsection{Motivation}


%\section{Introduction}

\subsection{Model Optimization}

Some physical properties of neurons can’t be easily measured in experiments. These unknown properties limit neuron model completeness, and yet it is possible to infer some unknown measurements by  finding a set of parameters, that help a given model "fit" the data, this process is sometimes called "parameter-fitting". For example, a common approach for approximating unknown ion channel densities is to ‘optimize’ the governing equations so that the evaluated equation  match's pre-existing waveform measurements scientists are confident about. The process of optimization involves what is known as an ‘inverse’ problem where we efficiently and sparsely search for the ‘optimal’ value of an parameter that satisfies the system of equations. Often an optimal value corresponds to a global minimum or maximum value of a function.\\*
 
Computational optimization techniques are often specific to a particular type of problem rather than being generalized. However, several notable algorithms have solved a wide range of problems including genetic algorithms and stochastic gradient descent (SGD). The popularity of these two algorithms is due to their robustness. Genetic Algorithms and SGD are able to avoid falsely reporting a local minimum when a more optimal solution is available.\\*


Adaptive Exponential \cite{brette2005adaptive}


\subsection{Multiobjective optimization} Multi objective optimization problems are a subset of optimization problems, where model fitness is evaluated against multiple independent constraints, rather than just one error. It is often possible to reduce multiple constraints into one constraint by summing the outputs of objective functions together, however the price of reducing multiple errors into one error, is that where multiple and diverse models give satisfactory solutions to the provided constraint, reducing error leads to a situation where a constraint that is easier to satisfy, rapidly drags down the error score and dominates the by contributing lower errors to the sum of error scores.\\
\\
However, of SGD and NSGA2, only NSGA2 is a natural choice for tackling multi-objective optimization problems. Default implementations of SGD are not able to utilize the principle of non-domination as an optimization strategy.\\
\\
There is a great diversity of real biological neurons, all of which differ substantially in their electrical behavior. There are a few different classes of general purpose neuronal models, that can reproduce these different types of electrical behaviours, given appropriate parameterizations of the models.\newline
\newline
An existing class of neuron model type, called The Izhikevich model \cite{izhikevich2003simple} (Iz model) was published with parameter sets believed to make the model outputs accurately align with a variety of real biological cell outputs. However since publication much very specific electro physiological recordings have accumulated, that in someways undermine model/experiment agreement. However it is now possible to constrain the Izhikevich model and find new parameterizations that more allow us to more accurately reproduce more recently published experimental data.\newline
\newline

\subsection{Optimization with NeuronUnit}
A software tool "NeuronUnit", is able to perform two functions that help with optimization.\\
\\
The first function, is the automatic scaling of model outputs, to match the statistical distribution of observed measurements.%
%
For example resting membrane potential is measure in $(mV)$, and it may vary according to a distribution with a mean and standard deviation of $(\mu,\sigma)=(-65mV,15mV)$. Additionally an optimizer may sample a model with a membrane potential of $-67mV$. Neuronunit is able to take these parameters together, combine them all into a a Z-score, such a collection of normalized Z-scores provides greater flexibility. In this way NeuronUnit converts a quantitative measure of model/data agreement into a useful error signal. A very natural application of this signal is to guide the process of optimization.\newline 

We have used Neuronunit to guide optimization by taking a flexible model types such as Generalized Linear Integrate and Fire model\cite{teeter2018generalized} or the Izh model and then fitted these models using relevant experimental measurements inside our optimization frame work.

As an example, NSGA2 was used to optimize models in conjunction with data driven tests based on pooled data from NeuroElectro.org \cite{tripathy2014neuroelectro}. A variety of compact and fast single compartment models were used to explore model optimization. Figure 4 demonstrates test error at the beginning of the optimization process for models with randomly sampled parameters and the smaller error following optimization. Figure 5 shows the evolution of the error during the optimization process. \newline
\newline
Optimized neuron models may vary from their neuron counterparts for several reasons. Table 3 shows an example where optimizing the model with respect to the rheobase test comes into conflict with minimizing with respect to input resistance. The solution to the optimization problem consists of two sets of model parameters, which can resolve this conflict differently. Examining the experimental data that these tests were derived from suddenly becomes important. By examining the data, we can see if the rheobase currents and the distributions of input resistance are bi-modal and uniformly distributed. If the data is treated as uni-modal, and the uni-modal mean is used to optimize then the model, then the model is not able to satisfy both constraints simultaneously. In this case, the measurements don’t correspond to neuron data, and the model can’t produce the artificial behavior. When comparing complex data and simple models we find that solutions are better represented using a combination of two optimization solutions.\newline
\newline

Another potential issue to consider when evaluating the scientific merit of a model is that neurons may have different behaviors under different stimulation paradigms. It might be appropriate to compare modeled behavior against measurements specific to each of two or more distinct modes. In this case, when optimizing single cell models, it’s appropriate to accept a solution set, rather than a single solution. For example, the cerebellar Purkinje cell is sensitive to intricately patterned dendrite input current combinations. Depending on a cell’s recent history of synaptic stimulation, a Purkinje cell may toggle between coincidence detection and integration modes (Ratté, Hong, De Schutter, \& Prescott, 2013).
\\
\section{Ecosystem of Modelling Resources}
The NEURON simulator is a software suite that wraps powerful and fast ordinary differential equation solvers based in the C programming language inside a mixed compiled/interpreted environment targeted at research scientists. NEURON is somewhat analogous to older, analog circuit simulators; however, rather than describing complex resistor-capacitor circuits, NEURON instead solves equations for the time varying membrane potential of multi-compartment models.\newline
\newline
These multi-compartmental models are based on cables of varying diameters and lengths that represent the morphology of neurons, where these cables support ionic currents in the membranes. These neuronal models can be coupled together into a network, where the electrical state of one neuron has an impact on the state of coupled neurons through synaptic currents. Specifying the system of differential equations representing these neuronal morphologies, ion channels, and synaptic connections is complicated, but NEURON makes multi-compartment neuron simulation efficient, convenient, and achievable. Models expressed in NEURON code are procedural in nature, and the code consists of low-level implementation details. Procedural descriptions of models are difficult to extend and re-use, leading to a need for a declarative model description language. NeuroML has been tasked with describing these with complex network models.\newline
\newline
Through jNeuroML, the NeuroML project also provides a simple code interface for generating complex simulator code, so that NeuroML models are readily exchanged between different types of simulators. Model interchange permits cross examination of results as a they vary across simulators, and this interchange promotes the movement of models between languages preferred by different modeling communities, reconciling and unifying their models. Because NeuroML is extensible and component based, it incentivizes a "plug\-in" environment for including pre\-existing model components in models in a different large-scale context.

\subsection{Significance}
Beyond experimental error, it is common to observe large variations in measurements of a single electrophysiological entity from neurons of the same classification. As an example, consider that measurements of neuron membrane input resistance may be different when recorded from different samples of the same neuron type. This variation is an essential consideration when evaluating the scientific merit of a computational model of a neuron-type. In this work, we propose to perform a large-scale analysis of model against data agreement and model against model agreement to expose the variation in biophysically realistic neuron models and cortical data. By analyzing the variance in data and among models and linking the variation to specific features and mechanisms, we also will better understand the heterogeneity of experimental measurements from a particular neuron-type. Performing a meta-analysis with a large number of models will provide other insights. We will determine whether there is higher variance in modeled electrical properties versus experimental neurophysiological measurements. We will examine whether an extensive collection of cortical models behaves more similarly to each other than to the data and will answer the question: does the space of all existing single cell models accurately represent the variability in experimental data?
Similarly, variation in the behavior of cortical neuronal networks is not well quantified. Tremendous research effort has been consumed producing several high-quality, experimentally informed cortical network models. Before creating another elaborate network model, we will determine whether these pre-existing approaches lead to networks with significantly different dynamic properties. Also, we will create an infrastructure that allows scientists to quantify the similarities and differences among networks and their dynamics – both biological and in silico.\newline
\newline
Existing data sets are incomplete, consisting of a sparse sampling of cells in the rodent brain. By necessity, models are constrained using these incomplete data sets, leading to compensatory model development that synthesizes missing information. Missing data occurs at multiple levels during network construction including exact neuron to neuron wiring patterns, un-sampled morphologies, unknown synapse activation times, and unknown axon and dendrite synapse locations. Published models should not be regarded as final, but to improve models, it is vital that they are validated against newly-obtained experimental data. The proposed work will facilitate ongoing validation of biophysically realistic models. 
%%
Some electrophysiology data are challenging to integrate into existing models.
These include data collected from animal species that are not widely used in models such as marmoset, guinea pig, and even humans. 
Additionally, neuron-type data may come from a human, but the tissue samples may have been extracted from brain tissue in a pathological condition, such as eplipsey or alzheimers disease.
In practice, open access data is not always useable, as it be derived from multiple species, multiple brain regions and different states of health. We will obtain a better understanding of region-dependent differences and species-dependent differences in order to help researchers map models onto a standardized rodent electrophysiological phenotype space.
%%
\subsection{Distinction From Other Approaches}
The large-scale meta-analysis described here has not been performed previously. For the first time, a large number of cortical neuron and neuronal network models are available in the standardized NeuroML format. Although the Allen Institute for Brain Science modeling project and the Blue Brain project both rigorously analyzed their single cell models, to the best of my knowledge there has not been an overarching meta-analysis across different cell and network model sources.\\
\\
Similarly, numerous modeling efforts have employed data-driven testing in model development workflows, but all these efforts have been based on non-standard ‘in-house’ model types and execution environments. In contrast, this work proposes to expand a pre-existing standardized model testing space, NeuronUnit, that supports model validation and re-use regardless of the model source. To date various NeuronUnit tests of action potential shape, electrical properties, and single cell morphologies exist; yet these tools are not unified. Some tests of network dynamics also currently exist; however, these tests are not integrated into a unified multiscale workflow. Although the work I describe only concerns single cell models in isolation, significantly, a unified workflow for exploring model data agreement would better locate errors in network behavior which are manifest at the network level but are caused by neuron-type models. \newline
\newline