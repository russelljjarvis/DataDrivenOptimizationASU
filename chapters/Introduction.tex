
%\section{Introduction}

\subsubsection{Motivation}
% This section should motivate a few big idea such as:
% - What do we care about computational models of the brain?
Diseases of the brain are very widespread and cost world governments a significant amount of money \cite{who}. Cortical network models have emerged as useful tools to euclidiate brain function in health and disease. Underlying many cortical network models are large numbers of reduced or neuronal models, that either supplament the behavior of multi-compartment conductance based cells, or in some cases completely replace more comprehensive and bottom descriptions of electrically excitable cells.
% https://www.who.int/bulletin/volumes/96/5/17-206599/en/ 
%The 2015 Global Burden of Disease study estimates that about a third of the population worldwide is affected by mental or neurological disorders across their lifespans

% disorders of the brain rank among the leading causes of ill-health and disability and account for 35% of Europe’s total disease burden with a yearly cost of 800 billion euros, of which 60% are related to direct health care and non-medical costs.4,5 
% The burden is growing

In addition to producing actions and goal directed behaviors, mammal brains execute many signal processing steps that are increasingly understood at mechanical and algorithmic levels \cite{marr1976understanding}. Innovation in Artificial Intelligence has been achieved by applying just a few such algorithms observed in the brain. For examples grid and place cell behavior has spontaneously occurred Artificial Neural Networks trained to navigate a landscape \cite{banino2018vector}. Also consider the recent success of "deep-neural networks" and "recurrent neural networks"

Neuroscience findings have been translated into collections instructions that are understood by computers. The abilities of brain inspired algorithms have sometimes been recognized as innovation in Artificial Intelligence.  Computer programmers and engineers want to find out the neural principles that underlie human learning and reasoning, so that these principles can be implemented in an electronic substrate. Improvements in AI directly contribute to improvements in robotics, improving Artificial Intelligence means that larger amounts of work can be done at a lower cost, but it also we may need may need artificial intelligence to create new science and to solve existential problems facing humanity. Computers and robots are also very good at performing repitative tasks with high precision.\\
\\
Our medical understanding of brain disease, and digital application of brain algorithms, would likely improve if existing spiking neural network models were both faster and more accurate. Importantly the work I describe in this document, documents tools that improve the speed and accuracy of some neural models. The tool can make some cell models better at mimicing appropriate experiments. By improving single cell models, this tool will likely improve network models. Network models are made of many thousands of reduced cell models, so there constituent behavior is the result of many interactions between reduced cell models.\\ 
\\

A common type of model of neuroscience model is the animal model, animal models have limitations which can make them scientifically unfavorable. Animal models often involve years of investment, genetic engineering, complex surgery, complicated behavioral learning paradigms. Many things can go wrong while developing an assay, potentially ruining a costly experiment, and diminishing years of work.\\
\\
Animal models are neuroscience models of brain diseases and learning where neural phenomena are investigated by manipulating the conditions in animal brain tissue such that mechanisms thought to underlie the phenomena being investigated are controlled for. 

% Examples include neural recordings from grid cells and place cells in a navigating rodent.
Examples include a rodent model alzheimers disease in humans. The "model" is formed by 
genetically engineering rodents so that an active group expresses beta-amalyoid proteins in the hippocampus, as beta-amalyoid plaques are a feature that would be explained by a comprehensive theory of Alzheimers disease. A different example is when circumstances are contrived so that rodents can self administer cocaine, in order to understand the neural mechanics of drug seeking behavior.\\
\\
Animal models try to express the essential features of brain phenomena in a different brain in a different animal. Animal models bare a very close resemblense to the conditions they attempt to explain. As both experiment, and phenomena occur in neural tissue. In animal models experimental conditions are created by manipulating neural substrate.\\  
\\
In in-silico models systems of differential equations represent biological processes under investigation. The main biological process is current flow.

%and physically blocked synaptic transmission.\\

In the rodent model of Alzheimers disease, it is possible to investigate rodent behavior for Alzheimers symptoms. In the digital model you would produce graphs of neural firing rates, which might reveal altered brain dynamics. Animal models have major draw backs. One such problem is, there are major limits on the density of recording electrode arrays. Another limitation is immune reactivity and glia scars.\\

Computational models are desirable because they are relatively cheap, understandable highly reproducible, and ethically favorable. All variables in digital models can be stored and re-used.
% - Why do we care about computational models at the level of neurons?

% - How are neurons modeled (i.e. math and equations)

\subsection{Approaches to modeling}
% What are 
% - (Briefly) conductance based models
Conductance based models are electrical and chemical representations of semi permeable neural membrane.  The phosholipid membrane of neurons is covered in a variety of pores called channels and each pore can permit the passage of molecular ions at specific rates, and some times those rates can vary in proportion to other electrical properties in membrane surface adjacent to the channel. In conductance based models ion channels are represented as differential equations which explicitly represent the rate at which ion channels conduct ions. Conductance based models differ in their spatial extant. In its most basic form a conductance based model, could can be a single point (in this case the model is so abstract that it dispenses with the notion of space in nervous tissue). In a large variety of conductance based models the 3D form of neural membrane is "modelled", but it is approximated by using elaborate systems of 2D branching patterns. \cite{rall1962electrophysiology} Rall pionered the digitization and discretization of 3D hodgkin huxley models. Rall took numerical methods used for modelling heat and electrical conductance in cables, and he showed that you could generalize this approach to 3D trees made from cables.\\
\\
%often inside the surface of a 3 dimensional neuron. 
Each ion channel is modelled by a set of differential equations that represent how conductance changes with time. In the context of realistic 3D modelling, conductance based model are very slow to solve because they contain so much detailed biophysics. For instance the thickness of neurons can change along their arbors. Voltage dependant ion channels are distributed along the branching neurite in varying density.\\
\\
% - Reduced models
Reduced Models by contrast ignore neuron biophysics. The 3D form of real neurons is replaced by a dimensionless point. The electrical behavior of neurons is represented by a very simple equation that takes input parameters.

\subsubsection{The Leaky Integrate Integrate and Fire Model and GLIF}
Natural Progression of model explanation: LIF, GLIF, Adexp, Qaudratic Integrate and fire (izhi). Now appropriately situated into  the section on reduced models
\subsubsection{The Adaptive Exponential Integrate and Fire (AdExp) model} \cite{brette2005adaptive}, is another type of reduced neural model. The Adexp model is a special instance of leaky integrate and Fire model which most often appears with an exponential spike shape. The models instantaneous spiking rate is calculated via a consulting finite time "windows" into the neurons recent spike history. Adaption is achieved by looking back into the window and counting the number of spikes that occurred in the previous $10ms$, for example. Although this model is potentially fast, often python implementations of it are slow due to code that was written to make running populations of many neurons fast, at the expense of running single neurons. Forinstance Brian2, and a related tool neurodynamics, have code for running Adexp neurons, however brian2 has network c-level cython code that interferes with DEAP python code.\\
\\
The Izhikevich model and the adaptive exponential integrate and fire model, are both related to simple Integrate and Fire model. The Izhikevich model takes a current injection value, and a capacitive current, a membrane potential is created by integrating an equation $\frac{d V_{M}}{dt}$. The Izhikevich model has quadratic multiplicative terms. The GLIF and the AdExp models have exponential multiplicative terms.


%quadratic, exponential 
%and maps them onto outputs like membrane potential. 

%Its as if a reduced model is a black box representation of a neuon, and the reduced model is a transfer function that maps inputs to outputs. Typically reduced models are very fast to solve.\\
%\\
% - Kinds of reduced models and what they can do
%\section{Introduction}
% above.
% 2-5pages
\subsection{Model Optimization}
%
Some physical properties of neurons can’t be easily measured in experiments. These unknown properties limit neuron model completeness. It is possible to access missing information by creating "determined" sets equations, that allow researchers to solve for unknown variables, as the number of known conditions on equations out number unknown variables.\\
\\
It is possible to infer some unknown measurements by finding a set of parameters that help a given model "fit" or agree with measurements from experiments, this process is sometimes called "parameter-fitting". For example, a common approach for approximating unknown ion channel densities is to ‘optimize’ the governing equations so that the evaluated equation  match's pre-existing waveform measurements scientists are confident about. 
\\
The process of optimization involves what is known as an ‘inverse’ problem. A naive approach to solving this inverse optimization problem works on the principle of elimination. A large number of unsatisfactory states must be sampled, to rule out the a current sample is not worse than another sample. Unfortunately solving these equations is sometimes computationally intractable.\\

%for the ‘optimal’ value of an parameter that satisfies the system of equations. An optimal value corresponds to a global minimum or maximum value of a function.\\*
Fortunately a stochastic technique has emerged where we can efficiently and sparsely search

% Describe here what such a function might look like, i.e. what would that function be measuring, generally?

In the context that I perform model optimization models are injected with specific amounts of current, because neurons are electrically excitable, the neuron soma membrane potential deflects from resting membrane potential in response to current injections.

An error function could be the simple difference between an experimentally measured spike amplitude (the peak of a voltage deflection), and the models peak spike amplitude. The units of both measurements will be milli-volts.  

% In this section you should talk more about stochasticity in general. Imagine first a deterministic gradient descent algorithm and why this might not work (local minima).  Then you can get to genetic algorithms and describe the tradeoff between exploration and exploitation.

\subsubsection{Local Minima} Many real optimization problems have error surfaces that not perfectly convex but are instead actually have a surface shape that is rippled, so that as you travel along the error surface in any direction, you may encounter multiple troughs of similarily low error value. These ripples make navigating the error surface non trivial.\\ 
\\
Rastragins function is a function for creating error surfaces to test the robustness of genetic algorithms. The surface described by the function is densely populated by local minima wells, the bottom of each well actually aligns with a globally convex pattern, such that the true minima of rastragrinds function is in the centre.\\
\\
Given enough samples an "elitist" Genetic Algorithm will learn the global minima of the rastragrinds function, although it is very likely to sample wells spanning many depths of the error function on the way down.\\
\\
There are at least two stochastic operations that help to deflect the GA away from the gravity of the local minima.

% My GA picture belongs here

\subsubsection{Genes}
In the context of genetic algorithms, a gene is the complete set of model parameters that are necessary to define a the thing being optimized, in this case a neural model. These parameter sets are collections of floating point numbers.
\subsubsection{cross-over}
The most important type of cross over to understand is binary cross-over. In a population of genes you can create many pairs of genes. You can take just a few of the possible pairs, and represent the each parameter in the gene as a binary number, then at random bit positions in the number, you can swap the swap or exchange the status of the bit in each gene. It's possible that both bits will be the same such that swapping does nothing, however there are many opportunities for this cross over event to occur, and it is not the only possibility to stochastic ally perturb the value of a gene.
\subsubsection{mutation}
Likewise mutation can also be understood in terms of the binary representation of a floating point number. A model parameter can take values in a particular range (0.1$\mu$, 0.5$\mu$). When the parameter value say 0.44 $\mu$ is represented in binary encoding. A bit at a random mutation is simply toggled, the magnitude of mutation in float number depends on the position of the bit that was toggled.
\begin{figure}
\includegraphics[]{rastagrind.jpg}
\end{figure}

%optimization

Computational optimization techniques are often specific to a particular type of problem rather than being generalized. However, several notable algorithms have solved a wide range of problems including genetic algorithms and stochastic gradient descent (SGD). The popularity of these two algorithms is due to their robustness. Genetic Algorithms and SGD are able to avoid falsely reporting a local minimum when a more optimal solution is available.\\*
\\
Stochastic gradient descent, and Genetic Algorithms both utilize concepts like mini-batches, random pertubation (mutation) and combining the concept of average fitness with stochasticity. \\


%to sub-optimal code.\\
%\\

% Move this up to precede a description of specific algorithms.  Multiobjective could apply to gradient descent as well.  You can then refer back to this when you talk specifically about algorithms like NSGA in the "genetic optimization" section.
\subsection{Multiobjective optimization} Multi objective optimization problems are a subset of optimization problems. In a multiobjective optimisation model fitness is evaluated against multiple constraints rather than just one constraint. A constraint is often an implementation of a mathematical function, very often the function seeks to difference a part of a signal that was sampled with a matching part of a known.\\
\\
It is often possible to reduce multiple constraints into one constraint by summing the outputs of objective functions together. There is a price of reducing multiple errors into one error. I will illustrate the problem of multiobjective optimisation. optimization using a sum over multiple error measurements leads to a situation where a single constraint that is easier to satisfy, rapidly drags down the error score and dominates the overall error reading in this manner. In the worst case, the optimizer achieves perfectly low error on one criteria, and high error on a different set of criteria.\\
%by contributing lower errors to the sum of error scores. 
\\
Additionally problems formulated in a multi-objective paradigm are better able to result in diverse solution sets. where multiple and diverse models give satisfactory solutions to the provided constraint, 
\\
However of SGD and NSGA2 only NSGA2 is a natural choice for tackling multi-objective optimization problems. Default implementations of SGD are not able to utilize the principle of non-domination as an optimization strategy.\\
\\
There is a great diversity of real biological neurons, all of which differ substantially in their electrical behavior. There are a few different classes of general purpose neuronal models, that can reproduce these different types of electrical behaviours, given appropriate parameterizations of the models.\newline
\newline
An existing class of neuron model type, called The Izhikevich model \cite{izhikevich2003simple} (Iz model) was published with parameter sets believed to make the model outputs accurately align with a variety of real biological cell outputs. However since publication much very specific electro physiological recordings have accumulated, that in someways undermine model/experiment agreement. However it is now possible to constrain the Izhikevich model and find new parameterizations that more allow us to more accurately reproduce more recently published experimental data.\newline
\newline

% First include a section about NeuronUnit more generally.  NeuronUnit really isn't/wasn't about optimization until your thesis.  
\subsection{Optimization with NeuronUnit}
A software tool "NeuronUnit", is able to perform two functions that help with optimization.\\
\\
The first function, is the automatic scaling of model outputs, to match the statistical distribution of observed measurements.%
%
For example resting membrane potential is measure in $(mV)$, and it may vary according to a distribution with a mean and standard deviation of $(\mu,\sigma)=(-65mV,15mV)$. Additionally an optimizer may sample a model with a membrane potential of $-67mV$. Neuronunit is able to take these parameters together, combine them all into a a Z-score, such a collection of normalized Z-scores. In this way NeuronUnit converts a quantitative measure of model/data agreement into a useful error signal. A very natural application of this signal is to guide the process of optimization.\newline 

We have used Neuronunit to guide optimization by taking a flexible model types such as Generalized Linear Integrate and Fire model\cite{teeter2018generalized} or the Izh model and then fitted these models using relevant experimental measurements inside our optimization frame work.

As an example, NSGA2 was used to optimize models in conjunction with data driven tests based on pooled data from NeuroElectro.org \cite{tripathy2014neuroelectro}. A variety of compact and fast single compartment models were used to explore model optimization. Figure 4 demonstrates test error at the beginning of the optimization process for models with randomly sampled parameters and the smaller error following optimization. Figure 5 shows the evolution of the error during the optimization process. \newline
\newline
Optimized neuron models may vary from their neuron counterparts for several reasons. Table 3 shows an example where optimizing the model with respect to the rheobase test comes into conflict with minimizing with respect to input resistance. The solution to the optimization problem consists of two sets of model parameters, which can resolve this conflict differently. Examining the experimental data that these tests were derived from suddenly becomes important. By examining the data, we can see if the rheobase currents and the distributions of input resistance are bi-modal and uniformly distributed. If the data is treated as uni-modal, and the uni-modal mean is used to optimize then the model, then the model is not able to satisfy both constraints simultaneously. In this case, the measurements don’t correspond to neuron data, and the model can’t produce the artificial behavior. When comparing complex data and simple models we find that solutions are better represented using a combination of two optimization solutions.\\
\\
Another potential issue to consider when evaluating the scientific merit of a model is that neurons may have different behaviors under different stimulation paradigms. It might be appropriate to compare modeled behavior against measurements specific to each of two or more distinct modes. In this case, when optimizing single cell models, it’s appropriate to accept a solution set, rather than a single solution. For example, the cerebellar Purkinje cell is sensitive to intricately patterned dendrite input current combinations. Depending on a cell’s recent history of synaptic stimulation, a Purkinje cell may toggle between coincidence detection and integration modes (Ratté, Hong, De Schutter, \& Prescott, 2013).
\\

% I think you can make this section a bit shorter, and refocus to talk about how one major goal is to build tools that can be integrated into the larger modeling ecosystem.  You could have simply written an optimizer in DEAP using jitted python code, custom for each model type, but then you would not be able to interface with everyone else and no one would be able to make use of your tool.  So talk about the importance of standards and interfacing.  and then introduce these tools as standards and components that you need to be able to work with.  
\section{Ecosystem of Modelling Resources}
The NEURON simulator is a software suite that wraps powerful and fast ordinary differential equation solvers based in the C programming language inside a mixed compiled/interpreted environment targeted at research scientists. NEURON is somewhat analogous to older, analog circuit simulators; however, rather than describing complex resistor-capacitor circuits, NEURON instead solves equations for the time varying membrane potential of multi-compartment models.\\
\\
In contrast to reduced models, multi-compartmental models represent the form of membrane tissue: cables of varying diameters and lengths that represent the morphology of neurons, and these cables support smaller scale representations of ion channels, and ion currents in the membranes. These neuronal models can be coupled together into a network, where the electrical state of one neuron has an impact on the state of coupled neurons through synaptic currents. Specifying the system of differential equations representing these neuronal morphologies, ion channels, and synaptic connections is complicated, but NEURON makes multi-compartment neuron simulation efficient, convenient, and achievable. Models expressed in NEURON code are procedural in nature, and the code consists of low-level implementation details. Procedural descriptions of models are difficult to extend and re-use, leading to a need for a declarative model description language. NeuroML has been tasked with describing these with complex network models.\newline
\newline
%Through jNeuroML, the NeuroML project also provides a simple code interface for generating complex simulator code, so that NeuroML models are readily exchanged between different types of simulators. Model interchange permits cross examination of results as a they vary across simulators, and this interchange promotes the movement of models between languages preferred by different modeling communities, reconciling and unifying their models. Because NeuroML is extensible and component based, it incentivizes a "plug\-in" environment for including pre\-existing model components in models in a different large-scale context.

% illustrate how a Genetic Algorithm (GA) generally works.
\begin{center}
\begin{figure}

    \includegraphics[width=0.7\linewidth]{figures/How_Genetic_Alg_Works.png}
  \caption{Genetic Algorithm Overview}
  \label{fig:GeneticAlgOver}
\end{figure}
  
\end{center}

In \ref{fig:GeneticAlgOver}  Genetic algorithms find satisfactory solutions by incompletely sampling the solution space. Evolution of the algorithm is guided by the combined application of stochastic sampling and selective pressure. Stochasticit contributions to model sampling act as a mild incentive for models to explore regions beyond local minima. Stochasticity is applied in the combined actions of cross-over, and mutation. Model parameterizations are encoded as binary strings called genes. When genes breed, eligible pairs of genes are aligned and at random bit locations, the status of a bit is exchanged.  

\subsection{%Significance}
% I wouldn't call this section "Significance".  In general, think of these sections as small modules that can be reshuffled and rearranged as needed.  You don't want to pin yourself to describing everything that is "Significant" right here.  In particular, you could call this section "Neuronal diversity".

\subsection{Analysis of Model and Experimental Variance}
%
Beyond experimental error, it is common to observe large variations in measurements of a single electrophysiological entity from neurons of the same classification. As an example, consider that measurements of neuron membrane input resistance may be different when recorded from different instances within the same neuron type. Within sample variation is an essential consideration when evaluating the scientific merit of a computational model of a neuron-type. In conjunction to optimization, we propose to perform a large-scale analysis of model against experiment agreement and model against model agreement to expose the variation in biophysically realistic neuron models and cortical data.\\
\\
By analyzing the variance in experiments and  models and linking the variation to specific features and mechanisms, we also will better understand the heterogeneity of experimental measurements from a particular neuron-type. Performing a meta-analysis with a large number of models will provide other insights. We will determine whether there is higher variance in modeled electrical properties versus experimental neurophysiological measurements. We will examine whether an extensive collection of cortical models behaves more similarly to each other than to the data and will answer the question: does the space of all existing single cell models accurately represent the variability in experimental data?
Similarly, variation in the behavior of cortical neuronal networks is not well quantified. Tremendous research effort has been consumed producing several high-quality, experimentally informed cortical network models. Before creating another elaborate network model, we will determine whether these pre-existing approaches lead to networks with significantly different dynamic properties. Also, we will create an infrastructure that allows scientists to quantify the similarities and differences among networks and their dynamics – both biological and in silico.\newline
\newline
% Shouldn't this be a new section, about data integration?
Existing data sets are incomplete, consisting of a sparse sampling of cells in the rodent brain. By necessity, models are constrained using these incomplete data sets, leading to compensatory model development that synthesizes missing information. Missing data occurs at multiple levels during network construction including exact neuron to neuron wiring patterns, un-sampled morphologies, unknown synapse activation times, and unknown axon and dendrite synapse locations. Published models should not be regarded as final, but to improve models, it is vital that they are validated against newly-obtained experimental data. The proposed work will facilitate ongoing validation of biophysically realistic models. 
\\
% And this can go up into the section about diversity
Some electrophysiology data are challenging to integrate into existing models.
These include data collected from animal species that are not widely used in models such as marmoset, guinea pig, and even humans. 
Additionally, neuron-type data may come from a human, but the tissue samples may have been extracted from brain tissue in a pathological condition, such as eplipsey or alzheimers disease.
\\
In practice, open access data is not always useable, as it be derived from multiple species, multiple brain regions and different states of health. We will obtain a better understanding of region-dependent differences and species-dependent differences in order to help researchers map models onto a standardized rodent electrophysiological phenotype space.\\
\\
\subsection{Distinction of Optimization Approach From Other Approaches}
% This can be a fusion of your sections about multiobjective optimization, unit testing, and data integration (or whatever set of background items you think is fundamental to understanding the novelty of the work you have done).
Please elaborate on these limitations in a section called 



\subsection{Distinction of Large Scale Ananlysis From Other Approaches}
The large-scale meta-analysis described here has not been performed previously. For the first time, a large number of cortical neuron and neuronal network models are available in the standardized NeuroML format. Although the Allen Institute for Brain Science modeling project and the Blue Brain project both rigorously analyzed their single cell models, to the best of my knowledge there has not been an overarching meta-analysis across different cell and network model sources.\\
\\
Similarly, numerous modeling efforts have employed data-driven testing in model development workflows, but all these efforts have been based on non-standard ‘in-house’ model types and execution environments. In contrast, this work proposes to expand a pre-existing standardized model testing space, NeuronUnit, that supports model validation and re-use regardless of the model source. To date various NeuronUnit tests of action potential shape, electrical properties, and single cell morphologies exist; yet these tools are not unified. Some tests of network dynamics also currently exist; however, these tests are not integrated into a unified multiscale workflow. Although the work I describe only concerns single cell models in isolation, significantly, a unified workflow for exploring model data agreement would better locate errors in network behavior which are manifest at the network level but are caused by neuron-type models. \newline
\newline

\chapter{Introduction}

\subsection*{Motivation for Reduced Models of electrically excitable cortical cells}

Multi-compartment conductance based models consist of fine grained biophysical models of neuron membrane including ion channel conductance, and membrane shape and form. These biophysical models take a significant time and computer resources to evaluate. A different class of neuronal model, known as a "Reduced model" is comparatively fast to solve. Speed, empirical validity, are among the two biggest constraints affecting the design of cortical brain models. 
% Further more reproducibility and accessibility are increasingly regarded as important model design constraints too., and they will also be addressed in this work. 

Brief duration model evaluations are critical to brain network simulation speed, as each of many models must be evaluated synchronously according to a global network clock, and there is a high dependency between the states of different models. Network simulation is only as fast as the slowest model, because (if you are waiting for one neuron to evaluate, the whole network will need to wait, as its state may depend on that neuron). Since these constraints are in direct conflict with each other, many scientists are interested in the finding the most optimal resolution to this speed/accuracy trade-off.\\ 
\\
The work herein is aligned with one particular approach to the trade-off. To make data specific versions of simpler models, such that simpler models better mimic experimentally observed electrical measurements. To understand this approach to improving model speed and accuracy, it is important to be familiar with the concept of reduced neural models. Some types of neural models are mimic neurons across many different forms and scales.\\
\\
This approach contrasts with other existing approaches to neuronal model optimization. Shape, and ephysiology based approach in contrast to a spike timing approach. 
\\
Examples of reduced models are:
Point conductance based model, Adaptive Exponential Integrate and Fire model, and Izhikevich model, and in fact, these are the only reduced models considered in the scope of this document.\\
\\
These models are easier faster to evaluate, but they are also easier to understand.

% I will discuss each different model in its own section.
\subsubsection{Importance of Simulation Duration}
speed of model evaluation is a critical factor for learning things from brain simulations. Digital modeling of physical properties of cells is occurring at the mesoscopic scale, however adding microscopic features to simulations significantly increases simulation time. Even when using High Performance Computing developing simulations involves a debugging errors in different modalities, sometimes the computer program needs correcting, but at other times the scientific expectations about neural dynamics might be wrong. The maths might be wrong. To correct for any mistakes, feedback in the form of observed model behavior, is essential, for the development of the model itself, and short simulation durations are critical to this.\\
\\
The are several valid instances when the complete three dimensional form of a neuron is an integral part of a brain simulation, such as in the Blue Brain somato-sensory cortex model \cite{markram2006blue} and the Allen Institute $V1$ model \cite{billeh2020systematic}, These simulations are improved by encasing a "core" of biophysically accurate models inside a "shell" of simple fast and reduced Izhi, GLIF, or AdExp models.\\ 
\\
Encasing a core of complex models inside a shell of simplified models mitigates an observable "edge effect" problem. The problem is that simulations concern sub divisions of brain tissue, subdivisions by nature exclude externally sourced synaptic inputs. These synaptic inputs are connections that are severed by the process of making a subdivision. All published highly detailed simulations to date, have necessitated the simulation of severed volumes of tissue, and this creates another problem to manage.\\
%There is a core of models of realistic models who are missing a substantial number of "extrinsic" synaptic inputs. 
Almost all cortical neurons experience "tonic" synaptic input and these tonic inputs originate from neurons from a different part of the brain. One strategy for handling inputs that are external to the region of interest, is to simply model spike trains for each input synapse. Modern programming languages have tools that can make matching the synthesis of statistically similar spike trains convenient. One big problem with this approach is it assumes that post synaptic neurons are mainly influenced by the firing rate of inputs, if they pre-synaptic neurons are actually conveying important code words via exact interspike intervals, a statistical approach to modelling spike trains would not do.\\

%An alternative approach is to reduce the complexity of neural models as one exceeds the boundary of the region of interest.

%than generating only psuedo random timed inputs to synapses, 
In the case of the Allen Institute Model, if the region of interest V1 is a "core" of realistic neurons. That is a kernel of realistic neurons encased by a shell of less realistic neurons. Inputs to V1 also come from the outer encasement of neurons. It is therefore of interest if these external GLIF models can or should be substituted with optimized Izhikivich and AdExp models, in case substiting GLIF for Adexp results in an overall more realistic network simulation. In that case, even the external shell of simulation could experience a marginal improvement in accuracy. In network models there are benefits of reduced models over the use of a point process or a spike train surrogate.\\

% benefits: interpretability, transparent function, has current so contributes to LFP

One of these benefits is that the firing of reduced neural models can be made to be causal, such that its spike times are not just what statistically matches missing models. Furthermore reduced models can still participate in networks, reduced models can become disconnected or participate in an dynamic assembley. Realistic levels of plasticity of the modelled network is more possible with included reduced models, than statistical surrogates of those models.

Furthermore Izhikivitch and AdExp models are commonly utilized in neuromorphic spiking neural networks in artificial intelligence and bio medical modelling contexts.
%archictecture.

%\subitem 
optimization is an interaction between models and constraints which guides a fitting process. Not all neural models are equally flexible.  
%\item  
Both the choice of constraining equations, and the choice of neural models must be favorable in order for models to be fitted to data.
%\item 
%\subitem 
if the combination of models and constraints is bad, then then a tractible error surface will not result.  

%\subsubitem 
Unfortunately, it is not always possible to know without trying which combinations of \subitem[A]: models, and \subitem[B], constraints will lead a tractable error surface, however a nicely smooth manifold surface with only minor oscillations is preferable

%that a Genetic Algorithms can use to find a global solution to. As an example consider  

I describe some code implementation experiments were the model/constraint combination lead to DEAP genetic algorithms matching model parameters to constraints and model/constraint selections that lead to optimizer performing only marginally better than a random search of parameter space\\
\\
%\item 
If the number of dimensions that are searched exceeds the degree and the effectiveness of constraints, then model optimization is only slightly better than random sampling of solution space.

\subsection*{Successful Optimization} 
Successful Optimization is more likely to occur if a favorable combination of models and objective functions occured. 
%\item 
The exact way, that model constraint combinations interact cannot always be known in advance, and yet the interaction can be assessed by attempting to optimize on the pair. Experimenting of model test combinations is what was done in this body of work.

It is important to note that, the measurements that are built off something that is relative to another changeable quality in a model are prone to resonance and oscillation. For example measuring \frac{1}{10} \times \frac{dv}{dt}

Efficient model examples: (generalized leaky integrate and fire model) GLIF, Izhikitch. Adaptive Exponential Integrate and fire model, single compartment conductance based model. 

%\item 
There are at least three different and com-possible approaches to more realistic brain simulations. In one approach you would make biophysically accurate models faster. In a different approach you could make reduced models more accurate. To make reduced models more accurate, you would find parameterizations of the models that let the models act as better mimics of experiments.
%\item 
%Herein we investigate how well Faster models can match experimental recording waveform shapes.
%\item 
%The reason why we want to investigate the match:
%\item 
%Large scale simulations cant evaluate on a timescale that is meaningful, unless a large ratio of modelled cells are "reduced models"
%\item 
%Reduced Models already enjoy wide spread usage. We want to investigate if reduced models can be made to be more realistic, by checking if they can mimic data better.
%\item 
%If reduced models can't be made more realistic (herein we show only marginal improvements), we need to show the limitations of reduced cells, with regards to a particular set of tests.
%\item
%We need to document the approach used, and how the approach contrasts with spike time approaches to model fitting.
%\end{itemize}

\subsection{Optimizing multispiking Behavior}

When using Allen experimental data to create multi-spiking tests, there was the potential to more tightly constrain model behavior. In order to evaluate the goodness of fit models the $Chi^{2}$ statistic could no longer be used, instead one could compute the 'variance explained ratio' between the allen institute experimental sweep and the optimized models sweep to a comparable current injection.

and there is also the potential to check the variance explained ratio of tests.
In such a multispiking paradigm the highest possible variance explained ratio of approx $1$. 

Caption in this figure there is visibly almost perfect  agreement between simulated and experiments and the optimized models, in the passive experimental conditions, and a close match for the spiking model behavior. There was a standard suite of tests if only spike-half-width, not spike-base-width. The Izhi models width as thus free to vary at the base, take that into account when eye balling the two graphs and you can see why almost binary match. EFEL does achieve spike width binary matching because it uses both half-width, and base-width. If you look at the last cells you can see I take a correlation matrix of the optimizers errors over its history. The idea is if it's normalized then I can sum the whole matrix and get a single scaler number to show how de-correlated both error sets are over the GA evolution. 

It was found that the elephant/neuroelectro-suite of NU tests don't fully constrain the spike width at the base of neuron action potential waveform. Since the base of the waveform was unconstrained, it was free to vary, half-spike-width is constrained, it is more appropriate to talk about variance explained of the spike snippet. $variance explained>0.95$ is a useful heurism. Allowing some margin acknowledges that we shouldn't assume we have represented all waveform features that can vary. If you want variance explained ==1   you could optimize using a variance explained cost function, but we don't want to do that.

%\begin{itemize}
%\item 
%First with $2$ constraints, then with $>100$ constraints via the feature extraction suite EFEL

%\item 
%This is a two staged approach. 
%\item 
%It uses three step protocols.

%\item 

%\item 
%In this two stage algorithm: I optimize, using constraints: spike count at injection strength $1.5 \times rheobase $, and $3.0 \times rheobase $
%I do a quick check of spike counts at 1.5 and 3.0 rheobase (2 constriants only).
%\item 
%In this scenario Rheobase is used only as a soft constraint. Firstly we optimize the model to spike count data we very rapidly narrow down the solution space using only two errors. \\
%\\
%In a second code experiment I use the standard NU suite, for a lot of generations. In the figure below
%%
% Figure
%%
 
%\end{itemize}


% Add a section about model databases and the opportunities they will make available to you.  Describe how we don't really know whether the models in these databases match the experimental data they claim to recapitulate.  Describe some of the specific challenges in using "in house" data.