%\chapter*{ABSTRACT}
%\chapterstyle{ASU}
%\pagestyle{ASU}
%\frontmatter

%\begin{center}
%    ABSTRACT
%\end{center}
%\begin{abstract}
%\setlength{\parindent}{.5in} 
%\indent 
\hspace*{\parindent} 
Neuron models that behave like their biological counterparts are essential for computational neuroscience.
Reduced neuron models, which abstract away biological mechanisms in the interest of speed and interpretability, have received much attention due to their utility in large scale simulations of the brain, but little care has been taken to ensure that these models exhibit behaviors that closely resemble real neurons.
In order to improve the verisimilitude of these reduced neuron models, I developed an optimizer that uses genetic algorithms to align model behaviors with those observed in experiments.
I verified that this optimizer was able to recover model parameters given only observed physiological data; however, I also found that reduced models nonetheless had limited ability to reproduce all observed behaviors, and that this varied by cell type and desired behavior.
These challenges can partly be surmounted by carefully designing the set of physiological features that guide the optimization. In summary, we found evidence that reduced neuron model optimization had the potential to produce reduced neuron models for only a limited range of neuron types.
%\end{abstract}

%Abstract
%Type ABSTRACT in all capital letters; not bolded, center aligned

%One double-spaced between “ABSTRACT” and text

%Double space all text following; no bold; no italics or underline (used only for species, genre, book titles, musical compositions or foreign words and phrases)

%All acronyms or abbreviations must be written out fully at first use with acronym/abbreviation in parenthesis

%Same size and font as text

%Must start on page i

% Do not exceed 350 words in length


%% Jimmy flagged this sentence.

%%

%\setcounter{page}{i}

\pagenumbering{roman}






%For the three types of reduced neuronal model we were able to obtain models that produced mostly biologically plausible measurements, however, approximately one in every eight fitness constraints needed to be pushed towards the outer limit of plausibility, in-order to produce "all-round" empirically valid cell models. Existing model fitting approaches have succeeded in fitting reduced neuronal models to the onset frequency and exact times of neuron firing, in this work we sometimes fitted to spike timing, but we prioritized other electrical measurements in cells relating to membrane impedance, action potential shape, and current vs firing frequency relationships.\\
%In order to investigate how well reduced models can reproduce experimental data we executed numerical simulations, and coupled the simulations with a genetic algorithm workflow. We used data sourced from neuroelectro.org, and 
%Among alternative models such as Point-process spike chain surrogates, and  multi compartment conductance models, only reduced neuronal models are faster to execute, and easier to interpret. 

%Three subtypes of reduced neuronal models, represent a midway point, in terms of their ability to faithfully reproduce experimental findings.\\
%combined it with a selection of differential equation solvers. custom written differential equation solvers for the Adaptive Exponential, Izhikevich, and simple conductance based models.\\
% Reduced models are a crucial component of many large scale brain simulations.
%in these fast to execute models. 
%We wanted to understand the limits of fitting reduced neural models to experimental electrophysiology measurements, and to use this knowledge to acquire the best possible fits between reduced neuron models and experimental electrical measurements.


%. in this class of models. \\
%\\

%We computed normalized Z-scores.

% * The methods were using a genetic algorithms to solve multiobjective optimization problem. Forward euler solvers


% Additionally reduced neuronal models simulate current flow, and thus are compatible with local field potential models. Although they are not spatially 


% Fulfil and understand the limits of reduced model fitting and agreeing to experiments. 

%as best as possible, fit reduced neural models to experimental measurements, and then to understand and describe limitations in models, that stop impede fitting and cause disagreement between models and experimental measurements.
%of the agreement between reduced neural models and experimental data reproduce experimentally observed neural measurements.
%Since This also involved exploring limitations in some types of neural data.
%* Euclidiating the limits of reduced models and the limits of data, that impede model fitting.
%* Euclidiating model measurement challenges (Rheobase as a soft constraint, threshold computations).
%to data. I
%METHODS 

% It is not novel to embed numerical model simulation inside genetic algorithms, but it is novel recompute rheobase at every sampled model parameter set, and thus to promote a diverse array of model types to the possibility of single spike waveform shape checking.


%Your research problem and objectives
%    Your methods
%    Your key results or arguments: python, numba, dask, dashboard, multiobjective optimisation, open source, reproducibility.
%    Your conclusion