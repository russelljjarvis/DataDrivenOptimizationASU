\section{Generalizing Behavior Across Neuron States}
Some neurons may have different behaviors under different stimulation paradigms. 
For example, the cerebellar Purkinje cell is sensitive to intricately patterned dendritic input current combinations.
Depending on that cell’s recent history of synaptic stimulation, it may toggle between coincidence detection and integration modes \cite{ratte2013impact}.
If the experimental data being used for optimization only describes a subset of these behaviors, then the optimized model may not generalize to the neuron's other behaviors.
This may not even require the neuron to have state-dependence; the neuron may simply have two very different behavioral regimes, where one regime is difficult to predict even from a large number of observations of behavior in the other regime.

%There is a core of models of realistic models who are missing a substantial number of "extrinsic" synaptic inputs. Many synaptic inputs are severed by the process of making a subdivision.

%archictecture.

%\subitem 
%optimization is an interaction between models and constraints which guides a fitting process. Not all neural models are equally flexible.  
%\item  
%Both the choice of constraining equations, and the choice of neural models must be favorable in order for models to be fitted to data.
%\item 
%\subitem 
%if the combination of models and constraints is bad, then then a tractible error surface will not result.  

%\subsubitem 
%Unfortunately, it is not always possible to know without trying which combinations of A: models, and B, constraints will lead a tractable error surface, however a nicely smooth manifold surface with only minor oscillations is preferable

%that a Genetic Algorithms can use to find a global solution to. As an example consider  

Making a model that does generalize may thus require a number of stimuli from each behavioral regime.
I included only a handful of subthreshold and suprathreshold stimuli (which was more than some previous efforts that only included suprathreshold stimuli), and through NeuroML-DB I also have access to many more precomputed stimuli.
I explored portions of the FI curve, but not the entire curve.
It remains possible that a fully general model may require still more stimuli, such as sinusoidal stimuli of frozen noise currents.
was as much I did herea A fitted model may not generalize to current stimulus at higher or lower current strengths, Unless the model is fitted with constraints informed by those different experiments. As discussed above, the true FI curve is described by a gradient and a bias. When one fits to both the slope and the bias of the FI curve, it means that the model will at least be able to recapitulate the right number of spikes for  a given current strength. %Although this does not necessarily mean that the shapes of the each individual spike will agree between model and experiment. 
% important for network modellers as I discuss elsewhere.

%Preserving 
%
%that required current injection increases causes proportionate firing rate increase, 
% suggesting No, it cannot reduced neuronal models are only okay at fitting one sweep at a time.

%The utility of the Rheobase approach is to demonstrate a work flow. Its possible that some researchers are interested in modelling single spiking behavior efficiently, however, in the context of network simulations, one may be better of fitting to a multispiking sweep.

Nonetheless, in the cell types I studied here, even optimizing against simple subthreshold electrophysiological features alone (Input resistance, Membrane Time Constant, Resting Potential, and Capacitance), in conjunction with a single suprathreshold feature (Rheobase) was often sufficient to reproduce the remainder of the observed suprathreshold behaviors.

%Which fits generalize better? Which tests have more predictive power?

%I fitted an adaptive exponential model, and an Izhikevich model to experimentally obtained values for input resistance, time constant, resting membrane potential, and rheobase.

%Surprisingly fitting to essential electrical properties of neuron membranes was distinctly different from fitting the cells directly to the FIcurve, as I will show below. 

%Specifically in in-vitro cell  cell with Allen specimen id $482493761$ injecting a current of $110*pA$ evokes a spike count of $3$, by contrast in an Izhikevich model fitted against criteria listed above, the model produces $33$ spikes, and not $3$ spikes, in otherwords, essential electrical properties did not predict firing rate very well in the model.

%In the in vitro cell with Allen specimen id 471819401, $290 pA$ causes $20$ spikes,
%in the optimized Adaptive Exponential model $290pA$ caused only $5$ spikes. 

%Motivated by these unexpected discrepancies, I produced a smaller sub group of tests consisting of rheobase and FISlope only, as the discrepancies show that cell capacitance, input resistance, and time constant are generally conflicted with FI curve proportions in in-silico cell models. I discuss a group of canonical tests. These are tests that employ cross-validation, to reveal tests that generalize versus tests that promote ``over-fitting". The first members of a generalizable and canonical set of tests are likely to be the FIcurve and Rheobase tests.

 %specified by:
 %\begin{minted}{python}
 %from quantities import pA, ms
 %current = {'amplitude':110*pA,'duration':1100*ms,'delay':100*ms}
% \end{minted}
 
%One can imagine that in cells that have been optimized to and FIslope rheobase only, that current spike count relationship might be more accurate and faithful to spike count aspects of experiments.
%and in the Izhikevich model $59$

%specified whih:
%\begin{lstlisting}[language=python]

%\begin{minted}{python}
% from quantities import pA, ms
%current = {'amplitude': 290*pA,
%           'duration': 1100*ms,
%           'delay': 100*ms}
%\end{minted}
%\end{lstlisting}

%

%Another way of writing this is that fitting a model to electrical properties of cells, does not
%guarantee that the cell is fitted to above threshold behavior, at current injection virtual experiments with larger amounts of injected current.
%A model fitted to one "sweep" is a model of only that sweep. 

%If you apply the sweep fitted model to a different current injection value, you will not get the expected waveform shape fitted.
%because even though one may have optimized to the shape of a single spike, and the membrane time constant, and input resistance should have a bearing
% on spike width and spike height respectively. Since spike width is dependant on membrane time constant, and input resistance, you 
% would expect that spike frequency and therefore firing rate, should be proportionately related to 
% to the membrane spike width, and therefore membrane capacitance. However, in reduced neuronal models, this was found
% to be not strictly the case.

% The Rheobase value is the biggest determinant of how current injection values map onto spike counts.


% Arguably then only the rheobase current injection value is important for fitting to Reduced Neuronal Models in order to make networks of Reduced Neuronal cells behave more realistically.
% It is unclear, however, what the relationship is between synaptic current and electrode current. Synaptic current is distributed, but electrode current is focused on a single point.
% If both are distal to the axon hillock, both synaptic current, and injected currents may behave in a similar way.


%These model parameters do not encode the spike frequency.
%you didn't optimize to the spike count at different injection values.

%Therefore in order to make realistic network models that make use of reduced models, 
%one should have a firing rate, and some sweep data, of cellular behavior with the same spike frequency.
%one can then fit simple multi-spiking waveform measurements to the models.

%Surprisingly some types of spiking information, like spike rate, and multispiking height, spike adaption.
%Are easier for reduced models to fit. I show this in some simulated data, virtual experiments.

% With spike counts, and spike frequencies, 

