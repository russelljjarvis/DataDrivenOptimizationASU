

\subsubsection{Persistently Incompatibility Tests: Trends and Patterns}



Time Constant measurements were consistantly harder to fit, and where often incompatible, in almost or all model types (including conductance based models, reduced models, and including all data types (NeuroElectro data, and Allen cell-types single experiment data). When the membrane time constant was mismatched, this didn't seem to be of much consequence to other membrane timing measurements in the cell, for instance membrane capacitance, spike widths and spike-thresholds could still be accurate. Frequently mismatched membrane time constants suggests either one of two possibilities: Either our method for measuring membrane time constant is wrong, and is off by a factor of 10, or, model builders write equations that are founded on electrical properties that are preference AP features and electrical properties slightly that don't involve the membrane time constant.

Interestingly there was better compatibility between firing rate versus current (FISlope) and the remainder of the electrical observations (where the remaining measurements tended to agree well with each other), than between rheobase and the same measurements. Compatibility was also experienced between the FISlope and Rheobase tests, so two natural clusters of tests that emerged are: FI curve properties, versus the other electrical properties.

As discussed in the methods \ref{sec:parallel-rheobase} Rheobase is defined as the minimum current injection to evoke exactly one spike, therefore, rheobase can sensibly be drawn onto the FISlope for an experimental cell.. The FI line is not fully defined by the slope, as the slope defines the lines gradient but not its bias, however, because the rheobase value falls on the FI-curve, it intersects with the line which has both the bias and the slope of the FI curve. 

%Taken together,
In the situation where, FI-slope is matched, other electrical measurements match, but rheobase is mismatched; when electrical properties agree with the FI-slope but not the rheobase, this suggests, that the overall linear relationship is correct but the exact quantity of current that biases the FIslope is wrong in models. One reason for this could be due to the difference between modelled resistance, and material resistance in neurons. Another reason, might be because, in the virtual experiments where I measured rheobase, I only used a single value of integration time that was a $1$ second duration,  square current pulse, but experiments could have used either shorter times (stronger currents needed to evoke single spike), or longer integration times (less current needed to evoke a spike).


\subsubsection{A Fitted model may not generalize to injections at higher or lower current stimulus, Unless the model is fitted with constraints informed by those different experiments}
As discussed above, the true FI curve is described by a gradient and a bias. When one fits to the slope and the gradient of the FI curve, it means that the model will at least be able to recapitulate the right number of spikes for a given current strength. 
% important for network modellers as I discuss elsewhere.

%Preserving 
%
%that required current injection increases causes proportionate firing rate increase, 
% suggesting No, it cannot reduced neuronal models are only okay at fitting one sweep at a time.

The utility of the Rheobase approach is to demonstrate a work flow. Its possible that some researchers are interested in modelling single spiking behavior efficiently, however, in the context of network simulations, one may be better of fitting to a multispiking sweep.

In reduced neuronal It was found that the electrical properties of cells:
 (Rheobase, input resistance, membrane time constant,Resting Potential, Capicitance)

 did not encode the above threshold spiking behavior of neurons, when neurons were undergoing a multispiking stimulus regime.

 forinstance in Allen cell $482493761$ injecting a current of 
 specified by $ current = {'amplitude':110*pq.pA,'duration':1100*pq.ms,'delay':100*pq.ms} $
 should elicit a spike count of $3$, in an optimized Izhikevich model that has good agreement with all values except for Rheobase these values it produces $33$ spikes.

One can imagine that in cells that have been optimized to rheobase only, that current spike count relationship might be more  accurate and faithful to the experiment.

In cell 471819401 , $290 pA$ causes $20$ spikes,
in the optimized Adaptive Exponential model $290$
specified by $ current = {'amplitude':290*pq.pA,'duration':1100*pq.ms,'delay':100*pq.ms} $
caused only $5$ spikes. 

and in the Izhikevich model $59$

\begin{verbatim}
RheobaseTest	TimeConstantTest	RestingPotentialTest	InputResistanceTest
observations	70.0 pA	24.4 ms	-71.6 mV	132.0 megaohm
predictions	234.67 pA	31.31 ms	-72.08 mV	130.26 megaohm
Z-Scores	3.98	0.09	0	0.01
\end{verbatim}

Another way of writing this is that fitting a model to electrical properties of cells, does not
guarantee that the cell is fitted to suprathreshold behavior, at current injection virtual experiments with larger amounts of injected current.
A model fitted to one "sweep" is a model of only that sweep. 

If you apply the sweep fitted model to a different current injection value, you will not get the expected waveform shape fitted.
because even though one may have optimized to the shape of a single spike, and the membrane time constant, and input resistance should have a bearing
 on spike width and spike height respectively. Since spike width is dependant on membrane time constant, and input resistance, you 
 would expect that spike frequency and therefore firing rate, should be proportionately related to 
 to the membrane spike width, and therefore membrane capacitance. However, in reduced neuronal models, this was found
 to be not strictly the case.

 The Rheobase value is the biggest determinant of how current injection values map onto spike counts.


 Arguably then only the rheobase current injection value is important for fitting to Reduced Neuronal Models in order to make networks of Reduced Neuronal cells behave more realistically.
 It is unclear, however, what the relationship is between synaptic current and electrode current. Synaptic current is distributed, but electrode current is focused on a single point.
 If both are distal to the axon hillock, both synaptic current, and injected currents may behave in a similar way.


 These model parameters do not encode the spike frequency.
you didn't optimize to the spike count at different injection values.

Therefore in order to make realistic network models that make use of reduced models, 
one should have a firing rate, and some sweep data, of cellular behavior with the same spike frequency.
one can then fit simple multi-spiking waveform measurements to the models.

Surprisingly some types of spiking information, like spike rate, and multispiking height, spike adaption.
Are easier for reduced models to fit. I show this in some simulated data, virtual experiments.

% With spike counts, and spike frequencies, 

\subsubsection{Future Work}
Because of the benefits of the NeuroML model specification, data fitted models should be downloadable in a NeuroML format.