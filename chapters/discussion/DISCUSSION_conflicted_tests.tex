
\subsection{Neuroelectro Measurements that were not Amenable to the Optimization Framework}
\begin{itemize}
\item ThresholdTest
\item SpikeHalfWidth
\item Spike Amplitude
\end{itemize}
As discussed previously this is because of an accumulating rheobase error which caused threshold measurements to differs between cells. This may be more of a problem in certain regions of model parameter space, but the problem was general, it occurred in multiple models. As discussed in \ref{section:pitfalls}, error residues from rheobase calculations can be propagated into these tests, where the error amplifies.
%Aim 1A, write something about tests overall.
%Overall the some 
Tests of static electrical properties amenable to optimization:
\begin{itemize}
\item FISlopeTests
\item Capacitance
\item Input Resistance
\item Time Constant 
\end{itemize}

The following two are amenable to optimization, however, they are a source of potentially growable noise and so they should be used with caution when used as a reference or basis for secondary measurements.
\begin{itemize}
\item Rheobase 
\item Spike Count measurements \end{itemize}


%, , , , test worked but was conflicted. The tests that did not work. This is somewhere else.

Tests that worked within optimization:
Via \emph{Elephant} toolchain: FITests, Rheobase, Capacitance, Input Resistance, Time Constant, Resting Membrane Potential.
Via. 

When optimizing in the supra threshold regime Druckmann used:
(1) spike rate; (2) an accommodation index; (3) latency to first spike;(4) average AP overshoot; (5)average depth of after hyperpolarization (AHP); 
(6) average AP width similar to Druckman, when optimizing in the supra threshold regime.
When optimizing with reduced models, I found that the those 6 measurements were not enough to tightly constrain a fit, and additional constraints were helpful. In this work a minimum of 12 constraints were typically used:
\emph{EFEL}
tool chain:
\begin{enumerate}
\item AHP-depth
\item all-ISI-values
\item Spikecount %$ (similar to rate)
\item adaptation-index
\item mean-AP-amplitude
\item min-voltage-between-spikes
\item minimum-voltage
\item peak-voltage
\item spike-half-width
\item time-to-first-spike
\item time-to-last-spike
\item voltage-base
\end{enumerate}
