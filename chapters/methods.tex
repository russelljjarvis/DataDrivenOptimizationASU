\chapter{Methods}
In addition to the deployment of existing methods to achieve my research goals, this dissertation contains a number of innovations which are best described as methodological.  I include some of these innovations here, especially those of an extremely technical nature.  Some other methodological innovations, especially those of more interest to the computational neuroscience community, are reported later in the Results chapters.

\section{Approach to optimization using NeuronUnit}
Model optimization follows the following basic approach:
\begin{enumerate}
	\item Identify a model class whose parameters are to be optimized, e.g. the Izhikevich model.
	\item Identify a neuron whose experimental data will be used to guide optimization.
	\item Identify a suite of tests that can use that experimental data to guide the optimizer.
	\item Execute optimization of that model class against that suite of tests to return an optimized model.
\end{enumerate}
Within these steps are also a number of smaller decisions, including where the experimental data will be obtained and what kind of simulator will be used to run the model.
Using NeuronUnit, the steps above take the follow pseudo-code form:
\begin{lstlisting}[language=python]
# Import code from NeuronUnit
from sciunit import TestSuite
from neuronunit.models import MyModelClass
from neuronunit.tests import MyTestClass1, MyTestClass2, MyTestClass3
from neuronunit.data import get_data_from_database_x

# Get data about my  neuron
neuron_type = "Russell's neuron" # Replaced with a real neuron type in production
neural_data = get_data_from_database_x(neuron_type)
test1 = MyTestClass1
test_suite = TestSuite([test1, test2, test3])
\end{lstlisting}

%\section{Approach to optimization using NeuronUnit}
Model optimization follows the following basic approach:
\begin{enumerate}
	\item Identify a model class whose parameters are to be optimized, e.g. the Izhikevich model.
	\item Identify a neuron whose experimental data will be used to guide optimization.
	\item Identify a suite of tests that can use that experimental data to guide the optimizer.
	\item Execute optimization of that model class against that suite of tests to return an optimized model.
\end{enumerate}
Within these steps are also a number of smaller decisions, including where the experimental data will be obtained and what kind of simulator will be used to run the model.

Using NeuronUnit, the steps above take the follow pseudo-code form:

\begin{lstlisting}[language=python]
# Import code from NeuronUnit
from sciunit import TestSuite
from neuronunit.models import MyModelClass
from neuronunit.tests import MyTestClass1, MyTestClass2, MyTestClass3
from neuronunit.data import get_data_from_database_x

# Get data about my neuron
neuron_type = "Layer 5 pyramidal Neuron"
neural_data = get_data_from_database_x(neuron_type)
test1 = MyTestClass1(observation=neural_data)
test_suite = TestSuite([test1, test2, test3])
results = test_suite.optimize(model)
\end{lstlisting}

There are two different paradigms for evaluating models in NeuronUnit. Under the first paradigm the model is supported by NeuronUnit, and dynamic simulations of the model are present on the host computer and can be efficiently re-run.

Under the second paradigm the NeuronUnit version of the model only consists of model inputs and outputs that are streamed as needed from a different environment. This may be the case due to the complexity of the model, where re-implementing the model would not improve clarity. Or the full model may exist on a remote machine with outputs accessed via an API. 
%Another way of saying this is that if a "model" is simply a digitized set of waveforms, converting it to a model is as simple as labeling things such that NeuronUnit and the scientist are both consistent about what is a model and what isn't. 
The optimization procedures I describe in this work involve both paradigms, where optimizing within the second paradigm required dedicated code that is not supported within the previously described frameworks.

\section{Model Class Implementations}
Because optimization may involve an extremely large number of independent simulations of the same class of model, each varying only in model parameter values, it is critical that both the overhead for model instantiation and the duration of simulation be as low as possible. We built faster Python implementations of two neuronal models -- the Izhikevich model and the adaptive exponential integrate and fire model. My implementations of the Izhikevich model was based on an existing MATLAB forward Euler implementation, while my implementation of the adaptive exponential integrate and fire model came from vectorized Python code, which was relatively slow.

I was able to use a tool to make both of these models perform faster than the standard versions for the brian2 and NEURON simulators. This tool, \cite{numba}, enables Just In Time compilation (JIT).
 %Without over stating things, 
 When regarding this thesis effort and its contribution to knowledge, some of the value to the neuroscience modelling community comes from the implementation of these two fast Python neural model simulations. Because these simulations are written in Python, the code that implementations them is easy to understand share and execute, because the models are fast to dispatch they are useful in both network and optimization contexts where performance is critical. 
 
%A component of this work the Izhikevich model, I plan to push to the Open Source Brain %\href{https://github.com/russelljjarvis/IzhikevichModel.git}

\subsubsection{Why Existing Approaches Were Not workable}
In this research effort significant time was spent shoe-horning pre-existing tools into an optimization frame work, with limited success. These tools included:  PyNN, brain2, \emph{NEURON}, jNeuroML-2. However, several unexpected road blocks were encountered on the way.
 
When applying reduced models to optimization, there was a cluster of known errors and limitations in pre-existing community standard reduced model simulators. 

To make optimization of models tractable it was important to do ongoing feasibility testing. For example its important to evaluate the the utility of established model implementations, as using these models to optimize may not in fact be feasible.\\ 
Despite an a large number of choices of FOSS reduced model implementations, many off the shelf implementations were not useful, or significant intervention was required to make some established implementations workable inside an optimization framework. \\
 
Some revisions to the Izhikevich model include equation tweaks, in order to let cells better reproduce in-vivo firing dynamics, such as bursting, and re-bound spiking \url{https://github.com/OpenSourceBrain/IzhikevichModel/blob/master/MATLAB/izhi2007.m}. However, when multiple equations, however closely related, are incorporated under the same umbrella, this fractures the implementation of the model into sub-models. The \emph{NEURON} implementation of the Izhikevich model is fractured. There are different implementations for different regimes. This meant that switching between model regimes during optimization would be a non-trivial exercise because the is already splintered between two languages (NEURON, NMODL, python), the source code would be highly complicated such that it would not be readable, and it would loose generality, overly customized towards one target model. The NEURON code, needed NMODL files to be compiled for each different regime, and promised performance of the C based NEURON library was not actualized. The only benefit to using this model, was that its parameter ranges are well tested and understood within OSB and NeuroML-2 community.

Because the PyNN implementations of the Izhikevich models are "wrapped" versions of essentially NEURON models, PyNN models still retained the disadvantages of the NEURON models (surprisingly poor speed on the Izhikevich model) while introducing other problems. PyNN is designed to preference the description and implementation network simulations. A data-type called a "lazy-array" is the smallest elemental container for neuron models in PyNN, but it is meant to store populations of neurons as opposed to single neurons, as such the lazy-array often slows down and gets in the way of the single neuron simulations, which model fitting depends on.
%(lazy array evaluation). 

 Sub models $[1-3]$ are really different parameterizations of the same equation, sub-models $[4-7]$ are actually different equations, meaning that there is a total of $4$ Izhikevich sub-models. To permit optimizer access to the broadest range of Izhikevich dynamics, a new meta-parameter $cell_type$, was created. This number ranging $[3-7]$ (unique model equations included only), allows the optimizer to change which Izhikevich sub-model it samples from. Later in on in results, we see that in fact even when fitting to static electrical properties, optimizer access to the broader set of firing regimes improved the quality of fits.


%The described limitations of the 
%https://github.com/NeuralEnsemble/PyNN/issues/370$%}, journal={GitHub}, author={Davison A}
%}


%@misc{neuralensembleadexp, title={pyNN.neuron %implementation of AdExp is unstable, gives poor results  Issue $#266 · NeuralEnsemble/PyNN$}, url={$https://github.com/NeuralEnsemble/PyNN/issues/266$}, journal={GitHub}, author={Davison A.}
%}
\url{https://github.com/NeuralEnsemble/PyNN/issues/370} 
%\cite{neuralensemble_adexp}
\url{https://github.com/NeuralEnsemble/PyNN/issues/266} 
%\cite{neuralensemble_adexp2}
 
A constant error warning plagued brian2 investmentallations was.
\begin{verbatim}
Brian2 causes error:
 ERROR      Brian 2 encountered an unexpected error. If you think this is bug in Brian 2, please report this issue either to the mailing list at <http://groups.google.com/group/brian-development/>, or to the issue tracker at <https://github.com/brian-team/brian2/issues>. Please include this file with debug information in your report: /tmp/brian_debug_t0acbm4l.log  Additionally, you can also include a copy of the script that was run, available at: /tmp/brian_script_juzhsbph.py Thanks! [brian2]
Traceback (most recent call last):
\end{verbatim}
In two classes of model a feasible choice of implementation did not exist and it was easier to re-implement those models. The two models I re-implemented were
the Adaptive Exponential Integrate and fire Model, and also the IZHI
model.  In the work below, I profile existing model implementations, and
justify the reasons for re-implementing.\\
\\
This is in contrast to the brian2/neuraldynamics AdExp model, which took
between 2 or 3 times longer to find a rheobase current injection value. However the slowness is not caused by the simulation backend (brian2 which is relatively fast and efficient). The slow down is caused by the way the model is defined. Specifically the
model is defined in a middle code layer neurodynamics\cite{gerstner2014neuronal}.\\
\\
It is very likely, that the model implementation is correct, since Gerstner is an author of one of the original adaptive exponential publications, and the neural dynamics book that the brian2 code is strongly affiliated with. Since Integrate and Fire models don't formally include spikes when an implementation does include spikes, it is an optional add on.\\
\\
The AdExp neurodynamics models default implementation causes spikes with peaks below $0mV$, since the IZHI model like all integrate and fire models do not explicitly include spikes\\
\\
This is not technically wrong, but it violates
assumptions in the \emph{NeuronUnit} feature extraction protocol. The default spiking behavior, looks odd, and it is simply this poor model definition that is causing a slower optimizer performance. The optimizer takes an unusual waveform shape, and searches for longer in distant
parameter regions to find a good fit.\\
\\
Over the course of evaluating the brian neural dynamics model \cite{gerstner2014neuronal}. I experienced some phenomena that only occurred in the context of genetic algorithm optimization. The reason why optimization provides a different evaluation context is because, in optimization simulation objects are required to be created and destroyed rapidly and on mass. Brian2 is designed to be an efficient network simulator, the case of being designed for network simulation, may assume you will want to create a lot of neural models that persist efficiently together in memory (this was also a problem with PyNN models). Therefore you might see below, that while only one brian model exists in memory, performance is okay, but when creating and destroying models rapidly and on mass a slow down occurs.\\
\\
Below I have implemented a Python integrator for the adaptive exponential integrate and fire model. This solver led to faster evaluations of current injection experiments. The integrator I developed has a $0mV$ spike when evaluated at default parameter values.\\

Brian2 and SciUnit sometimes collided in name space and logging.
%\href{https://github.com/scidash/sciunit/pull/124/files/83907ba68740642178ebb91084f6e382e06a43c4#diff-d68791d2ed5dfaa96a900be6180bd950}

\section{Profiling the JIT Enabled AdExp Model}
Mean time taken on single model evaluation:$ 0.0012554397583007812s $
Mean time taken to compute Rheobase:
$0.183s $
This was slightly faster than Izhikevich implementation which was for total rheobase solution $ 0.462s $ and $  0.002 s$ per run. Solving for Rheobase takes a average of 15 model evaluations.
%\begin{figure}    
%\begin{center}
%\includegraphics[width=0.25\linewidth]{figures/backend_check_files/backend_check_6_2.png}
%\caption{}

%\end{center}
%\end{figure}

%\begin{verbatim}
%    251 ms +- 5.02 ms per loop (mean +- std. dev. of 2 runs, 1 loop each)
%    240 ms +- 11.1 ms per loop (mean +- std. dev. of 2 runs, 1 loop each)
%    223 ms +- 12.5 ms per loop (mean +- std. dev. of 2 runs, 1 loop each)
%\end{verbatim}

\begin{verbatim}
    922 ms +- 12.7 ms per loop (mean +- std. dev. of 7 runs, 1 loop each)
\end{verbatim}

\subsection{Comparison of Parallel and Serial Speeds and Accuracy}

Below is the Brian2/NeuralDynamics AdExp model. In-order to make the spike height greater than $0mV$ it was easier to use computer code to schedule waveform modifications that occur straight after the the brian2 simulation, these scheduled waveform modifications can be considered part of a peripheral shell of simulation code. In postprocessing
the waveform data type is a Neo Wave form object that is artificially the algorithm of determining rheobase and displaying results. The time of this model is determined on multiple factors, as discussed elsewhere, execution time is not uniform across model parameterizations. Models with multispiking behavior will take longer to solve.

Simulation times for this model vary, dramatically possibly because of
lazy evaluation, the simulation times may vary according to what else
you are running on your computer. Not all models experience a speed up when executed in parallel, however
this model was faster in the parallel Rheobase determination algorithm. Some common times are: $3.92,6.75,4.48,5.17$. Mean time was:

\subsection{Comparison of Times Taken to Find Rheobase}

Custom implementation JIT enabled implementation: $4.0s$. 
Brian2 taken to find Rheobase: $4.40s$ (serial), $3.976s$ (parallel).

The evaluation times between Brian2 and the custom written
integrator are similar. Both have average rheobase solution times of approximately 4 seconds, however the spike shape derived from the custom written integrator look more realistic under default paramaterizations. The biological plausibility of default model paramaterization has consequences for model optimization speed, because when  models undergo mutation and cross-over the mean of random models regressors towards the default model initialization, and if the default model is a bad fit to data, the average model sampled by the genetic algorithm will also be bad to data.\\

\begin{figure}
\centering
\includegraphics[scale=0.45]{figures/backend_check_files/backend_check_12_10}
\caption{Model parameterization of the brian2 simulator with the customization: interpolated spike height, forced to be above $0mV$}
\label{fig:sub1}
\end{figure}

\begin{figure}
\centering
\includegraphics[scale=0.45]{figures/backend_check_files/backend_check_4_2}
\caption{Default model parameterization of the custom written integrator}
\label{fig:sub2}
\caption{Comparison between two Adxaptive Exponential Implementations}
\label{fig:test}
\end{figure}




    
$272 ms +- 66.5 $/mu$s per loop (mean +- std. dev. of 2 runs, 1$ loop each)

The next model to be evaluated is the NEURON Izhikevich model. The NEURON Izhikevich model has various draw backs. 1. It depends on an external file which must be recompiled each time this project is recreated. 2. The build environment of NEURON is non-trivial, and only a super dedicated NEURON modeller would install it on their system. Any performance advantage of using NEURON investment does not exceed the installation cost of installing the program. 3. The model implementation code is less generalizable than than the published Izhikevich model itself. Where the standard NEURON-NeuroML code only covers the Regular-Spiking model * This is likely due to a name space conflict between Capacitance. Neuron has a `capacitive' mechanism inside modelled Neurons, this particular model has section capacitance as well as an introduced capacitive term inside a C-compiled mechanism. Both contribute to a the membrane
potential calculation. * The NEURON Izhi model took $78$ seconds to find the rheobase current injection value $ 51.79367065 * pA $.

    
%\begin{center}
 %   \includegraphics[width=0.7\textwidth,]{chapters/figures/backend_check_files/backend_check_14_2.png}
%    \caption{where is picture}
%\end{center}


%\begin{figure}
%    \centering
%    \includegraphics{chapters/normal_distribution}
%    \caption{This is your image%}
%    \label{fig:my_label}
%\end{figure}
%A tool numba JIT

% https://www.overleaf.com/learn/how-to/Images_not_showing_up 

        
The enabled forward Euler python Izhikevich model was very fast. The forward euler
implementation utilized Numba JIT \cite{lam2015numba}. Rheobase is found in under a second,
and in many cases close 0.5 seconds. This represents a very dramatic
speed up. Unlike the NEURON NeuroML implementation of the izhikitich equation,
this implementation is just as generalizable as the original MATLAB
implementation of the Izhikevich model, because it was possible to unify the fractured implementations in the one python simulator backend.

\subsection{NEURON+Python single compartment Conductance Model.}

Conductance based models took approximately the same amount of time to evaluate the Rheobase search algorithm as the python implementation.

The author also engineered GLIF model support for $NeuronUnit$ tests. In practice these models where hard to configure without expert knowledge, GLIF models contain the most parameters of all models, and many of these parameters are multi dimensional. GLIF models do not by necessity spike, interpolated spike times, are added in however, it does not make sense to evaluate GLIF models on spike shape features.

It is worth noting that the layer 5 neocortical pyramidal neuron was very slow to dispatch relative to the reduced models developed in this thesis work. Where as a typical reduced model described here evaluated in the order of $2.5 ms$, this model on average took $5.74$s, for a single run and $34.8$s to solve for the models Rheobase, current. To be fair, the model was run without activating NEURONs variable time step cvode. However, even with variable time step applied to the differential equation solver the magnitude of the disparity is still still several $seconds:$ several $ ms$. 

% time taken to compute rheobase $ 12.6s $


%\begin{verbatim}
%  time taken on
%  block 0.6859951019287109 \textbackslash{}n3.3 ms +- 9.79 %$\mu$s per loop (mean +- std. dev. of 2
%  runs, 100 loops each)\textbackslash{}n3.32 ms +- 30.9 us per loop (mean +- std. dev. of 2 runs,
%  100 loops each)\textbackslash{}n3.19 ms +- 10.9 us per loop (mean +- std. dev. of 2 runs, 100
%\end{verbatim}
        


%This problem in the default parameterization of the python model was later located in the scale or units of capacitance, if default capacitance parameterization is multiplied by 100.0 the problem goes away.


%\begin{center}
%\includegraphics{figures/backend_check_files/backen%d_check_22_2}
%\end{center}

%$ 1.40762329 * pA $


% \subsection{NEURON versions of single compartment Conducance
% model.}

% Took $8.57$ seconds to find Rheobase.

%Hodgkin Huxley Conductance based channels models took approximately the same amount of time to evaluate the Rheobase search algorithm as the python implementation.

%The NEURON implementation was slightly faster, and the default parameterization of the model lacked `ringing'', or below threshold oscillations that the Python ODE version had under default conditions.

%This problem in the default parameterization of the python model was later located in the scale or units of capacitance, if default capacitance parameterization is multiplied by 100.0 the problem goes away.


%which makes debugging their behavior very difficult. %None the less GLIF models where among the fastest to evaluate, and the author had success in making fitting these models to Allen Rheobase data.

    %\graphicspath{ {../figures/} }
%    \begin{center}
%    \begin{figure}
%    \includegraphics{figures/backend_check_files/backend_check_26_2}
    %kend_check_files/backend_check_26_2.png}
%    \end{figure}
    
%    \end{center}
%\begin{verbatim}
% 112.5 pA
%'value': array(1.40645904) * pA
%\end{verbatim}

% parameters of an adaptive exponential model
%\begin{verbatim}
%\{'El\_reference': -0.07016548013687134, %'C': 3.990452661875942e-10%,
%'init\_threshold': 0.02964956889477108, %'th\_inf': 0.02964956889477108,
%'spike\_cut\_length': 109.5, %'init\_voltage': -35.0, 'R\_input': %910258965.9792937\}
%\end{verbatim}
%$ Rheobase = 112.5pA $
%time taken to execute GLIF model when deliberately undersampling to save time.
%$ 0.23476457595825195 $

    


%$ 112.5 pA $
%$0.0 mV$ $-0.065 mV$

%    \begin{verbatim}
%    \{'value': array(183.33333333) * pA\}
%    \end{verbatim}

%\begin{verbatim}
%array(112.5) * pA
%\end{verbatim}


%\begin{verbatim}
%    0.017240506310425608 mV -0.08583939747094235 mV
%    0.017240506310425608 mV% -0.08583939747094235 mV
%\end{verbatim}

    %\begin{center}
    %\includegraphics{figures/backend_check%_files/backend_check_32_2.png}
    %\end{center}

%\input{chapters/methods/rheobase}
\section{Model Class Implementations}
For various reasons described below, existing implementations of some models were not adequate for this research.
These reasons included speed, generality, and consistency.

\section{Model Execution: The Need for Speed}
Because optimization may involve an extremely large number of independent simulations of the same class of model, each varying only in model parameter values, it is critical that both the overhead for model instantiation and the duration of simulation itself, be as low as possible.
Existing modeling tools contain overhead associated with model initialization, shuttling results in memory, which are a trivial cost for single simulations, but begin to add up in optimization runs of thousands of simulations.  
Even the simulation of reduced models themselves are often slower than necessary using existing tools, due to some of these tools being written to accommodate more complex, biophysical models.
To overcome this and accelerate optimization, I built faster "pure python" implementations of two neuronal models (the Izhikevich model and the AdEx model).
One of the these was inspired from the existing MATLAB forward euler implementation of the Izhikevich model, while the other was adapted from an existing python implementation of the AdEx model using vectorized code.
While neither of these was especially fast, they provided the basic recipe upon which a faster Python implementation could be built.
Do note that the purpose of these new implementations was not model exploration, analysis, or sharing; existing tools are adequate for these purposes.
The purpose of the new implementations was simply to make large optimization runs computationally tractable.

Python does not have a reputation for speed; implementation details have a large impact on performance.
Therefore, I used a tool called \cite{numba} that enables Just-In-Time compilation (JIT) of Python code, making at fast as e.g. compiled C code.
This tool cannot be applied to any arbitrary Python code, so functions to which it is applied much be designed with only a subset of the usual syntax and library of Python.
In other words, it cannot be used to simply speed up any pre-existing Python code.
I hand-coded the two models above to be JIT-compliant, with the result that both became significantly faster than analogous models using NEURON or Brian2 simulators.
Importantly, simulation outputs retained a binary near-match in all cases, confirming that nothing was lost in the course of gaining this performance improvement.
I used these new implementations extensively throughout the project, and they are available to others at \href{https://github.com/russelljjarvis/IzhikevichModel} and XXXX.
The code that implements them is fairly easy to understand, share, and execute, and I hope they may be useful to others who have similar performance needs, either for optimization contexts or large network models where small performance gains are worth chasing.

\subsection{Model Design: Lack of Generality}
Significant time was spent in the early years of this project shoe-horning pre-existing tools into the desired optimization framework, with limited success.
These tools included, among others, model designers and neural simulators such as PyNN, Brian2, NEURON, and jNeuroML.
However, several unexpected road blocks were encountered on the way.

The \emph{NEURON} implementation of the Izhikevich model is fractured.
There are different implementations for different parameter regimes.
For a single simulation, this is not much of a problem.
However, it means that switching between such regimes during optimization (as would occur when a parameter value crossed a regime boundary), is a non-trivial exercise.
Even if successful, any source code successfully implementing this would be complicated, unreadable, and lack generality.
Specifically, NEURON requires NMODL files to be compiled for each different regime, and it may be difficult to know in advance which regimes the optimizer is likely to sample from.
Thus, the claimed performance of the C-based NEURON library is not actualized in an optimization context.
Because NEURON is well-understood within OSB and NeuroML community, I still used it only to produce reference simulations to verify that the output of my model implementations were in fact accurate.

PyNN provides the convenience of working in Python, and with a convenient procedural interface for model design and execution.
However, its implementations of most reduced models (e.g. Izhikevich) are simply "wrapped" versions of NEURON models; consequently PyNN has the same disadvantages as NEURON.
PyNN is also designed with network simulations in mind, which means its designers have chosen performance trade-offs that favor network simulations over single neuron simulations.
For example, a data-type called the "lazy-array" is the most elemental container for neuron models in PyNN, but it is meant to store populations of neurons as opposed to single neurons;
as such the lazy-array results in slow single neuron simulations \url{https://github.com/NeuralEnsemble/PyNN/issues/370}, \url{https://github.com/NeuralEnsemble/PyNN/issues/266}, which optimization depends on.
% The above links should be turned into citations.

A constant error warning plagued brian2 investmentallations was.
\begin{verbatim}
Brian2 causes error:
 ERROR      Brian 2 encountered an unexpected error. If you think this is bug in Brian 2, please report this issue either to the mailing list at <http://groups.google.com/group/brian-development/>, or to the issue tracker at <https://github.com/brian-team/brian2/issues>. Please include this file with debug information in your report: /tmp/brian_debug_t0acbm4l.log  Additionally, you can also include a copy of the script that was run, available at: /tmp/brian_script_juzhsbph.py Thanks! [brian2]
Traceback (most recent call last):
\end{verbatim}

To make optimization of models tractable it was important to do ongoing feasibility testing. For example its important to evaluate the the utility of established model implementations, as using these models to optimize may not in fact be feasible.\\ 
\\
Despite an a large number of choices of FOSS reduced model
implementations, many off the shelf implementations were not useful, or significant intervention was required to make some established implementations workable inside an optimization framework. \\
\\
In two classes of model a feasible choice of implementation did not exist and it was easier to re-implement those models. The two models I re-implemented were
the Adaptive Exponential Integrate and fire Model, and also the IZHI
model.  In the work below, I profile existing model implementations, and
justify the reasons for re-implementing.\\
\\
This is in contrast to the brian2/neuraldynamics AdExp model, which took
between 2 or 3 times longer to find a rheobase current injection value. However the slowness is not caused by the simulation backend (brian2 which is relatively fast and efficient). The slow down is caused by the way the model is defined. Specifically the
model is defined in a middle code layer neurodynamics\cite{gerstner2014neuronal}.\\
\\
It is very likely, that the model implementation is correct, since Gerstner is an author of one of the original adaptive exponential publications, and the neural dynamics book that the brian2 code is strongly affiliated with. Since Integrate and Fire models don't formally include spikes when an implementation does include spikes, it is an optional add on.\\
\\
The AdExp neurodynamics models default implementation causes spikes with peaks below $0mV$, since the IZHI model like all integrate and fire models do not explicitly include spikes\\
\\
This is not technically wrong, but it violates
assumptions in the \emph{NeuronUnit} feature extraction protocol. The default spiking behavior, looks odd, and it is simply this poor model definition that is causing a slower optimizer performance. The optimizer takes an unusual waveform shape, and searches for longer in distant
parameter regions to find a good fit.\\
\\
Over the course of evaluating the brian neural dynamics model \cite{gerstner2014neuronal}. I experienced some phenomena that only occurred in the context of genetic algorithm optimization. The reason why optimization provides a different evaluation context is because, in optimization simulation objects are required to be created and destroyed rapidly and on mass. Brian2 is designed to be an efficient network simulator, the case of being designed for network simulation, may assume you will want to create a lot of neural models that persist efficiently together in memory (this was also a problem with PyNN models). Therefore you might see below, that while only one brian model exists in memory, performance is okay, but when creating and destroying models rapidly and on mass a slow down occurs.\\
\\
Below I have implemented a python integrator for the Adaptive
Exponential Integrate and fire model. This solver lead to faster
evaluations of current injection experiments. The integrator I developed
had a $0mV$ spiked when evaluated at default
parameter values.\\


Brian2 and sciunits sometimes collided in name space, and logging.
%\href{https://github.com/scidash/sciunit/pull/124/files/83907ba68740642178ebb91084f6e382e06a43c4#diff-d68791d2ed5dfaa96a900be6180bd950}

\section{Profiling the JIT enabled AdExp Model}
Mean time taken on single model evaluation:$ 0.0012554397583007812s $
Mean time taken to compute Rheobase:
$0.183s $
This was slightly faster than Izhikevich implementation which was for total rheobase solution $ 0.462s $ and $  0.002 s$ per run. Solving for Rheobase takes a average of 15 model evaluations.
%\begin{figure}    
%\begin{center}
%\includegraphics[width=0.25\linewidth]{figures/backend_check_files/backend_check_6_2.png}
%\caption{}

%\end{center}
%\end{figure}

%\begin{verbatim}
%    251 ms +- 5.02 ms per loop (mean +- std. dev. of 2 runs, 1 loop each)
%    240 ms +- 11.1 ms per loop (mean +- std. dev. of 2 runs, 1 loop each)
%    223 ms +- 12.5 ms per loop (mean +- std. dev. of 2 runs, 1 loop each)
%\end{verbatim}

\begin{verbatim}
    922 ms +- 12.7 ms per loop (mean +- std. dev. of 7 runs, 1 loop each)
\end{verbatim}

\subsection{Compare parallel to serial speeds, and accuracy}

Below is the Brian2/NeuralDynamics AdExp model. In-order to make the spike height greater than $0mV$ it was easier to use computer code to schedule waveform modifications that occur straight after the the brian2 simulation, these scheduled waveform modifications can be considered part of a peripheral shell of simulation code. In postprocessing
the waveform data type is a Neo Wave form object that is artificially the algorithm of determining rheobase and displaying results. The time of this model is determined on multiple factors, as discussed elsewhere, execution time is not uniform across model parameterizations. Models with multispiking behavior will take longer to solve.

Simulation times for this model vary, dramatically possibly because of
lazy evaluation, the simulation times may vary according to what else
you are running on your computer. Not all models experience a speed up when executed in parallel, however
this model was faster in the parallel Rheobase determination algorithm. Some common times are: $3.92,6.75,4.48,5.17$. Mean time was:

\subsection{Comparison of Times Taken to Find Rheobase}
Custom implementation JIT enabled implementation: $4.0s$. 
Brian2 taken to find Rheobase: $4.40s$ (serial), $3.976s$ (parallel).

The evaluation times between Brian2 and the custom written
integrator are similar. Both have average rheobase solution times of approximately 4 seconds, however the spike shape derived from the custom written integrator look more realistic under default paramaterizations. The biological plausibility of default model paramaterization has consequences for model optimization speed, because when  models undergo mutation and cross-over the mean of random models regressors towards the default model initialization, and if the default model is a bad fit to data, the average model sampled by the genetic algorithm will also be bad to data.\\

\begin{figure}
\begin{center}
\centering
  \centering
\includegraphics[scale=0.45]{figures/backend_check_files/backend_check_12_10}
\caption{Model parameterization of the brian2 simulator with the customization: interpolated spike height, forced to be above $0mV$}

  \label{fig:sub1}
  \centering
  \includegraphics[scale=0.45]{figures/backend_check_files/backend_check_4_2}
    \caption{Default model parameterization of the custom written integrator}
  \label{fig:sub2}
\caption{Comparison between two Adxaptive Exponential Implementations}
\label{fig:test}
\end{center}
\end{figure}




    
$272 ms +- 66.5 $/mu$s per loop (mean +- std. dev. of 2 runs, 1$ loop each)

The next model to be evaluated is the NEURON Izhi model. The NEURON Izhi model has various draw backs. 1. It depends on an external file which must be recompiled each time this project is recreated. 2. The build environment of NEURON is non-trivial, and only a super dedicated NEURON modeller would install it on their system. Any performance advantage of using NEURON investment does not exceed the installation cost of installing the program. 3. The model implementation code is less generalizable than than the published Izhi model itself. Where the standard NEURON-NeuroML code only covers the Regular-Spiking model * This is likely due to a name space conflict between Capacitance. Neuron has a `capacitive' mechanism inside modelled Neurons, this particular model has section capacitance as well as an introduced capacitive term inside a C-compiled mechanism. Both contribute to a the membrane
potential calculation. * The NEURON Izhi model took $78$ seconds to find the rheobase current injection value $ 51.79367065 * pA $.

    
%\begin{center}
 %   \includegraphics[width=0.7\textwidth,]{chapters/figures/backend_check_files/backend_check_14_2.png}
%    \caption{where is picture}
%\end{center}


%\begin{figure}
%    \centering
%    \includegraphics{chapters/normal_distribution}
%    \caption{This is your image%}
%    \label{fig:my_label}
%\end{figure}
%A tool numba JIT

% https://www.overleaf.com/learn/how-to/Images_not_showing_up 
        
The forward Euler python IZhi model is very fast. The forward euler
implementation utilized Numba JIT \cite{lam2015numba}. Rheobase is found in under a second,
and in many cases close 0.5 seconds. This represents a very dramatic
speed up. Unlike the NEURON NeuroML implementation of the izhikitich equation,
this implementation is just as generalizable as the original MATLAB
implementation of the izhikitich model.

%\begin{verbatim}
%  time taken on
%  block 0.6859951019287109 \textbackslash{}n3.3 ms +- 9.79 %$\mu$s per loop (mean +- std. dev. of 2
%  runs, 100 loops each)\textbackslash{}n3.32 ms +- 30.9 us per loop (mean +- std. dev. of 2 runs,
%  100 loops each)\textbackslash{}n3.19 ms +- 10.9 us per loop (mean +- std. dev. of 2 runs, 100
%\end{verbatim}
        
\section{Python/LEMS and NEURON versions of single compartment Conducance Model.}

Conductance based models took approximately the same amount of
time to evaluate the Rheobase search algorithm as the python
implementation.

%This problem in the default parameterization of the python model was later located in the scale or units of capacitance, if default capacitance parameterization is multiplied by 100.0 the problem goes away.

time taken to compute rheobase $ 12.6s $

%\begin{center}
%\includegraphics{figures/backend_check_files/backen%d_check_22_2}
%\end{center}

%$ 1.40762329 * pA $


\subsection{NEURON versions of single compartment Conducance
model.}

Hodgkin Huxley Conductance based channels models took approximately the same amount of time to evaluate the Rheobase search algorithm as the python implementation.

%The NEURON implementation was slightly faster, and the default parameterization of the model lacked `ringing'', or below threshold oscillations that the Python ODE version had under default conditions.

%This problem in the default parameterization of the python model was later located in the scale or units of capacitance, if default capacitance parameterization is multiplied by 100.0 the problem goes away.

    \begin{verbatim}
time taken on block 8.573923826217651
    \end{verbatim}

The author also engineered GLIF model support for $NeuronUnit$ tests. In practice these models where hard to configure without expert knowledge, GLIF models contain the most parameters of all models, and many of these parameters are vectors, not scalars. GLIF models do not visualy spike which makes debugging their behavior very difficult. None the less GLIF models where among the fastest to evaluate, and the author had success in making fitting these models to Allen Rheobase data.

    %\graphicspath{ {../figures/} }
%    \begin{center}
%    \begin{figure}
%    \includegraphics{figures/backend_check_files/backend_check_26_2}
    %kend_check_files/backend_check_26_2.png}
%    \end{figure}
    
%    \end{center}
%\begin{verbatim}
% 112.5 pA
%'value': array(1.40645904) * pA
%\end{verbatim}

% parameters of an adaptive exponential model
%\begin{verbatim}
%\{'El\_reference': -0.07016548013687134, %'C': 3.990452661875942e-10%,
%'init\_threshold': 0.02964956889477108, %'th\_inf': 0.02964956889477108,
%'spike\_cut\_length': 109.5, %'init\_voltage': -35.0, 'R\_input': %910258965.9792937\}
%\end{verbatim}
$ Rheobase = 112.5pA $

time taken to execute this model.
$ 0.23476457595825195 $

    


%$ 112.5 pA $
%$0.0 mV$ $-0.065 mV$

%    \begin{verbatim}
%    \{'value': array(183.33333333) * pA\}
%    \end{verbatim}

%\begin{verbatim}
%array(112.5) * pA
%\end{verbatim}


%\begin{verbatim}
%    0.017240506310425608 mV -0.08583939747094235 mV
%    0.017240506310425608 mV% -0.08583939747094235 mV
%\end{verbatim}

    %\begin{center}
    %\includegraphics{figures/backend_check%_files/backend_check_32_2.png}
    %\end{center}

\section{Parallel Rheobase Determination}\label{sec:parallel-rheobase}
In the preceding section, I discussed determination of the rheobase of a model as one of most computationally expensive steps in evaluating a given set of model parameters. 

\subsection{Why is the Rheobase Important?}
The rheobase is defined as the minimum current required to elicit at least one action potential.
In slice physiology experiments, this usually means a square pulse of somatically-injected current lasting for a fixed amount of time, for example $500 ms$.
The rheobase not only characterizes the excitability of a cell, but it also serves as a landmark or anchor for computing many other features of a cell's suprathreshold behavior.
For example, once the rheobase is known, one can compute a so-called "FI curve" -- the number or frequency of action potentials in response to a given amount of injected current -- at fixed multiples of the rheobase, providing a compact summary of excitability.
Both the Allen Institute and the Blue Brain Project use such rheobase-linked excitability measures.
The rheobase current can also be used to compute features of spike waveforms.
These features may vary with the amount of injected current, because the rising phase of an action potential may include both sodium current and pipette currents.
By using the rheobase current, this latter confound is minimized because the patch pipette current is roughly offset by outward currents (were the pipette current any less, the outward currents would have prevented a spike, by the definition of rheobase).
Consequently, action potential waveform features like threshold, width, and height are often performed at the rheobase current.

\subsection{How is the Rheobase Determined?}
Determining the rheobase involves repeated application of a more general algorithm that runs one simulation to determining the number of action potentials evoked by a particular magnitude and duration of somatic current injection.
Because the rheobase value partitions suprathreshold stimulus amplitudes from subthreshold ones, its determination can be accelerated by treating as a search tree problem.
In a search tree, the search space is adaptively narrowed between two endpoints until a target is identified.
For the rheobase, this means asking (1) "What is maximum current injected so far that resulted in zero spikes?" and (2) "What is the minimum current so far that resulted in one or more spikes?" and then running a simulation at some current amplitude in between those two values (e.g. halfway between in the case of a binary search tree).

\subsubsection{Serial rheobase determination}
The procedure above can be run in serial (i.e one simulation after another) until the rheobase is narrowed down to an acceptably narrow range, e.g. +/- 1 pA.
The initial search begins with no knowledge of any minimum suprathreshold or maximum subthreshold current amplitudes, so I use the starting range 300 pA to -100 pA, respectively.
A binary search is applied within this space, with additional code to handle edge cases outside this range.
Ignoring those edge cases, such a binary search requires $log_2(I/i)$ simulations, where $I$ is the range being searched (here, 400 pA) and $i$ is the resolution of the solution (here, 1 pA).
Thus, ~9 simulations are required to obtain the rheobase using this binary search strategy.

\subsubsection{Parallel rheobase determination}
This process can be accelerated by running simulations in parallel.
While each step of the search requires knowledge of the outcomes of the previous simulations (and so there can be no parallelism across steps, other than brute force parallel search of the entire range of currents, which is extremely inefficient), it is possible to parallelize within each step.
A binary search partitions the search space in two by simulating a current injection at $(sub+super)/2$ pA, where $sub$ is the previous maximum subthreshold current, and $super$ is the previous minimum suprathreshold current.
The value $(sub+super)/2$ pA either does or does not produce a spike, leading to its value being used to update either $super$ or $sub$, respectively.
The search space is cut in half, so this simulation effectively generates one additional bit of information about the amplitude of the rheobase.
This repeats ~9 times until all 9 bits of uncertainty (from the initial 400 pA) range have been eliminated.
Parallelism accelerate this by applying an N-ary search (rather than a binary search), where N is the number of parallel processes, and N+1 the number of regions of current amplitude to search.
This is described in Figure \ref{fig:rheobase}.
Consider the initial 400 pA range.
With only one thread, this range is bisected and a simulation run at it midpoint $(-100 pA + 300 pA)/2 = 100 pA$.
With seven threads, this range can be octo-sected, with concurrent simulations run at each of seven values, i.e. ${-50, 0, 50, ..., 300, 350}$ pA.
The highest of these seven value that produces a spike is assigned to $sub$ and the lowest that does not to $super$, resulting in the search space now being restricted to only one of these 50 pA wide regions.
This is $1/8$ as a wide as the initial space, so 3 bits of information about the rheobase have been obtained.
This parallel process is then repeated serially (i.e. octo-section of the new 50 pA region, octo-section of the ensuing 6.25 pA region, etc.), until the rheobase has been determined.
Because 3 bits of information are obtained in every step instead of 1 bit, the search is 3x faster.
In general, the parallel N-ary approach is $log_2(N+1)$ faster than the plain serial binary search approach, with speedup gain therefore growing logarithmically in the number of concurrent threads (usually, proportional to the number of CPU cores) being used.
As architectures with hundreds of cores are now common, speedups of 7-10 fold are achievable.

\begin{figure}    
  \begin{center}
  \includegraphics[width=0.7\linewidth]{{figures/rheobase_algorithm.png}}
    \caption{We developed a generic algorithm which took models, and found the minimal current injection value that would cause only one spike. The normal structure of this algorithm is a binary search, however we modified the algorithm so it would map onto multiple processors at once. This lead to significant speed ups for multicompartment NEURON models}
    \label{fig:rheobase}
  \end{center}
\end{figure} 
    
\subsection{Practical application}
I benchmarked this approach using simulations of multi-compartment neuron models.
The parallel rheobase determination algorithm resulted in a significant speed up relative to the serial algorithm.
To amplify the effect, I also considered a scenario where one wants to learn the value of the rheobase with much more precision (down to small fractions of a pA), for example in studies of dynamics in the neighborhood of a bifurcation where all other state variables can be considered nearly unchanged in the sub- and suprathreshold scenarios.
To achieve such precision, i.e. for such a small value of $i$ $0.0001*pq.fA$, the number of simulations $log_2(I/i)$ may be ~20.
Since additional model features may require only a few additional simulations to extract, it is clear that in this scenario the rheobase completely dominates the bulk of the total simulation budget.

In this scenario, using only 16 threads (with a theoretical speedup of $log_2(16+1) ~ 4.09$), I achieved the following results:
\begin{verbatim}
NEURON simulation of multicompartmental model
Serial Rheobase determination: 18.7 s
Parallel Rheobase determination: 4.8 s
Speed up = 3.9x

Brian2 simulation of AdEx model
Serial Rheobase determination: 0.791 s
Parallel Rheobase determination: 0.259 s
Speed up = 3.0x
\end{verbatim}

The total speedup approached but fell a bit short of the theoretical speedup due to overhead in the parallel search algorithm itself.
As the complexity of each simulation increases, and as the number of CPU cores brought to bear increases, this overhead should become a vanishingly small fraction of the total rheobase determination time.

\subsection{Generalization to target spike counts}
This approach determining rheobase was also generalized into a more fundamental algorithm for determining the amplitude of current required to generate a target number of action potentials.
In other words, it can be used to invert points along a models "F-I" function.
\url{https://github.com/russelljjarvis/neuronunit/blob/master/neuronunit/tests/target_spike_current.py}.

\section{Electrophysiological Measurement Distributions from Experimental Literature}\label{sec:data-sources}
Organized, publicly available electrophysiological measurements from single, biological neurons can form an optimization target.
Together, they can be used to parameterize the suite of tests against which a model is optimized.
Optimization makes corresponding model electrophysiological measurements as similar as possible to those observed in biological neurons.

\subsection{NeuroElectro}
One general source of such measurements is The NeuroElectro Project \citep{tripathy2014neuroelectro}, which contains experimental values for 47 distinct electrophysiological measurements across 235 different neuron types.
As with most of the data discussed here, most (but not all) of these measurements were obtained from slice physiology experiments in rodents.
These measurements were programatically extracted from peer-reviewed journal articles over a $\sim20$ year period from $\sim1990-2012$,
and are made easy to access by an application programming interface (API) that NeuronUnit provides bindings to.
Importantly, the measured values--even for a single neuron type--reflect experiments done in many labs using (in some cases) variable methods.
Therefore, the mean of these values (e.g. the mean input resistance across reported Purkinje cells) averages over heterogeneity across cells within a slice, slices within an animal, animals within a lab, and labs within the field.
The sample size for one measure (e.g. input resistance) may be larger than for another (e.g. resting potential) meaning that they may reflect different subsets of experiments.
With those caveats in mind, NeuroElectro is still the most direct way to get a large number of optimization-constraining data values for most neuron types.

In order to verify that the data from NeuroElectro was plausible and was being captured correctly for the purposes of the work in this thesis, I used the API along with a batch visualization pipeline to  visualize the distributions of electrophysiological measurements and inspect them for a) quality control and b) evidence of multimodality.
Multimodality, meaning multiple peaks in the histogram of a single measurement type for a single cell, could be evidence of a physiological heterogeneity not easily explained by random measurement error.
Two peaks in the histogram, for example, could result from two distinct subclasses of a single nominal neuron type, each with its own (narrower) distribution of the same measurement.
In some instances, the mean and standard deviation alone well-described the measurement distributions, as would be expected for random, normally-distributed measurements of a single cell type under reasonably consistent conditions.
These values were then ``approved" for use in model-fitting.
In other cases, these conditions were not met, as exemplified in the figures below.

%\begin{comment}
%\begin{figure}
%\centering
%   \includegraphics[scale=0.8]{notebooks_converted/needata_thesis_files/needata_thesis_5_5}
%\end{figure}

%\caption{Model parameterization of the brian2 simulator with the customization: interpolated spike height, forced to be above $0mV$}
%
%  \label{fig:sub1}
%\end{subfigure}%
%\begin{figure}
%\centering
%  \includegraphics[scale=0.8]{notebooks_converted/needata_thesis_files/needata_thesis_5_6}
%\end{figure}

%    
%    %\caption{Default model parameterization of the custom written integrator}
%  \label{fig:sub2}
%
%\begin{subfigure}
%  \centering
%      \includegraphics[scale=0.8]{notebooks_converted/needata_thesis_files/needata_thesis_5_7}
%      %\caption{Default model parameterization of the custom written integrator}
%  \label{fig:sub2}
%\end{subfigure}
%
%\caption{Comparison between two Adxaptive Exponential Implementations}
%\label{fig:test}
%\end{center}
%\end{figure}
%
%\end{comment}

%\begin{center}
%\includegraphics[width=0.7\linewidth]{notebooks_converted/needata_thesis_files/needata_thesis_5_8}
%\end{center}
%
For the majority of cell types and electrophysiological features, the distributions obtained from NeuroElectro were well-described by a normal (or log-normal) distribution.
However, I manually identified and labeled those cases were the data were not so well-behaved, as these might be likely to produced optimized models that did not reflect anything of biological relevance.

Some methods for hypothesis testing that a distribution is unimodal exist \citep{maechler2013package}, however I achieved greater quality control by simply inspecting each case in a peace-meal graphic manner. % manually means to do something by hand. 
I estimated that across all NeuroElectro data sampled here, about $2/3$ is well represented by a unimodal and normal-ish distribution (e.g. Figure \ref{fig:normal-feature} and Figure \ref{fig:normal-feature2}.
In the remaining $1/3$ where this did not hold, I observed a small but still significant number of odd cases: highly skewed distributions (Figure \ref{fig:skewed-feature}), bimodal distributions (Figure \ref{fig:bimodal-feature}), uniform-like distributions (Figure \ref{fig:uniform-feature}), and distribution with insufficient samples to make any judgement at all.

\begin{figure} 
    \begin{center}
   \includegraphics[scale=0.8]{figures/mean_well_served.png}
   \caption[AP Threshold Data Distribution, Layer 5 Pyramidal Cell]{The majority of Neuroelectro data sets were well served by a Gaussian normal distribution. As you can see in this plot the mean is surrounded by a very high density of samples, which slowly thin out with increasing distance from the mean. The distribution is approximately symmetrical, or as symmetrical as one might hope when examining real data sets.}
   \label{fig:normal-feature}
    \end{center}
\end{figure}   

\begin{figure} 
    \begin{center}
    \includegraphics[scale=0.8]{figures/mean_well_served2.png}
    \end{center}
    \caption[AP Amplitude Data Distribution, Layer 5 Pyramidal Cell]{Although the data is skewed to the left and it has outliears. I believe this data distribution is symmetrical enough, to be considered "overall" a normal distribution. In this plot the mean is surrounded by a very high density of samples, which slowly thin out with increasing distance from the mean, although there are a small proportion of outliers these are likely explained by experimental noise.}
    \label{fig:normal-feature2}
\end{figure}   
 
\begin{figure} 
    \begin{center}   \includegraphics[scale=0.8]{figures/skewed_distribution.png}
    \end{center}
    \caption[Example of skewed NeuroElectro data]{In this distribution of Input Resistance for the Neo-cortical Pyramidal neuron, one can see that values are skewed, and fall off quickly to the right, in this case the median represents the middle of the data better than the mean or the mode.}
    \label{fig:skewed-feature}
\end{figure}   


%\begin{figure} 
%\caption[NeuroElectro data - uniform distribution]{}
%    \begin{center}
%    \includegraphics[scale=0.8]{figures/uniform_distribution.png}
%    \end{center}
%\end{figure}       
%%
% Neuronunit code handles under sampled neuroelectro code.
%\begin{figure} 
%\caption[NeuroElectro data - undersampled distribution]{}
%    \begin{center}
%    \includegraphics[scale=0.8]{figures/undersampled_distribution.png}
%    \end{center}
%\end{figure}   
%%
    
%\begin{figure} 
%    \begin{center}
%   \includegraphics[scale=0.8]{chapters/notebooks_converted/needata_thesis_files/needata_thesis_5_9}
%   \caption{The Action Potential Width of the Hippocampus CA1 basket cell possibly has either an underlying uniform distribution or a multimodal distribution. Since the samples are few, the true distribution is unknown. If the distribution is uniform the gaps in the distribution, that give the histogram a multimodal appearance, as the sample size is lower enough that such gaps may only represent missing samples.}
%    \end{center}
%\end{figure}


%\begin{figure}   
%\begin{center}
%   \includegraphics[scale=0.8]{chapters/notebooks_converted/needata_thesis_files/needata_thesis_5_21}
%         \caption[Input Resistance Olfactory Neuron, Perhaps Bimodal]{Input resistance of the Olfactory Mitral cell showed some tendency towards underlying bi-modal distribution, however in the second block of histogram bins, centered around $200-300pA$ only contains approximately $5$ samples. Due to a lack of samples it is also possible to conclude that the data belong to an under sampeled uniform distribution. This data set was important, as one Olfactory neuron test was constructed from this data.}
%\end{center}
%\end{figure}
   
\begin{figure}  
\begin{center}     
  \includegraphics[scale=0.8]{chapters/notebooks_converted/needata_thesis_files/needata_thesis_5_22}
      \caption{Among different measurement sources of neuroelectro data, the resting membrane potential of the Olfactory Mitral cell, showed the greatest tendency of an underlying bi-modal distribution
      In the top panel of this plot we see a binned histogram of Resting membrane potential in the olfactory mitral cell. Although excluded here for considerations of brevity, the input resistance of the olfactory neuron, also perhaps followed a bi-modal or uniform distribution. 
      }
      \label{fig:bimodal-feature}
\end{center}     
\end{figure}
%%
% There are plenty of examples of bi-modal distributions in measurements from cells which are not relevant to this work.
%
%\begin{figure}
%  \centering
%  \includegraphics[scale=0.8]{chapters/notebooks_conv%erted/needata_thesis_files/needata_thesis_5_6}
%   \caption{Default model parameterization of the custom written integrator}
%  \label{fig:sub2}
%\end{figure}
%\begin{figure}
%\begin{center}
%includegraphics{chapters/notebooks_converted/needata_thesis_files/needata_thesis_5_13}
%\end{center}
%\end{figure}
    
\begin{figure}
\begin{center}
\includegraphics[width=0.7\linewidth]{chapters/notebooks_converted/needata_thesis_files/needata_thesis_5_16}
\caption{This is a binned histogram of Neuroelectro spike width measurements in the Neocortical Pyramidal neuron.
The mode is of the distribution is denoted by a blue vertical line. In this way the mode of the data distribution can be compared to the mean in the box plot. Often modes, and means of the measurements disagree, as they do in the case of CA1 basket cell spike widths. When consulting NeuroElectro measurement, a very common distribution shape is one which is possibly uniform, or multi-modal. It is perhaps obvious but worth noting that a uniform distribution, is not well described by a normal distribution. Under a normal distribution a range of measurement values are equally likely to occur.}
\label{fig:uniform-feature}
\end{center}
\end{figure}

\begin{figure}
\begin{center}
\includegraphics[scale=0.8]{chapters/notebooks_converted/needata_thesis_files/needata_thesis_5_23}
\caption{Similar to the figure above, but for a different neuron type.}
\label{fig:uniform-feature2}
\end{center}
\end{figure}

\subsection{EFEL and The Allen Institute Cell Types Database}
The Blue Brain Project developed the Electrophysiology Feature Extraction Library (EFEL) \citep{EFEL}. Although EFEL computes common spike train statistics related to spike timing, approximately 2/3rds of features extracted by EFEL pertain to spike shapes, and some of those features are shown below. The Allen SDK comes with a very comprehensive python based feature extraction suite. Just like EFEL, the Allen suite does well to represent a large number of spike shape measurements as well as spike train statistics. Unfortunately, the Allen SDK feature extractor is significantly slower than EFEL as EFEL was implemented using the very fast language $C++$. The performance cost may not be felt when dispatching single runs, but slow performance is a significant impediment to optimization. In optimisation feature extraction is directly coupled to chromosome fitness calculations, and it is executed very often across the evolution of the genetic algorithm. Additionally by default, the AllenSDK feature extractor assumes that its user will be applying very high sample frequency and noisy NeuronData Without Borders \citep{teeters2015neurodata} encoded traces. These traces require filtering before they can have Allen features computed on them. Significant intervention is required to turn off filtering. In appropriately applying filtering to model traces causes problems, because the the lower sampling frequency intrinsic to simulated model traces is not predicted by the digital filter. Overall, the EFEL was fast enough to be useful, and its default settings where appropriate to my use case. 
% Membrane potential waveforms where already stored in a container Neo Neo 
\cite{garcia2014neo}


The data available through NeuroElectro covers a large number of cell types, but recording conditions and measurement algorithms are heterogeneous.
It is unclear if the distribution of measurements across such an ensemble is actually a good summary of any one, real cell.
In order to ensure that reduced models could be optimized against data recorded exclusively from single neurons, I also used the Allen Institute Cell Types Database \citep{celltypes}, a project of the Allen Institute for Brain Science.
This Cell Types database consists of summary physiological, morphological, and histological data for thousands of individual neurons (across a few dozen subtypes) from mouse visual cortex, obtained using patch clamp recordings in slices.
Each experiment is done using exactly the same methods and with the same sequence of stimuli (described in \cite{celltypes}), ensuring not only that models generated using this data are directly comparable, but that each such model is reflective of a single neuron.

The Cell Types Database does provide some limited pre-computed measures of action potential waveform details. 
However, the data is not organized in a way that makes it is useful for the types of optimization and data analysis I perform here.
Specifically, I require features that are computed on cell responses to current injection values that are fixed multiples of rheobase.
Additionally, the pre-computed features are thin relative to those that I brought to bear in the optimizations described in the Results section.
Because raw data is available through the Cell Types API, I therefore re-computed all necessary features from this raw data, according to the consistent standard reflected in the NeuronUnit code.

In contrast to NeuroElectro, the Cell Types database also has a great deal more information relevant to the above threshold dynamics of neurons, such as the number and pattern of action potentials they discharge in response to somatically-injected currents much larger than rheobase, or in response to non-square injected currents.
In order to exploit these, I developed several additional NeuronUnit tests using EFEL, such as: ``time to first spike test" and ``mean AP amplitude test". In principal any feature measured in the Cell Types Data could be upgraded to a NeuronUnit test, and I created a code-generation template to accelerate this task.
I also created some tests from scratch, especially those adapted from other descriptions of feature extraction from the literature (as described in sections \ref{sec:allensdk}, \ref{sec:efel}, and \ref{sec:elephant}. These are shown in Table \ref{tab:featurez}.


\begin{table}
\centering

\resizebox{\textwidth}{!}{
\begin{tabular}{ll}
            \textbf{Test Name} & \textbf{Test Description}\\
adaptation-index & Measures spiking fatigue in response to constant current\\
 adaptation-index2 & The same as Adaption index1, except it is used as an alternative when spikes below $0mV$ occur.   \\
time-to-first-spike & amount of time elapsed until first spike \\ mean-AP-amplitude & The average spike height in a spike train \\
spike-half-width & The width of a spike is obtained at point when spike height is half its total amplitude\\    
AHP-depth & The after hyperpolarisation depth\\
minimum-voltage & The minimum voltage\\
peak-voltage & the maximum voltage, usually a spike peak. \\
time-to-last-spike & The time of last spike onset \\
AHP-depth-abs & After Hyperpolarisation depth.\\
all-ISI-values & All interspike interval times\\
voltage-base & minimum voltage while undergoing stimulus, often below the threshold of APs. \\
min-voltage-between-spikes & Needed because during  high frequency firing AHPs may be skipped.\\
Spikecount & Just the number of spikes that occured in the provided stimulus window\\
\end{tabular}}
\caption[List of features]{A number of features that were encoded into NeuronUnit tests for optimization}
\label{tab:features}
\end{table}

Below I include some graphs of the many different features utilized in optimization.

\begin{figure}
    \centering
    \includegraphics{figures/voltage_features.png}
    \caption[Passive membrane properties probed with a negative amplitude current stimulus]{Although applying a negative current stimulus, only tends to activate so called ``passive" ion channels in the membrane, probing the neuron membrane in this way is simple way of ascertaining essential information about neuron membrane resistance and capacitance. Figure taken from \url{https://efel.readthedocs.io/en/latest/eFeatures.html}}
    \label{fig:voltage_figures}
\end{figure}

\begin{figure}
    \centering
    \includegraphics{figures/AP_Amplitude.png}
    \caption[]{When a neuron fires multiple spikes, many features in the waveform are readily measurable, these include, mean AP overshoot, every AP amplitude, wave troughs, spike threshold and after hyperpolarisation potential to name a few}
    \label{fig:features_example}
\end{figure}

\begin{figure}
    \centering
    \includegraphics{figures/sag_amplitude}
    \caption{Caption}
    \label{fig:sag_amplitude}
\end{figure}

%XXXX Russell, can you list the new tests above.

These tests can be used to assess the agreement between model and biological neurons on suprathreshold dynamics, largely reflected in patterns of spiking such as bursting and adaptation, but also mean spike height, and mean spike width

\subsection{The Blue Brain Project Neocortical Microcircuit Portal}
\label{sec:bluebrain-data}
I also made use of one additional data source, the The Blue Brain Project Neocortical Microcircuit Portal, similar in some ways to the Allen Institute Cell Types Database but reflecting measurements taken from mouse somatosensory cortex (again in patch-clamp recordings from slices).
From this dataset I exclusively used a collection of experiments from animal B-95, which for reasons unknown yielded a tremendous amount of data (\url{http://microcircuits.epfl.ch/#/animal/8ecde7d1-b2d2-11e4-b949-6003088da632}).
Conceptually, this dataset added anything new, but it did allow for high-quality optimized models to be produced from another brain region (somatosensory, rather than visual cortex).
These data are also linked to--and constrain--the on-going Human Brain Project effort to simulated biophysically-detailed multi-compartmental models of the same neurons (and whole neural circuits).
This means that the reduced models produced here can be compared directly to those more detailed models, or that the general NeuronUnit-driven genetic optimization framework developed here could be used to optimize detailed models which should, in principle, be similar to those produced through the larger Human Brain Project effort.
Indeed, the Human Brain Project is already a user of the SciUnit framework developed in my lab, on which NeuronUnit is based.

The tests which lead to the best fits in the above threshold experiments, where the tests derived from the above EFEL features applied to NeuronUnit data (Figure \ref{fig:supra-threshold-tests}), the measurement type, and the test type did not change between Allen Cell Types, and Blue Brain Data, only the reference data which informed comparison measurements changed.

The table below is constitutes a summary of both NeuroElectro and Allen experimental data reports. This data can naturally be reported in tabular form. 

\subsection{The Experimental Measurements}

\begin{table}[ht]
\centering
\resizebox{\textwidth}{!}{
\begin{tabular}{lllllllll}
\toprule
name & Hippocampus CA1 pyramidal cell & Cerebellum Purkinje cell & Neocortex pyramidal cell layer 5-6 &      olf\_mit &      623960880 &      623893177 &      471819401 &      482493761 \\
\midrule
RheobaseTest                   &                      189.24 pA &                680.79 pA &                          213.85 pA &          NaN &        70.0 pA &       190.0 pA &       190.0 pA &        70.0 pA \\
InputResistanceTest            &                    107.08 Mohm &              142.06 Mohm &                        120.67 Mohm &  130.08 Mohm &  241.0 megaohm &  136.0 megaohm &  132.0 megaohm &  132.0 megaohm \\
TimeConstantTest               &                        24.5 ms &                      NaN &                           15.73 ms &     24.48 ms &        23.8 ms &        27.8 ms &        13.8 ms &        24.4 ms \\
CapacitanceTest                &                        89.8 pF &                620.27 pF &                          150.58 pF &    235.75 pF &            NaN &            NaN &            NaN &            NaN \\
RestingPotentialTest           &                      -65.23 mV &                -61.59 mV &                          -68.25 mV &    -58.14 mV &       -65.1 mV &       -77.0 mV &       -77.5 mV &       -71.6 mV \\
InjectedCurrentAPWidthTest     &                        1.32 ms &                  0.41 ms &                            1.21 ms &      1.61 ms &            NaN &            NaN &            NaN &            NaN \\
InjectedCurrentAPAmplitudeTest &                       86.36 mV &                 71.23 mV &                           80.44 mV &      68.4 mV &            NaN &            NaN &            NaN &            NaN \\
InjectedCurrentAPThresholdTest &                       -47.6 mV &                -46.89 mV &                          -42.74 mV &     -38.9 mV &            NaN &            NaN &            NaN &            NaN \\
FITest                         &                            NaN &                      NaN &                         0.05 Hz/pA &          NaN &     0.18 Hz/pA &     0.12 Hz/pA &     0.18 Hz/pA &     0.09 Hz/pA \\
\bottomrule
\end{tabular}}
\end{table}


\section{Technical Details of the Optimizer}
\label{sec:tech-details}
The sections above describe my innovations in model construction and simulation, as well as the experimental data brought to bear on optimization. These data are used to parameterize NeuronUnit tests, one per measurement type.
For example, input resistance data for one neuron type in NeuroElectro, or one specific neuron in the Cell Types database, is passed to an \textit{InputResistanceTest} defined in NeuronUnit.
This test ``asks" the model to generate a corresponding simulation, measures the input resistance in this simulation output, and then assesses model/data agreement, resulting in a score.
These mechanics have been described at length previously in \cite{omar2014collaborative}, \cite{gerkin_neuronunit}, and \cite{birgiolas2019towards}.

\subsection{Generating and Using Scores}
\begin{figure}
\begin{center}
    \label{fig:normal-dist}
	\includegraphics{figures/normal_distribution}
    \caption[Z-scores for NeuronUnit Tests]{As discussed in the section \ref{sec:neuronunit}, error functions were evaluated with the assistance of the \emph{NeuronUnit} library.
    This involves obtaining an experimental distribution over electrophysiology feature measurements for a cell type, measuring corresponding model output features and then locating those features in that experimental distribution. 
    Scores that are closer to the experimental mean are identified as low error.
	The Z-score encodes this information; a Z-score of 0 is the lowest possible error.}
\end{center}
	
\end{figure}
%XXXX Something about Z-score vs RatioScore.

One way to ask whether the simulated feature is a good match to the biological data distribution is to use a Z-score.
The Z-score is defined as:
\begin{equation}
Z-Score = \frac{s - b_{\mu}}{b_{\sigma}}
\end{equation}
where $s$ is the value of the feature in the model simulation, and $b_{\mu}$ and $b_{\sigma}$ are the mean and standard deviation of that feature in the biological data distribution.
The Z-score does not specifically assume that the biological data are normally distributed, although this generates the most natural interpretations (Figure \ref{fig:normal-dist}).
In cases when the biological data from one neuron type comes from a single experiment on a single neuron (as with some data from the Allen Cell Types database or the Blue Brain Project), there is no mean or standard deviation, so I compute a \emph{RatioScore}:
\begin{equation}
Ratio-Score = \frac{s}{b}
\end{equation}
where $b$ is the observed biological feature value.
Both types of scores were then normalized to produce a an error signal in the range $(0, \inf)$ for use by the optimizer.
For example, suppose a feature(e.g. the rheobase current) had value $\mu \pm \sigma = (100pA \pm 40pA)$ in the biological data, and $110 pA$ in the simulated model output.
Then the following steps were taken to transform it into an error signal:
\begin{enumerate}
 \item A Z-score is computed: $\frac{110 pA - 100 pA}{40 pA} = 0.25$
 \item This is converted to the range (0, 1) using the error function: $abs(erf(Z)) = 0.27$ 
 \item The logarithm is computed: $\log_{10}(0.27) = 0.56$ 
\end{enumerate}
The value 0.56 above represents a larger model/data disagreement than the ``best" possible value of 0 (corresponding to a Z-score of 0), but less disagreement then the ``worst" possible value of $\inf$ (truncated in practice at 100) representing a Z-score of $+\inf$ or $-\inf$. The summed error signal over all $n$ NeuronUnit tests (e.g. rheobase, input resistance, spike rate adaptation, etc.) is:
\begin{equation}
Total Error = \sum_i^n error_i
\end{equation}
i.e. the sum of all of the errors.  Again, 0 would represent perfect model/data agreement across all tests.

\subsection{Mechanics of Optimization using NeuronUnit}
Here I will describe how I generate these scores concurrently for many parameterizations of the same model and how they guide the optimization path.
I created two different optimization code bases based on the DEAP Python package for genetic optimization \citep{DEAP_JMLR2012}, one that relied on DEAP directly, and one that relied on the DEAP-derived BluePyOpt package produced by The Human Brain Project \citep{bluepyopt} (These have since been merged together), in order to achieve optimization using NeuronUnit.
A few key modules are essential to both approaches.
I wrote the file \emph{optimization-management.py} to contain the logic of and methods for managing complex optimization jobs.
It helps the optimizer handle both fixed and varying  model parameters, contains methods for random sampling of model parameter spaces, can plot models output for visualization of this space, and assists in computing the F-I curve.
I also add several methods for inter-converting between representations of the models themselves and the chromosomes that represent only parameter values.

A created a \emph{NUFeature\_standard\_suite} class to convert NeuronUnit features to BluePyOpt objective functions, as outlined in simpler terms in the enumerated list above.
These classes contain a complicated nesting of fault handling statements, as there are many reasons why a candidate model could return unusable simulation output (typically non-biological parameter values), resulting in values like $NaN$ and $\inf$; such values must be recast as poor but finite errors so that the optimizer can see a smooth error surface.
There are two flavors of \emph{NUFeature\_standard\_suite}, one for supra-threshold simulation experiments and another for at threshold or sub-threshold experiments, since each experiment type produces different feature requiring different feature extractors, and producing different sets of edge cases to be handled independently.
For example, there are more ways for a model to fail to elicit multiple action potentials (causing all ISI-based feature extraction functions to return $NaN$ values), then there are to fail to exhibit a hyperpolarizing response to a small outward current injection for the measurement of input resistance.
 
I created a \emph{model-parameters.py} file, a collection of ordered dictionaries, that informed the optimizer which parameters should be modifiable (in the highest-dimensional cases, all of them) and what are reasonable (biologically plausible) search boundaries.
This file also contains example parameter sets representing notable dynamical regimes, such as those shown in \cite{izhikevich2003simple}.
I also made this file and its methods inter-operable with BluePyOpt model parameter management scheme.

\subsubsection{Optimization Parameters}
Optimization requires searching for better and better solution across multiple generations of chromosomes (parameter sets), as noted in section \ref{sec:genetic-algorithms}.
Robust optimization for the models used here required $NGEN\sim150$ generations with a population size (number of parameter sets explored in each generation) of $\mu=35$.
In other words, it took about $150$ generations of mutation, crossover, and selection to achieve convergence, and in each generation about $35$ models had each of their feautures computed and scored.
These parameter values achieved an acceptable balance between exploration of the parameter space and exploitation of favorable regions.
In some cases, values as small as $NGEN=10$ and $\mu=10$ were tolerable, for example when optimizing only low-dimensional cross-sections of parameter space.
In other cases, such as when the number of optimization objectives (i.e. the number of electrophysiological features being tested) was $NOBJ>25$, values as high as $NGEN=300$ and $\mu=100$ were required to obtain adequate results.

\subsubsection{Multiobjective Scoring and Selection}
One potential scientific goal is to maintain a diverse set of solutions (i.e. very different parameter sets that nonetheless each produce simulations that adequately match observed experimental measurements).
The optimization literature has developed many competing approaches for doing this \citep{deb2000fast}, but it usually involves two popular algorithms, named IBEA and NSGA2, which I investigated here.
NSGA2 uses some additional ranking mechanisms, to re-weight the perceived fitness of each chromosome and influence the probability that it survives (or is bred into) the next generation.
For example, it tries to minimize ``crowding distance", penalizing chromosomes that aggregate in clusters, as persistent cluster formation means that the GA becomes preoccupied with more limited regions of the solution space, harming solution diversity.
Another meta-constraint called ``non-dominated sorting" ranks most highly each chromosomes that is not unanimously defeated by any other chromosome on any feature score.
For example, though one parameter set $P$ might produce a model which score poorly on all features except Input Resistance, if no other parameter set has a higher-scoring Input Resistance feature then $P$ is retained. Consistent with personal communication with \cite{van2007neurofitter}, adding in crowding distance and non-dominated sorting can harm optimizer performance in the context of neuronal model optimization, though the reason for this remain unclear.
A simpler ``select best" algorithm (labelled IBEA) dispenses with these tricks, performs no meta-constraint scoring, and simply retains the fittest chromosomes for mutation, crossover, and selection.

\subsection{Comparison to Previous Approaches to Optimization}

\subsubsection{Time-dependent mutation}
Other labs have previously developed schemes to optimize neuron models, e.g. \cite{druckmann2007novel}.
I retained the conceptual insights of these approaches where they were useful for the problems at hand.
For example, I utilized a time-dependent mutation magnitude ($\eta$).
The idea is that big mutations are more helpful in the early stage of optimization, when it is important to explore the vast hyper-volume of parameter sets and get a general picture of the error surface, and that these mutations should be smaller during the later ``refinement" stage of optimization, as the best solutions are approached.

\subsubsection{Variants on somatic current injection}
Nearly all neuron optimization work (including this one) relies on the responses to somatically-injected current as the domain for optimization.
This is largely motivated by the existence of a common and simple experimental analogue using patch clamp (which drives experimental design for both the Allen Cell Types Database and The Blue Brain Project).
But there are three different strategies for choosing the subset of these experiments that are recapitulated in optimization.

The first strategy involves blindly choosing a fixed set of current amplitudes (e.g. ascending 100 pA steps) from the data and probing the model with these, then comparing model and data within this subset.
This strategy is more direct and slightly more rapid, but it is inefficient in terms of constraining the model, because all of the "action" in the F-I curve occurs above rheobase, but not too far above rheobase (i.e. not at currents that induce depolarization block).

The second strategy involves first identifying the rheobase current, and choosing multiples of or steps around that current for subsequent evaluation. This strategy was used not just in optimization, but also in the analysis and re-organization of existing ephys data.
Action potential waveforms are most regular and consistent in shape when evaluated very near rheobase. 
if the rheobase uniquely determined the entire F-I curve, and if all neurons showed an identical regular, non-accommodating spiking pattern, this strategy would probably be sufficient to identify the rheobase current and move on. To the extent that this (poor) assumption is violated, various additional suprathreshold current injections will be needed. 
If one follow this strategy, one must also decide how many current amplitudes (above rheobase) must be evaluated in order to adequately constrain model parameters.

The third strategy is like the second, but deliberately samples different locations on the F-I curve (not just rheobase).
It attempts to identify the minimum current required to cause some target spike number observed in a dataset. With matching spike counts across simulated and biological data, downstream feature analysis (e.g. spike rate adaptation indices) are likely to be more directly comparable.
Differences in spike shape and spike timing statistics are thus only modulated by differences in model parameters.
The consequences of these decisions are explored in the Results section.

\subsection{Feature Extraction}
Each NeuronUnit test used in the optimizer represents the evaluation of a single feature of simulated output, for example the Input Resistance.
I greatly extended the number of such features/tests covered by NeuronUnit in order to produce a rich, multi-objective optimizer that could capture important spiking dynamics and to obtain insights into what would be the most compact subset for subsequent use in unique identification of models.

\subsubsection{Elephant Features Test Suite}
\label{sec:elephant}
Elephant \citep{elephant18} provides feature extraction capabilities for membrane potential time series expressed using the Neo library in Python \citep{davison_neo}.
Eight NeuronUnit tests (five used here) are derived from Elephant feature extraction, particularly those associated with passive membrane properties assessed with subthreshold stimuli, or action potential waveform properties assessed at rheobase.
Fundamental quantities such as capacitance or input resistance are among these are of  particular interest since they are not "emergent properties" of the model but are roughly predictable from the parameters of the model equations.

\subsubsection{Electrophysiology Feature Extraction Library (EFEL)}
\label{sec:efel}
%Most suprathreshold dynamics were summarized by descriptive statistics of patterns of action potentials. For example, the coefficient of variation of the inter-spike intervals in a spike train can serve as a measure of "burstiness". -- No EFEL has many spike shape qualities. as the noted well as spike train statistics. I'd say the ratio of spike train statistics to spike shape measurements is about 1:1.
The Blue Brain Project developed the Electrophysiology Feature Extraction Library (EFEL) to compute many such statistics from spike trains, and I used these to generate tests of suprathreshold dynamics for optimization \citep{EFEL}.

\subsubsection{Allen Institute Software Development Kit (Allen SDK)}
\label{sec:allensdk}
The Allen Institute offers yet another set of tools for feature extraction, applying to both sub- and suprathreshold features of neuron responses to current injection.
The reason to use these in addition to the above is that some of these features are predicated on particular current stimuli (e.g. a stimulus that is exactly 20 pA stronger than the rheobase current).
Such stimuli either were or were not delivered to the various experimentally recorded neurons, and for the purposes of this thesis there is no going back and delivering additional ones.
Consequently, it makes sense to use features--and the stimuli that generate them--that can be directly compared to the recorded neurons.
For the biological neurons in the Allen Cell Types database, these features are available in the Allen SDK.

