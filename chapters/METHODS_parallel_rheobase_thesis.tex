  

    
    Below I show that for many classes of neuron models using a parallel
Rheobase algorithm adds a significant speed up to model evaluation. It
is important to consider that determinig rheobase usually involves
between 10-35 model simulations at different current injection
strengths. Where is the remaining error score calculations only invole
at most one simulation each. Consider an optimization problem where
there are only eight error scores to calculate, and four of the eight
errors are for non spiking simulations of less than 500ms, a rheobase
simulation is for 1200ms, and it involves both non-spiking and
multi-spiking behavior. Multi-spiking models take longer to simulate
than non spiking models for technical reasons. In this particular
instance where finding the unique rheobase value for each different gene
is important, that rheobase determination is the biggest computational
bottleneck.

It is also important to consider that the speed of model simulation is
often determined by the exact parameterization of the model in question.
During optimization parameterization frequently changes. Multi-spiking
models that are slower to evaluate are frequently sampled by a genetic
algorithm optimizer.

\begin{Verbatim}[commandchars=\\\{\}]
    time taken on block 4.555711984634399
    time taken on block 18.55247473716736
    parallel Rheobase search time NEURON  4.831164836883545
    serial Rheobase search time NEURON  18.69422459602356
    speed up 3.8695066774170974
\end{Verbatim}

\begin{Verbatim}[commandchars=\\\{\}]
    time taken on block 0.7909517288208008
    elapsed serial:  0.7914438247680664
    time taken on block 0.2581171989440918
    elapsed parallel:  0.2590057849884033
    speed up parallel:  3.055699411515084
\end{Verbatim}


    % Add a bibliography block to the postdoc
    
