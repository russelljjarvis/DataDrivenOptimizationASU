

%Blue brain data.
%Real cells conflicts between tests.



\subsection{Section 3.11}
tests that did not work, ThresholdTest, SpikeHalfWidth, Spike Amplitude, as discussed previously this is because of a threshold measurement that differs between cells.


%Aim 1A, write something about tests overall.
%Overall the some 
What tests worked and what didn't 
FITests, Rheobase, Capacitance, Input Resistance, Time Constant test worked but was conflicted. The tests that did not work 

Tests that worked within optimization:
Via \emph{Elephant} toolchain: FITests, Rheobase, Capacitance, Input Resistance, Time Constant, Resting Membrane Potential.
Via. 

When optimizing in the supra threshold regime Druckmann used:
(1) spike rate; (2) an accommodation index; (3) latency to first spike;(4) average AP overshoot; (5)average depth of after hyperpolarization (AHP); 
(6) average AP width similar to Druckman, when optimizing in the supra threshold regime.
When optimizing with reduced models, I found that the those 6 measurements were not enough to tightly constrain a fit, and additional constraints were helpful. In this work a minimum of 12 constraints were typically used:
\emph{EFEL} tool chain:
AHP_depth,all_ISI_values,Spikecount (similar to rate), adaptation_index,
mean_AP_amplitude, min_voltage_between_spikes,minimum_voltage,peak_voltage,spike_half_width
,time_to_first_spike,time_to_last_spike,voltage_base

%3.12


%Restucture Results.


%Move L5PC to 2.a


%Neuroelectro data, room temperature, everything is slower, and spike width is longer

%Everything is slower spike width.

%Include standard error.

% pd.set_precision

%call outs.

%Insert re
%
%Measurement error.


%Call out to figure

%Results

%Discussion


%Multipanel success and failure gallery for multispiking fitting.

%pooled summary measurements, 