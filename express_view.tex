\documentclass{article}
\usepackage[utf8]{inputenc}
\usepackage{graphicx}
\usepackage{enumitem}


\renewcommand{\labelitemi}{$\bullet$}
\renewcommand{\labelitemii}{$\cdot$}
\renewcommand{\labelitemiii}{$\diamond$}
\renewcommand{\labelitemiv}{$\ast$}
\begin{document}

\begin{itemize}
\item[-]  Try to organize your sections more hierarchically, \begin{itemize}

 \item[-] Rather than just a bunch of bullet points for the Introduction, group them by topic, and order them how you think they should be presented to the reader.  
\end{itemize}


\item Also, you will want both a label general introduction and label general discussion, but also within each chapter you will need a mini-introduction that explains what you are aiming to do or show in that chapter and why. 

\item That mini-introduction can be essentially like the content of the first few cells of some of your notebooks, but you will want to make sure that every sentence is comprehensible to someone who has read and understood the general introduction but otherwise doesn’t have much special expertise other than being a neuroscientists of some kind.


% Explanatory comment Name of section, above call to include.
Need to explain two things:
- why optimize, but also why reduced models. Probably why Reduced models first.
\section{Motivation for Reduced Models of electrically excitable cortical cells}

\item[-] Biophysically accurate models take a significant time and resources to evaluate. A different class of neuronal model, known as a "Reduced model" is comparatively fast to solve especially, when many models are required to be simulated simultaneously. Some of examples of reduced models are: 
\subitem AdExp, Izhi, GLIF

\item \section{speed of simulation is important for learning about the brain}. Digital modeling of physical properties of cells is occuring at the mesoscopic scale, however adding microscopic features to simulations significantly increases simulation time. Even when using High Performance Computing a super computer developing simulations involves a model debugging phase and model development and error checking requires simulation output occurs sooner.\\
\\
In the instances when the complete three dimensional form of a neuron is an integral part of a brain simulation, such as in the Blue Brain somatosensory cortex model \cite{} and the Allen Institute $V1$ model \cite{}, These simulations are improved by encasing a "core" of biophysically accurate models inside a "shell" of simple fast and reduced Izhi, GLIF, or AdExp models. 
Encasing a complex core inside a shell of simplified models, solves an "edge effect" problem. The problem is that at the edge of a complex modelled brain region, there is a core of models whose inputs are "extrinsic", or external from the region being modelled. Almost all cortical neurons experience "tonic" synaptic input and these tonic inputs originate from neurons from a different part of the brain. Rather than generating only psuedo random timed inputs to synapses, inputs to V1 also come from a shell of ring neurons. It is there for of interest if these external GLIF models can or should be substituted with optimized Izhikivich and Adexp models, such that even the external shell of the simulation is more realistic.

Furthermore Izhikivitch, GLIF and AdExp models are commonly utilized in neuromorphic archictecture.


\subitem optimization is an interaction between models and constraints which guides a fitting process. 

\subitem if the combination of models and constraints is bad, then then a tractible error surface will not result.  

\subsubitem Unfortunately, it is not always possible to know without trying which combinations of \subitem[A]: neural models, and \subitem[B], constraints will lead to the Genetic Algorithms ability to converge on around the minimal error. 

I describe some code implementation experiments were the model/constraint combination lead to DEAP genetic algorithms matching model parameters to constraints and model/constraint selections that lead to optimizer performing only marginally better than a random search of parameter space\\
\\
\item If the number of dimensions that are searched exceeds the degree and the effectiveness of constraints, then model optimization is only slightly better than random sampling of solution space.

\item \subsection{Successful Optimization} is more likely to come about by a good selection of models and objective function combinations. 
\item How model constraint combinations interact cannot always be known in advance, and the interaction has to be explored experimentally. Experimenting of model test combinations is what was done in this body of work.

Efficient model examples: (generalized leaky integrate and fire model) GLIF, Izhikitch. Adaptive Exponential Integrate and fire model, single compartment conductance based model. 

\item You could make biophysically accurate models faster, or you could make reduced models more accurate. To make reduced models more accurate, you would find parameterizations of the models that let the models act as better mimics of experiments.
\item Herein we investigate how well Faster models can match experimental recording waveform shapes.
\item The reason why we want to investigate the match:
\item Large scale simulations cant evaluate on a timescale that is meaningful, unless a large ratio of modelled cells are "reduced models"
\item Reduced Models already enjoy wide spread usage. We want to investigate if reduced models can be made to be more realistic, by checking if they can mimic data better.
\item If reduced models can't be made more realistic (herein we show only marginal improvements), we need to show the limitations of reduced cells, with regards to a particular set of tests.
\item We need to document the approach used, and how the approach contrasts with spike time approaches to model fitting.
\end{itemize}

\subsection{Optimizing multispiking Behavior}
\begin{itemize}
\item First with $2$ constraints, then with $>100$ constraints via the feature extraction suite EFEL

\item This is a two staged approach. 
\item It uses three step protocols.

\item In a multispiking paradigm you can see a high variance explained ratio of approx $1$.  

\item In this two stage algorithm: I optimize, using constraints: spike count at injection strength $1.5 \times rheobase $, and $3.0 \times rheobase $
I do a quick check of spike counts at 1.5 and 3.0 rheobase (2 constriants only).

\item In this sceanario Rheobase is used only as a soft constraint.  By first the model to spike count data we very rapidly narrow down the solution space using only two uncorrelated errors. In the second notebook I use the standard NU suite, for a lot of generations, you can see a perfect binary match for the passive tests, and a close match for the active tests, but thanks to do things the EFEL way. Our standard suite of tests if only spike-half-width, not spike-base-width. The Izhi models width as thus free to vary at the base, take that into account when eye balling the two graphs and you can see why almost binary match. EFEL does achieve spike width binary matching because it uses both half-width, and base-width. If you look at the last cells you can see I take a correlation matrix of the optimizers errors over its history. The idea is if it's normalized then I can sum the whole matrix and get a single scaler number to show how de-correlated both error sets are over the GA evolution. 

Given that the standard suite of NU tests don't fully constrain the "base spike width" of the waveform which is free to vary, half-spike-width is constrained, it is more appropriate to talk about variance explained of the spike snippet. $variance explained>0.95$ is a useful heurism. Allowing some margin acknowledges that we shouldn't assume we have represented all waveform features that can vary. If you want variance explained ==1   you could optimize using a variance explained cost function, but we don't want to do that. 
\end{itemize}


\section{Introduction}

\subsubsection{Motivation}
\begin{itemize}
Need to explain two things:
- why optimize, but also why reduced models. Probably why models first.

\item Biophysically accurate models take a significant time and resources to evaluate. A different class of neuronal model, known as a "Reduced" model is comparatively fast to solve especially at a large scale. Speed of simulation is important for learning about the brain. Even in the many valid instances when the complete 3D shape is an integral part of a cortical brain simulation, such simulations are often enriched, by encasing a "core" of biophysically accurate models inside a "shell" of simple fast and reduced Izhikitch, GLIF, or adaptive exponential \cite{brette2005adaptive} Wulfram Gerstner models. Further more Izhikivitch, AD

\section{Introduction}

\subsection{Model  Optimization with NeuronUnit}
Some neural properties can’t be easily measured in experiments. These unknown properties hamper modeling accuracy and require parameter fitting. For example, a common approach for approximating unknown ion channel densities is to ‘optimize’ the governing equations to match known waveforms. The process of optimization involves what is known as an ‘inverse’ problem where we efficiently and sparsely search for the ‘optimal’ value of an parameter that satisfies the system of equations. Often an optimal value is corresponds to a global minimum or maximum value of a cost function.\\*

Computational optimization techniques are generally specific to a particular type of problem rather than being generalized. However, several notable algorithms have solved a wide range of problems including genetic and algorithms and stochastic gradient descent (SGD). The popularity of these two algorithms is due to their robustness. NSGA2 and SGD are able to avoid falsely reporting a local minimum when a more optimal solution is available. 
\Subsection{Multiobjective optimization} Multi objective optimization problems are a subset of optimization problems, where model fitness is evaluated against multiple independent constraints, rather than just one error. It is often possible to reduce multiple constraints into one constraint by summing the outputs of objective functions together, however the price of reducing multiple errors into one error, is that where multiple and diverse models give satisfactory solutions to the provided constraint, reducing error leads to a situation where a constraint that is easier to satisfy, rapidly drags down the error score and dominates the by contributing lower errors to the sum of error scores.

However, of SGD and NSGA2, only NSGA2 is a natural choice for tackling multi-objective optimization problems. Default implementations of SGD are not able to utilize the principle of non-domination as an optimization strategy.%%
%%
There is a great diversity of real biological neurons, all of which differ substantially in their electrical behavior. There are a few different classes of general purpose neuronal models, that can reproduce these different types of electrical behaviours, given appropriate parameterizations of the models.\newline
\newline
An exisiting class of neuron model type, called The Izhikevich model was published with parameter sets believed to make the model outputs accurately align with a variety of real biological cell outputs. However since publication much very specific electro physiological recordings have accumulated, that in someways undermine model/experiment agreement. However it is now possible to constrain the Izhikevich model and find new parameterizations that more allow us to more accurately reproduce more recently published experimental data.\newline
\newline
NeuronUnit easily converts a quantitative measure of model/data agreement into a useful error signal. A very natural application of this signal is to guide the process of optimization. We have used Neuronunit to guide optimization by taking a flexible model type such as a generalized linear integrate and fire model or the Izhikevich model and constraining the model against relevant experimental data. As an example, NSGA2 was used to optimize models in conjunction with data driven tests based on pooled data from NeuroElectro.org. A variety of compact and fast single compartment models were used to explore model optimization. Figure 4 demonstrates test error at the beginning of the optimization process for models with randomly sampled parameters and the smaller error following optimization. Figure 5 shows the evolution of the error during the optimization process. \newline
\newline
Optimized neuron models may vary from their neuron counterparts for several reasons. Table 3 shows an example where optimizing the model with respect to the rheobase test comes into conflict with minimizing with respect to input resistance. The solution to the optimization problem consists of two sets of model parameters, which can resolve this conflict differently. Examining the experimental data that these tests were derived from suddenly becomes important. By examining the data, we can see if the rheobase currents and the distributions of input resistance are bi-modal and uniformly distributed. If the data is treated as uni-modal, and the uni-modal mean is used to optimize then the model, then the model is not able to satisfy both constraints simultaneously. In this case, the measurements don’t correspond to neuron data, and the model can’t produce the artificial behavior. When comparing complex data and simple models we find that solutions are better represented using a combination of two optimization solutions.\newline
\newline


\subitem optimization is an interaction between models and constraints which guides a fitting process. 

\subitem if the combination of models and constraints is bad, then then a tractible error surface will not result.  

\subsubitem Unfortunately, it is not always possible to tell without trying which combinations of A: neural models, and B, constraints will lead to the Genetic Algorithms ability to converge on around the minimal error. 


I describe some code implementation experiments were the model/constraint combination lead to DEAP genetic algorithms matching model parameters to constraints and model/constraint selections that lead to optimizer performing no better than a random search of parameter space.


\item successful optimization means model objective choices. How model constraint combinations interact cannot always be known in advance, and the interaction has to be explored experimentally.

Efficient model examples: (generalized leaky integrate and fire model) GLIF, Izhikitch. Adaptive Exponential Integrate and fire model, single compartment conductance based model. 

\item You could make biophysically accurate models faster, or you could make reduced models more accurate. To make reduced models more accurate, you would find parameterizations of the models that let the models act as better mimics of experiments.
\item Herein we investigate how well Faster models can match experimental recording waveform shapes.
\item The reason why we want to investigate the match:
\item Large scale simulations cant evaluate on a timescale that is meaningful, unless a large ratio of modelled cells are "reduced models"
\item Reduced Models already enjoy wide spread usage. We want to investigate if reduced models can be made to be more realistic, by checking if they can mimic data better.
\item If reduced models can't be made more realistic (herein we show only marginal improvements), we need to show the limitations of reduced cells, with regards to a particular set of tests.
\item We need to document the approach used, and how the approach contrasts with spike time approaches to model fitting.
\end{itemize}
\section*{Technical Details of the Optimizer}
\begin{itemize}

\item Experimental Recording Features from Neuroelectro. And some from the Allen SDK.
\item I am using selBest and NSGA2 to optimize currently.
\item Python, NeuroUnit, 

* Model/Test combinations

Contrast with other optimization approaches:

\item Other optimizers, rheobase current injection used is a hard constraint. However, this work differs from other work in that 
\item Rheobase value as a soft constraint. This means that current injection values are a variable model parameter. The exact value of current used is determined via exploring the model response to current injections of varying amplitude. The exact value, its value is determined by other model parameters.

\item spike shape measurements and error functions. We used a set of 8 different experimental measurements NeuronUnit.
\end{itemize}


\begin{itemize}
\item  $N-free-model-parameters << N-constraints$
\item  In our specific design this expands to:
\item  $(model parameters + current-injection-value-parameter) << N (independent and uncorrelated)$ constraints.
\item  For the different classes of Reduced Model we show that the optimizer converges when data is simulated.
\end{itemize}

In a simulated experiment, existing models were instantiated using a randomly chosen model parameters.

In the class of reduced neural models we are optimizing, are not arbitrary waveform generators. The models have intrinsic restrictions that prevent them from matching perfectly with experimental waveforms.

When constraints are derived from model measurements, intrinsic model restrictions no longer apply. Optimized models should match perfectly with the simulated experiments. 

* Failure to match is indicative of: -- Failure to setup tractable optimization problems. Too higher a dimension.

- When inverting linear equations, finding a unique solution requires that the number of constraining equations is greater than the number of free variables you are solving for:

\section{Pitfalls}
Choosing optimizer constraints, that cause visible ripples in error surface.
To protect against a situation where the collection of error sources guiding optimization are too correlated with each other, to act as 


\section{Verification}
Ground truths are model solutions that we know are correct independently from the optimizer. One way to establish ground truths is to identify the global minima by exhaustively searching the solution space. An exhaustive search is a reasonable approach when you consider only one or two model parameters are free parameters, however if one does 100 samples in each of N dimensions then one must make samples $100^{N}$ total samples to be sure of ehaustively searching, assuming the most efficient code, hardware and development time $N=3$, may be the highs.  Also the choice of 100 samples is nominal, 100 samples could be either too fine or too sparse, depending on how if the parameter being searched exhibits 2nd order sensitivity.

It is more computationally efficient to obtain ground truths by simulating constraining data using digital models. It is easy to simulate constraining data, all that is required is that you take a neuronal model and measure its behavior in response to carefuly chosen current injection values. Measured behavior can then be used to construct NeuronUnit tests, were the measurements become "observations", or observed behaviors. To make the simulated data cover a range of circumstances, one can make different NU measurements by randomly choosing different parameter values of models to find.

It was important to be able to establish ground truths that were always possible for the optimizer to match exactly. Often experimental data implies waveform shapes that are beyond the capabilities of the model that is to be fitted. Simulating experimental measurements meant, that model limitations can be understood separately from optimizer limitations.

\section{Optimizer Limitations}


\begin{center}
    \begin{figure}
    
        \includegraphics[width=\linewidth]{figures/correlated_errors.png}
    
    	\caption{Test caption}
    \end{figure}
\end{center}

\subsection{chapter over view}
\begin{itemize}
\item to assess the accuracy of current neural modelling work we explored:
\item variance in models
\item variance in experiments,  
\item variance in the combined set of models and experiments.
\item combine this discussion with model re-purposing discussion.
%\item model re-purposing?

\end{itemize}

%Druckmann \cite{druckmann2008evaluating}

%\cite{buil}


\subsubsection{Cortical Model and Cortical Experiment Agreement}

NeuroML-DB \cite{birgiolas2016rapid} catalogues over 1,500 published models obtained in NeuroML format from Open Source Brain [5]. Complementing OSB, NeuroML-DB provides systematic characterizations of model complexity, electrophysiology, and morphology, making it easy to find, evaluate, and reuse models and their components.\\
\\
It is known that generally that neural models and experimental measurements diverge in some respects,  however, we needed to locate specific sources of divergence. Specific knowledge of model/data disagreement informs the question: "In  cortical neuron models which aspects of a voltage recordings should be prioritized as optimization constraints?"

\subsubsection{Features} Consider a voltage recording at the location of the membrane of a neuron. Teams of researchers have already segmented voltage recordings into labelled sections, each section has a classification that is based on the shape of waveform in a limited region. Rather than specifying by name each measurement it is often useful to refer collectively to these measurable shapes as "features". In a multivariate analysis we analyze hundreds of such features, and we summarize important differences in a subset of this high dimensional feature space. 

Below, I describe some neuronal model features that agreed well with experiments, and some features that diverged.


\subsubsection{Publications Associated with Model Sources}

\begin{table}[ht]
\centering
\resizebox{\textwidth}{!}{
\begin{tabular}{lll}
\toprule
{} & Large Scale Model &   Publication 
Allen Institute V1 Model &  \cite{gouwens2018systematic}
Somatosensory Cortex & \cite{markram2015reconstruction}
\bottomrule
\end{tabular}}
\end{table}

\subsubsection{Feature Extraction Libraries}
\begin{table}[ht]
\centering
\resizebox{\textwidth}{!}{
\begin{tabular}{lll}
\toprule
{} & EFEL Ephys Feature Extraction Library & AllenSDK & Druckmann (2012) 
\bottomrule
\end{tabular}}
\end{table}


\subsubsection{Virtual Experiment Three Step Protocol Stimulate for 2s}

\begin{table}[ht]
\centering
\resizebox{\textwidth}{!}{
\begin{tabular}{lll}
\toprule
{} Injection 1 & Injection 2 & Injection 3 &
 at 1.0×Rheobase & at 1.5×Rheobase & at 3.0×Rheobase 
\bottomrule
\end{tabular}}
\end{table}



%\includegraphics[]{chapters/app_tex/Allen_rush}
\begin{figure}
    \begin{center}
    
    \includegraphics[width=0.6\linewidth]{figures/multi_spiking_large_allen}
    \caption{A voltage recording from a suprathreshold experiment waveform used as a basis for the Allen Brain Institute cell types data base. Publication Gouwens rat \cite{gouwens2018systematic}}

    \end{center}
    
\end{figure}    

\begin{figure}  
    \begin{center}
    
    \includegraphics[width=0.6\linewidth]{figures/multi_spiking_large_bbp}
    \caption{An example of a multispiking waveform used as a basis for the Blue Brain Project. Publication Jouvanile rat \cite{toledo}}


    \end{center}
\end{figure}    



%\end{tcolorbox}
\begin{figure}    
\begin{center}

\includegraphics[width=1.0\linewidth]{figures/cortical_model_data_agreement_52_1.png}
\caption{}

\end{center}
\end{figure}    
\begin{figure}    
\begin{center}
\includegraphics[width=1.0\linewidth]{figures/cortical_model_data_agreement_54_1.png}
\caption{}
\end{center}
\end{figure}    
\cite{wang2019sag}
\begin{itemize}
    \item upstroke\_t\_1.5x allen feature
    \item  peak\_t\_1.5x allen feature
    \item threshold\_t\_1.5x allen feature
    \item fast\_trough\_t\_1.5x allen feature
    \item fast\_trough\_t\_3.0x allen feature
    \item upstroke\_t\_3.0x allen feature
    \item peak\_t\_3.0x allen feature
    \item threshold\_t\_3.0x allen feature
    \item peak\_indices\_1.5x efel feature
    \item min\_AHP\_indices\_1.5x efel feature
\end{itemize}


\begin{itemize}

    \item fast\_trough\_index\_1.5x allen feature
    \item fast\_trough\_index\_3.0x allen feature
    \item threshold\_index\_1.5x allen feature
    \item peak\_index\_1.5x allen feature
    \item upstroke\_index\_1.5x allen feature
    \item peak\_index\_3.0x allen feature
    \item upstroke\_index\_3.0x allen feature
    \item threshold\_index\_3.0x allen feature
\end{itemize}





\end{document}
